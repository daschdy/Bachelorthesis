\chapter{Theoretische Grundlagen}

\section{Didaktische Simulatoren in der Lehre - Relevanz und Herausforderungen}

\section{Konzepte und Lernziele}


"Law of Effect" (deutsch: \textit{Gesetz der Wirkung oder Effektgesetz}) des amerikanischen Psychologen Edward L. Thorndike. Sie zählt zu den Grundlagen wissenschaftlicher Lerntheorien, insbesondere im Bereich des Behaviorismus und der operanten Konditionierung.

Einordnung Simulator als E-Learning?

Gamification

M-Learning

Was ist ein Simulator

Blended Learning

Blended Learning~2.0 an Bedeutung: Klassische Präsenzformate wurden stärker mit digitalen, hybriden Komponenten verbunden, wodurch flexible, personalisierbare Lernsettings entstanden.\parencite{bonk2020}

immersive Technologien

Learning Analytics 2-0

\begin{itemize}
    \item Gamification
    \item Gamified Learning
    \item 4 Modi vom Lernen: Aktiv, Passiv, konstruktiv, interaktiv
\end{itemize}

\section{Definition und Arten von Simulatoren}
