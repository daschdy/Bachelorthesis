\begin{abstract}

Die vorliegende Bachelorarbeit untersucht systematisch den Einsatz didaktischer Simulatoren in der Lehre der Rechnerarchitektur. Diese Werkzeuge dienen dazu, komplexe und abstrakte Inhalte anschaulich zu vermitteln, indem sie Konzepte visualisieren, Interaktivität ermöglichen und unmittelbares Feedback bieten

Ziel der Arbeit war es, durch eine systematische Literaturrecherche relevante Simulatoren und wissenschaftliche Beiträge umfassend zu erfassen, anhand eines Kriterienkatalogs zu analysieren und dadurch Best Practices sowie aktuelle Trends abzuleiten. Die Methodik umfasst eine systematische Literaturrecherche in IEEE Xplore, ACM Digital Library und weiteren Quellen sowie eine ergänzende Erhebung veröffentlichter Simulatoren. Insgesamt wurden 151 Publikationen und 57 bereits veröffentlichte Simulatoren identifiziert und nach thematischen Clustern (u. a. Prozessoren und Architekturen, Speicher und Performance, Hardware und Logik) sowie weiteren didaktischen Merkmalen ausgewertet.

Die Ergebnisse zeigen deutliche Schwerpunkte bei Prozessorarchitekturen, insbesondere im Kontext von RISC, sowie eine wachsende Relevanz von GPUs, KI-basierten Ansätzen und immersiven Technologien. Didaktisch wirksam sind vor allem reduzierte, klar strukturierte Darstellungen nach Prinzipien der Cognitive Theory of Multimedia Learning. Als zentrale Erfolgsfaktoren erweisen sich kostenfreie Verfügbarkeit, plattformübergreifende Nutzung und eine umfassende Dokumentation. Gamification-Elemente sind zwar selten integriert, bieten jedoch ein hohes Potenzial zur Steigerung der Motivation.

Auf dieser Basis werden Empfehlungen für die zukünftige Gestaltung didaktischer Simulatoren abgeleitet: Stärkere Integration adaptiver und spielerischer Elemente, konsequente didaktische Reduktion, hybride Zugriffsmodelle und kleine Lerneinheiten.

\TODO{weiter ausführen, wenn fazit da}
\end{abstract}