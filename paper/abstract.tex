\begin{abstract}
Die vorliegende Bachelorarbeit untersucht systematisch den Einsatz didaktischer Simulatoren in der Lehre der Rechnerarchitektur. Ziel war es, relevante Simulatoren und wissenschaftliche Beiträge zu erfassen, anhand eines Kriterienkatalogs zu analysieren und daraus Best Practices sowie aktuelle Trends abzuleiten. Grundlage ist eine systematische Literaturrecherche in IEEE Xplore, ACM Digital Library und weiteren Quellen sowie eine ergänzende Erhebung veröffentlichter Simulatoren. Insgesamt wurden 151 Publikationen und 57 Simulatoren identifiziert und nach thematischen Clustern sowie didaktischen Merkmalen ausgewertet.  

Die Ergebnisse zeigen deutliche Schwerpunkte bei Prozessorarchitekturen, insbesondere im Kontext von \acs{RISC}, während GPUs, KI-gestützte Ansätze und immersive Technologien zunehmend an Bedeutung gewinnen. Didaktisch wirksam sind vor allem reduzierte Darstellungen nach Prinzipien der Cognitive Theory of Multimedia Learning (CTML). Erfolgsfaktoren sind eine kostenfreie, plattformübergreifende Verfügbarkeit und eine verlässliche Dokumentation. Gamification wird bislang selten umgesetzt, bietet jedoch Potenzial zur Steigerung der Motivation.  

Die Diskussion verdeutlicht, dass empirische Wirksamkeitsstudien unter realen Lehrbedingungen fehlen und der Kriterienkatalog ausgeweitet werden sollte.  

Das Fazit lautet, dass Simulatoren gezielt für kurze, klar strukturierte Lerneinheiten konzipiert und durch motivierende Elemente ergänzt werden sollten. Künftige Forschung sollte Feldstudien durchführen, objektive Qualitätsmaßstäbe entwickeln und den schulischen Bereich stärker berücksichtigen. Damit liefert die Arbeit sowohl konsolidierte Gestaltungsprinzipien für die Praxis als auch Impulse für die zukünftige Forschung.
\end{abstract}