\documentclass[
germanthesis
]{i3thesis}
% options:
% [germanthesis] - Thesis is written in German
% [plainunnumbered] - Don't print numbers on plain pages
% [earlydraft] - Settings for quick draft printouts
% [watermark] - Print current time/date at bottom of each page
% [phdthesis] - switch to PhD thesis style
% [twoside] - double sided

\usepackage{multirow}
\usepackage{floatflt}
%\usepackage[babel=true]{csquotes}
%\usepackage[backend=biber,style=ieee]{biblatex}

%% CUSTOM
% ------------------------------
% for figures
\usepackage{caption}
\usepackage{subcaption}
\usepackage{tocloft}

% tables
\usepackage{longtable}
\usepackage{pdflscape}
\usepackage{xurl}

% subchapters in capital
\newcommand{\sh}[1]{
  \vspace{1em}
  \noindent\MakeUppercase{#1}\\
}

% acronyms
\newcommand{\listofacronyms}{%
  \chapter*{\glossarytitlename}%
  \addcontentsline{toc}{chapter}{\glossarytitlename}%
  \begin{acronym}[XXXX]
  \end{acronym}%
}
% ------------------------------

\author{Dustin Heither}
\title{Best Practices und Trends bei didaktischen Simulatoren für Rechnerarchitektur:
Eine systematische Literaturrecherche}
\thesistype{Bachelorarbeit}
\thesiscite{Bachelor's Thesis~(Bachelorarbeit)}
\birthday{10. Mai 1992}
\birthplace{Oberhausen}
\thesisstart{30. April 2025}
\advisors{M.Sc. Tobias Baumeister}

\addbibresource{thesis.bib}
\graphicspath{{images/},{pictures/}}

\begin{document}
\pagenumbering{roman}

\maketitle
\acresetall

\cleardoublepage
\tableofcontents

\cleardoublepage
\listofacronyms
\TODO{Alphabetisch sortieren}

\begin{acronym}[1234567890]
    \acro{FPGA}{Field Programmable Gate Array}
    \acro{OER}{Open Educational Resources}
    \acro{MOOC}{Massive Open Online Course}
    \acro{AR}{Augmented Reality}
    \acro{VR}{Virtual Reality}
    \acro{MR}{Mixed Reality}
    \acro{XR}{Extended Reality}
    \acro{CTML}{Cognitive Theory of Multimedia Learning}
    \acro{ELT}{Experiential Learning Theory}
\end{acronym}


\cleardoublepage
\pagenumbering{arabic}

\chapter{Einleitung}

\sh{Motivation}
Die nachfolgende Bachelorarbeit behandelt das Thema "Best Practices und Trends bei didaktischen Simulatoren für Rechnerarchitektur: Eine systematische Literaturrecherche". Sie wurde am Lehrstuhl für Rechnerarchitektur der Friedrich-Alexander-Universität Erlangen-Nürnberg im Sommersemester 2025 geschrieben und bildet die abschließende Prüfungsleistung des Bachelorstudiums Informatik.

Studien zeigen, dass die Erfolgsquote von Studienanfänger:innen in naturwissenschaftlichen Studiengängen deutlich geringer ist als in anderen Fachrichtungen \parencite[S.~370]{burdinski_lehrvideos_2024}. Laut dem aktuellen MINT-Nachwuchsbarometer 2024 lag die Abbruchquote im Jahr 2022 bei 50,5\,\%, Tendenz steigend. Die Hintergründe für diese Studienabbrüche betitelt das MINT-Nachwuchsbarometer als vielfältig. Die wesentlichen Ursachen liegen jedoch in einer Diskrepanz zwischen den Erwartungen und Interessen der Studierenden und den tatsächlichen Studieninhalten sowie in unzureichenden fachlichen Voraussetzungen \parencite[S.~21]{joachim_herz_stiftung_mint_2024}.

Gerade im Bereich der Rechnerarchitektur stellen komplexe und abstrakte Inhalte eine besondere Hürde dar \parencite[S.~1]{grober_championship_2022}. Didaktische Simulatoren können hier ansetzen, indem sie schwer fassbare Konzepte anschaulich machen, den Lernenden unmittelbares Feedback ermöglichen und so den Zugang zu den Inhalten erleichtern \parencite[S.~11]{zeichner_using_2020}. 

Weitere Studien verdeutlichen, dass Simulatoren das Verständnis der internen Funktionsweise von Rechnern verbessern \parencite[S.~215]{prasad_using_2015}, von Studierenden überwiegend positiv bewertet werden \parencite[S.~8]{besim_understanding_2012} und durch spielerische Ansätze zusätzliche Motivation erzeugen können \parencite[S.~453]{schlag_gamifizierung_2021}. Auch aktuelle Evaluierungen visueller CPU-Simulatoren belegen den didaktischen Nutzen solcher Werkzeuge \parencites[S.~11]{maxnuck_soares_use_2016}[S.~75]{cortinovis_further_2024}. Dadurch tragen didaktische Simulatoren potenziell dazu bei, die Motivation und den Lernerfolg zu steigern und langfristig Studienabbrüche zu reduzieren \parencite[S.~272]{kornelsen_expedition_2005}.

\sh{Zielsetzung}
Ziel dieser Arbeit ist es, didaktische Simulatoren mit Relevanz für die Lehre in der Rechnerarchitektur systematisch zu erfassen und vergleichbar zu machen. Hierzu werden sowohl bereits veröffentlichte als auch wissenschaftlich erläuterte Simulatoren anhand definierter Kriterien untersucht und kategorisiert. Auf dieser Grundlage entsteht eine strukturierte Übersicht, die die Vielfalt der Werkzeuge sowie ihre spezifischen Einsatzgebiete sichtbar macht.

Darüber hinaus werden aktuelle Trends und Entwicklungen identifiziert, um einen zeitgemäßen Überblick über den Stand der Forschung und Lehre in diesem Bereich zu geben. Dies umfasst die Analyse der historischen Entwicklung didaktischer Simulatoren sowie eine Clustering nach definierten Kriterien. Aus diesen Analysen werden schließlich zentrale Entwicklungen und Best Practices abgeleitet.

\sh{Aufbau der Arbeit}
Die Arbeit gliedert sich in sechs Kapitel, die inhaltlich aufeinander aufbauen. Nach der Einleitung, in der Motivation und Zielsetzung dargelegt werden, folgt im zweiten Kapitel die Methodik. Dort wird beschrieben, wie relevante Simulatoren identifiziert wurden, welche Kriterien zur Analyse herangezogen werden und in welchem Umfang die Literaturrecherche und Simulatorrecherche abgegrenzt ist. Im dritten Kapitel werden die theoretischen Grundlagen gelegt: Es werden die Relevanz und Herausforderungen didaktischer Simulatoren in der Lehre diskutiert, zentrale Konzepte wie etwa Gamification vorgestellt und verschiedene Definitionen sowie Typen von Simulatoren erläutert. Zusätzlich geht das Kapitel auf lernpsychologische Theorien ein und stellt die Entwicklungen der Rechnerarchitektur als Teilgebiet der Informatik dar.

Das vierte Kapitel widmet sich den Ergebnissen der Literatur- und Simulatorrecherche. Dabei wird die historische Entwicklung didaktischer Simulatoren nachgezeichnet und die erfassten Publikationen werden sowohl thematisch nach Inhalten der Rechnerarchitektur als auch nach Simulatortypen und weiteren Kriterien analysiert. Darauf aufbauend erfolgt in Kapitel~\ref{chap:5-discussion} eine Diskussion der Ergebnisse. Neben der Einordnung aktueller Trends und Herausforderungen werden dort Best Practices sowie Empfehlungen für zukünftige Entwicklungen abgeleitet.

Den Abschluss bildet Kapitel sechs, das die wichtigsten Erkenntnisse zusammenfasst, die zentrale Zielsetzung der Arbeit reflektiert und einen Ausblick auf weiterführende Forschungsansätze gibt.

\chapter{Methodik}

\section{Recherche und Auswahl der Simulatoren}

Die Suche erfolgte nach dem PRISMA-Schema und fand in den Datenbanken Google Scholar,
SpringerLink, IEEEXPlore, ERIC und Scopus statt

\section{Kriterienkatalog}

Tabelle~\ref{tab:kriterien} fasst die für die Analyse herangezogenen Kriterien zusammen.

Ein zentrales Kriterium ist der \textit{Zugriff}. Je nach organisatorischem Kontext ist entscheidend, ob ein Simulator online oder offline verfügbar ist. Während eine Online-Lösung eine Internetverbindung erfordert und dadurch eingeschränkt nutzbar sein kann, setzt ein Offline-Simulator häufig eine Installation voraus, die wiederum von Betriebssystem oder Hardwareanforderungen abhängt.

Die \textit{Programmiersprache} ist insbesondere für Studierende der Informatik von Relevanz. Sie kann die Möglichkeit eröffnen, Erweiterungen oder Plugins zu entwickeln und den Simulator an spezifische Bedürfnisse anzupassen.

Unter dem Kriterium \textit{Simulatorart} wird der Abstraktionsgrad eingeordnet, also ob der Simulator didaktisch reduziert oder realitätsnah ist. Ergänzend werden auch die Visualisierung und der Grad der Interaktivität berücksichtigt.

Das Kriterium \textit{Zielgruppe} erlaubt eine Zuordnung zu unterschiedlichen Bildungskontexten (z. B. Schule, Hochschule) und gibt Auskunft darüber, ob und in welchem Umfang Vorkenntnisse erforderlich sind.

Für Lehrende spielt zudem der \textit{Preis} eine zentrale Rolle. Es wird unterschieden, ob es sich um freie oder lizenzpflichtige Software handelt, da der Kostenfaktor maßgeblich über den praktischen Einsatz entscheidet.

Ein weiterer Aspekt ist die \textit{Gamification}. Studien zeigen, dass spielerische Elemente die Lernmotivation steigern können. Unterschieden werden hier insbesondere levelbasierte und storytelling-orientierte Ansätze.\parencite[S. 106f]{sailer_gamification_2020}\parencite[S. 13]{baah_enhancing_2024}

Da die Rechnerarchitektur eine Vielzahl von \textit{Themenbereichen} abdeckt, wird erfasst, auf welchen inhaltlichen Schwerpunkt sich ein Simulator bezieht (z. B. Pipelining oder Caching).

Unter \textit{Beschäftigungsdauer} wird die zu erwartende Nutzungszeit betrachtet. Dabei ist relevant, ob die Dauer der Auseinandersetzung mit dem Simulator die notwendige Einführungs- und Erklärungszeit übersteigt.

Die Qualität der \textit{Dokumentation} ist ein weiteres zentrales Kriterium. Eine klare und umfassende Dokumentation erleichtert Installation, Anwendung und Problemlösung und wirkt sich direkt auf die Nutzbarkeit des Simulators aus.

Da die Arbeit auch Trends und Entwicklungen berücksichtigt, werden zudem der \textit{Bekanntheitsgrad}, der \textit{Wartungsstand} (letztes Update) sowie das Jahr der \textit{Veröffentlichung} aufgenommen, um den Stellenwert eines Simulators einzuordnen und veraltete Software zu identifizieren.

\begin{table}

\centering
\caption{Kriterienkatalog}
\label{tab:kriterien}
\small

\begin{tabular}{|l|l|l|}
\hline

\textbf{Nr.} & \textbf{Kriterien} & \textbf{Kategorie} \\
\hline

\multirow{4}{*}{1} & \multirow{4}{*}{Zugriff} & Offline \\
                   &                          & Betriebssystem \\
                   &                          & Online \\
                   &                          & Hardwareanforderungen \\
\hline

\multirow{3}{*}{2} & \multirow{3}{*}{Programmiersprache} & Entwicklungssprache \\
                   &                                     & Embedding von Programmiersprachen \\
                   &                                     & Integration in Lernplattformen \\
\hline

\multirow{3}{*}{3} & \multirow{3}{*}{Simulatorart} & Abstraktionsgrad \\
                   &                               & Visualisierung \\
                   &                               & Interaktivität \\
\hline

\multirow{2}{*}{4} & \multirow{2}{*}{Zielgruppe} & Institution \\
                   &                             & Vorkenntnisse \\
\hline

\multirow{2}{*}{5} & \multirow{2}{*}{Preis} & Freeware \\
                   &                        & Open Source \\
\hline

\multirow{2}{*}{6} & \multirow{2}{*}{Gamification} & Level \\
                   &                               & Storytelling \\
\hline

\multirow{2}{*}{7} & \multirow{2}{*}{Themenbereich} & Pipelining \\ 
                   &                                & Caching \\
\hline

\multirow{3}{*}{8} & \multirow{3}{*}{Beschäftigungsdauer} & kurzfrisitg \\
                   &                                      & mittelfristig \\
                   &                                      & langfrisitig \\ 
\hline

\multirow{2}{*}{9} & \multirow{2}{*}{Dokumentation} & Verfügbarkeit \\
                   &                                & Zugriff \\
\hline

10 & Bekanntheitsgrad & Skala von 0 - 10 \\
\hline

11 & Wartungsstand & letztes Update \\
\hline

12 & Veröffentlichung & Aktualität \\
\hline

\end{tabular}
\end{table}

\section{Limitierungen}

\chapter{Theoretische Grundlagen}

\section{Didaktische Simulatoren in der Lehre - Relevanz und Herausforderungen}

\section{Konzepte und Lernziele}


"Law of Effect" (deutsch: \textit{Gesetz der Wirkung oder Effektgesetz}) des amerikanischen Psychologen Edward L. Thorndike. Sie zählt zu den Grundlagen wissenschaftlicher Lerntheorien, insbesondere im Bereich des Behaviorismus und der operanten Konditionierung.

Einordnung Simulator als E-Learning?

Gamification

M-Learning

Was ist ein Simulator

Blended Learning

Blended Learning~2.0 an Bedeutung: Klassische Präsenzformate wurden stärker mit digitalen, hybriden Komponenten verbunden, wodurch flexible, personalisierbare Lernsettings entstanden.\parencite{bonk2020}

immersive Technologien

Learning Analytics 2-0

\begin{itemize}
    \item Gamification
    \item Gamified Learning
    \item 4 Modi vom Lernen: Aktiv, Passiv, konstruktiv, interaktiv
\end{itemize}

\section{Definition und Arten von Simulatoren}

\chapter{Ergebnisse der Literaturrecherche}\label{results}

\section{Ergebnisse aus der Literaturrecherche}\label{chap:results_lit}

\sh{Allgemeine Informationen}
Insgesamt umfasst die Analyse 151 Veröffentlichungen, die in Tabelle~\ref{tab:literatur} aufgeführt sind. Abbildung~\ref{fig:1-veroeffentlichungen-jahr} zeigt die jährliche Verteilung der Publikationen. Von den insgesamt 151 Arbeiten wurden 58 im Zeitraum von 2020 bis 2025 veröffentlicht, was einem Anteil von ca. 38~\% entspricht, während weitere 55~\% der Publikationen in den Jahren 2000 bis 2020 erschienen.

\begin{figure}[!htbp]
    \centering
    \includegraphics[width=0.90\textwidth]{graphics_lit/1-veroeffentlichungen-jahr.png}
    \caption{Übersicht der Veröffentlichungen pro Jahr}
    \label{fig:1-veroeffentlichungen-jahr}
\end{figure}

Aus Abbildung~\ref{fig:2-typ} ist ersichtlich, dass insgesamt 98~\% der Veröffentlichungen auf Journalartikel (ca. 36~\%) und Konferenzbeiträge (ca. 62~\%) entfallen. Während Journalartikel in Fachzeitschriften mit ausführlicherem Begutachtungsprozess erscheinen, werden Konferenzbeiträge überwiegend in Tagungsbänden veröffentlicht und dienen der schnellen Verbreitung aktueller Forschungsergebnisse \cite{abbadia_conference_2022}. Beide Publikationstypen sind daher für eine Trendanalyse sowie zur Ableitung von Best Practices geeignet.

\begin{figure}[!htbp]
    \centering
    % --- linke Seite: Grafik ---
    \begin{subfigure}[b]{0.48\textwidth}
        \centering
        \includegraphics[width=1\textwidth]{graphics_lit/2-typ.png}
        \caption{Übersicht des Typs} 
        \label{fig:2-typ}
    \end{subfigure}
    \hfill
    % 
    % --- rechte Seite: Tabelle ---
    \begin{subfigure}[b]{0.48\textwidth}
        \centering
        \tiny
        \begin{tabularx}{\textwidth}{X c}
            \hline
            \multicolumn{2}{l}{\textbf{Journals}} \\
            \hline
            IEEE Transactions on Education & 10 \\
            Computer Applications in Engineering Education & 7 \\
            Journal on Educational Resources in Computing & 7 \\
            \hline
            \multicolumn{2}{l}{\textbf{Konferenzen}} \\
            \hline
            Workshop on Computer Architecture Education & 8 \\
            ACM Technical Symposium on Computer Science Education & 7 \\
            Annual International Symposium on Computer Architecture & 6 \\
            \hline
        \end{tabularx}
        \caption{Meistgenutzte Journals und Konferenzen}
        \label{tab:2-typ-detail}
    \end{subfigure}
    %
    \caption{Informationen zum Typ der Publikationen}
    \label{fig:2-typ-gesamt}
\end{figure}

Tabelle~\ref{tab:2-typ-detail} bietet eine Übersicht über die Zeitschriften und Konferenzen, in denen die meisten der untersuchten Publikationen veröffentlicht wurden. Die drei genannten Zeitschriften vereinen 44~\% aller Journalartikel, während die drei aufgeführten Konferenzen rund 22~\% der Konferenzbeiträge ausmachen.

\sh{Chronologische Entwicklungen}
Abbildung~\ref{fig:3-anzahl-themen} verdeutlicht, dass die häufigsten Themen der untersuchten Publikationen \enquote{Prozessoren und Architekturen} (44~\%), \enquote{Speicher und Performance} (11~\%) sowie \enquote{Hardware und Logistik} (10~\%) sind. Diese drei Themen umfassen zusammen rund 70~\% aller Publikationen. Dicht darauf folgen die Themen \enquote{Grundlagen und Theorien} sowie \enquote{Programmierung} mit jeweils 9~\% der Arbeiten.

\begin{figure}[!h]
    \centering
    \includegraphics[width=0.90\textwidth]{graphics_lit/3-thema.png}
    \caption{Anzahl der Veröffentlichungen pro Thema}
    \label{fig:3-anzahl-themen}
\end{figure}

Eine detaillierte Untersuchung der drei am häufigsten vertretenen Themen ist in Abbildung~\ref{fig:4-top3-themen} dargestellt. Diese Grafik zeigt die Verteilung dieser Themen über verschiedene Zeitspannen. Die zugrunde liegenden Werte sind in der Tabelle~\ref{tab:themen-zeit} aufgeführt.

Im Themenbereich \enquote{Prozessoren und Architekturen} sind die Subthemen \textit{CPU}, \textit{MIPS}, \textit{Mikroprozessor}, \textit{Prozessor} und \textit{RISC} in annähernd gleicher Häufigkeit vertreten. Auch wenn diese Einordnung zunächst klassische Architekturkonzepte widerspiegelt, zeigt sich insbesondere beim Subthema \textit{RISC} eine aktuelle Relevanz in der Rechnerarchitektur.

Wie in Kapitel~\ref{chap:3-3-development_ca} beschrieben, erfährt das \ac{RISC}-Paradigma durch die weite Verbreitung von ARM-basierten Systemen sowie durch die zunehmende Bedeutung von energieeffizienten Architekturen eine neue Aktualität. Somit spiegelt sich in den didaktischen Simulatoren ein Themenfeld wider, das auch in den jüngsten Entwicklungen der Rechnerarchitektur von zentraler Bedeutung ist.

Obwohl die Themenbereiche \enquote{AI} und \enquote{GPU} nur einen geringen Anteil der untersuchten Publikationen ausmachen (etwa 5~\%), ist erkennbar, dass entsprechende Simulatoren zu 88~\% in Veröffentlichungen nach 2020 thematisiert werden. Diese Beobachtung lässt sich in den Kontext zentraler Entwicklungen der Rechnerarchitektur einordnen: Große Technologieunternehmen investieren verstärkt in die Forschung zu \ac{AI}, während GPUs im Zuge paralleler Datenverarbeitung zunehmend an Bedeutung gewinnen. Auch in der Entwicklung didaktischer Simulatoren (vgl. Kapitel~\ref{chap:3-2-development-sim}) wird in jüngerer Zeit der Einsatz von \ac{AI} als didaktisches Instrument diskutiert.

\begin{figure}[!htbp]
    \centering
    % --- linke Seite: Grafik ---
    \begin{subfigure}[b]{0.48\textwidth}
        \centering
        \includegraphics[width=0.90\textwidth]{graphics_lit/4-top3-themen-jahr.png}
        \caption{Jährliche Aufteilung Top 3 Themen (grafisch)}
        \label{fig:4-top3-themen}
    \end{subfigure}
    \hfill
    % 
    % --- rechte Seite: Tabelle ---
    \begin{subfigure}[b]{0.48\textwidth}
        \centering
        \tiny
        \begin{tabularx}{\textwidth}{lXXX}
            \hline
            \textbf{Zeitraum} & \textbf{Hardware und Logik} & \textbf{Prozessoren und Architekturen} & \textbf{Speicher und Performance} \\
            \hline
            vor 2000      & 0  & 6  & 2 \\
            2000--2010    & 5  & 19 & 7 \\
            2010--2020    & 3  & 27 & 4 \\
            nach 2020     & 7  & 22 & 4 \\
            \hline
        \end{tabularx}
        \caption{Jährliche Aufteilung Top 3 Themen (detailliert)}
        \label{tab:themen-zeit}
    \end{subfigure}
    %
    \caption{Jährliche Aufteilung Top 3 Themen}
    \label{fig:pub-typen}
\end{figure}

Für die Themen \enquote{Grundlagen und Theorien} sowie \enquote{Programmierung} zeigt Abbildung~\ref{fig:5-top5-themen} die jährliche Verteilung. Die entsprechenden Werte sind in Tabelle~\ref{tab:themen-zeit-2} aufgeführt. Auf eine detaillierte Analyse der verbleibenden Themen wir hier verzichtet.

Die Themenbereiche \enquote{Grundlagen und Theorien}, \enquote{Programmierung} sowie \enquote{Speicher und Performance} zeigen in Bezug auf die zeitliche Entwicklung keine besonderen Auffälligkeiten. Über die Zeiträume \enquote{vor 2000}, \enquote{2000 -- 2010}, \enquote{2010 -- 2020} und \enquote{nach 2020} sind diese Themen relativ gleichmäßig verteilt. Das Themenfeld \enquote{Monitoring} ist hingegen so gering vertreten, dass hierzu keine belastbaren Aussagen getroffen werden können.

\begin{figure}[!htbp]
    \centering
    % --- linke Seite: Grafik ---
    \begin{subfigure}[b]{0.48\textwidth}
        \centering
        \includegraphics[width=0.90\textwidth]{graphics_lit/5-top5-themen-jahr.png}
        \caption{Jährliche Aufteilung weitere Themen (grafisch)}
        \label{fig:5-top5-themen}
    \end{subfigure}
    \hfill
    % 
    % --- rechte Seite: Tabelle ---
    \begin{subfigure}[b]{0.48\textwidth}
        \centering
        \tiny
        \begin{tabularx}{\textwidth}{lXX}
            \hline
            \textbf{Zeitraum} & \textbf{Grundlagen und Theorien} & \textbf{Programmierung} \\
            \hline
            vor 2000      & 0 & 2 \\
            2000--2010    & 5 & 4 \\
            2010--2020    & 3 & 4 \\
            nach 2020     & 6 & 4 \\
            \hline
        \end{tabularx}
        \caption{Jährliche Aufteilung weitere Themen (detailliert)}
        \label{tab:themen-zeit-2}
    \end{subfigure}
    %
    \caption{Jährliche Aufteilung weitere Themen}
    \label{fig:pub-typen}
\end{figure}

Verbleibend ist die zeitliche Analyse des Themenbereichs \enquote{Systeme und Anwendungen}, der einen Anteil von 8~\% an den gesamten wissenschaftlichen Publikationen ausmacht. Innerhalb dieses Clusters wird im Wesentlichen \ac{VR} behandelt. Immersive Technologien finden zunehmend auch in modernen didaktischen Simulatoren Anwendung und sind seit den 2010er-Jahren als Lern- und Lehrmethode erkennbar (vgl. Kapitel~\ref{chap:3-2-development-sim}). Die im Rahmen der Literaturrecherche identifizierten Publikationen zu \ac{VR}-Simulatoren stammen überwiegend aus dem Zeitraum 2020 bis 2025.

\sh{Gamification}
Hinsichtlich der Frage, ob die untersuchten Simulatoren spielerische Elemente enthalten, bietet die Tabelle~\ref{tab:gamification} einen Überblick.

\begin{table}[!htbp]
    \centering
    \begin{tabular}{l r r}
        \hline
        \textbf{Gamification} & \textbf{Anzahl} & \textbf{\%} \\
        \hline
        Keine Elemente     & 145 & 96\% \\
        Elemente vorhanden & 6   & 4\%  \\
        \hline
        \textbf{Summe}     & 151 & 100\% \\
        \hline
    \end{tabular}
    \caption{Verteilung der Publikationen in Bezug auf enthaltene Gamification-Elemente}
    \label{tab:gamification}
\end{table}

Wie in Kapitel~\ref{chap:3-2-development-sim} erläutert, stellt Gamification ein zentrales Konzept didaktischer Lehr- und Lernsysteme dar. Aus Tabelle~\ref{tab:gamification} geht hervor, dass lediglich 4~\% der untersuchten Publikationen über didaktische Simulatoren der Rechnerarchitektur Gamification-Elemente berücksichtigen.

Das Konzept wird in den Themenbereichen \enquote{Grundlagen und Theorien}, \enquote{Prozessoren und Architekturen} sowie \enquote{Systeme und Anwendungen} aufgegriffen und in Publikationen zwischen 2007 und 2024 behandelt. Die zeitliche Verortung dieser Arbeiten entspricht der bereits dargestellten allgemeinen Veröffentlichungstendenz und liefert daher keine zusätzlichen Erkenntnisse.  

\sh{Abstraktionslevel}
Von den 151 Simulatoren der untersuchten wissenschaftlichen Publikationen wurden 15~\% als realitätsnah und 85~\% als didaktisch reduziert eingestuft. Die Verteilung bezogen auf die einzelnen Themenbereiche ist in Abbildung~\ref{fig:7-abstraktion-themen} dargestellt, wohingegen die Abbildung~\ref{fig:8-abstraktion-jahr} die zeitliche Entwicklung des Abstraktionslevels verdeutlicht.

\begin{figure}[!htbp]
    \centering
    % --- linke Seite: Grafik ---
    \begin{subfigure}[b]{0.48\textwidth}
        \centering
        \includegraphics[width=0.90\textwidth]{graphics_lit/7-abtraktion-themen.png}
        \caption{Verteilung Abstraktionslevel auf Themen}
        \label{fig:7-abstraktion-themen}
    \end{subfigure}
    \hfill
    % 
    % --- rechte Seite: Grafik ---
    \begin{subfigure}[b]{0.48\textwidth}
        \centering
        \includegraphics[width=0.90\textwidth]{graphics_lit/8-abstraktion-jahr.png}
        \caption{Zeitliche Entwicklung des Abstraktionslevels}
        \label{fig:8-abstraktion-jahr}
    \end{subfigure}
    %
    \caption{Analysen zum Abstraktionslevel}
    \label{fig:abstaktion-analysen}
\end{figure}

Für den Einsatz von Simulatoren als Lehr- und Lernmethode ist das Abstraktionsniveau beziehungsweise der Grad der didaktischen Reduktion von zentraler Bedeutung. Mit Hinblick auf die \ac{CTML} nach Mayer, die auf den drei Kernprinzipien \textit{Dual-Channel Processing}, \textit{Limited Capacity} und \textit{Active Processing} basiert, lässt sich ableiten, dass realitätsnahe Simulatoren tendenziell zu einer Überlastung der verfügbaren kognitiven Ressourcen führen können. Die Vielzahl simultaner Reize und irrelevanter Details in hochrealistischen Umgebungen erschwert die Selektion und Organisation relevanter Informationen und reduziert dadurch die Effektivität des Lernprozesses. Demgegenüber unterstützen didaktisch reduzierte Simulationen die Einhaltung zentraler \ac{CTML}-Prinzipien und erweisen sich insbesondere in den Bildungskontexten \enquote{Schule} und \enquote{Hochschule} als lernförderlicher \parencites[S.~169]{tremblay_task_2023}[S.~955]{haji_thrive_2016}[S.~360]{reedy_using_2015}~\footnote{Hinweis: Die wissenschaftlichen Quellen, die diese Aussage bestätigen, behandeln didaktische Simulatoren im medizinischen Kontext.}.

Eine detailliertere Betrachtung der vorgestellten didaktischen Simulatoren zeigt, dass 99~\% der Hochschulbildung zugeordnet werden können. Damit wird ersichtlich, dass in der Hochschullehre die Prinzipien der \ac{CTML} Anwendung finden und Studierende nicht durch übermäßig komplexe Aufgabenstellungen oder Simulatoren überlastet werden.

Über nahezu alle Themenbereiche hinweg sind realitätsnahe Simulatoren in weniger als 30~\% der Fälle vertreten. Lediglich in den Bereichen \enquote{AI} und \enquote{GPU} treten sie häufiger auf.

Der Zeitverlauf in Abbildung~\ref{fig:8-abstraktion-jahr} zeigt keine besonderen Auffälligkeiten. Die kumulierte Entwicklung realitätsnaher und didaktisch reduzierter Simulatoren entspricht dem bereits zuvor beschriebenen zeitlichen Verlauf.

\sh{Institutionen}
Anhand von Abbildung~\ref{fig:9-institution} ist zu erkennen, dass 89~\% der Publikationen die Hochschulbildung als Zielgruppe adressieren. 9~\% der untersuchten Simulatoren sind für Forschung und Beruf bestimmt, während sich die verbleibenden 2~\% auf schulische Bildung und Weiterbildungen aufteilen.

Untersucht man die Verteilung der Zielgruppen genauer, so zeigt Abbildung~\ref{fig:10-institution-themen} die Aufteilung der einzelnen Zielgruppen (\enquote{Schule}, \enquote{Hochschule}, \enquote{Forschung, Beruf}, \enquote{Weiterbildung}) auf die jeweiligen Themenbereiche. Dabei wird deutlich, dass sich zum Beispiel alle GPU-bezogenen Simulatoren der Forschungs- bzw. Berufsgruppe zuordnen lassen.

\begin{figure}[!htbp]
    \centering
    % --- linke Seite: Grafik ---
    \begin{subfigure}[b]{0.48\textwidth}
        \centering
        \includegraphics[width=0.90\textwidth]{graphics_lit/9-institution.png}
        \caption{Aufteilung Institutionen}
        \label{fig:9-institution}
    \end{subfigure}
    \hfill
    % 
    % --- rechte Seite: Grafik ---
    \begin{subfigure}[b]{0.48\textwidth}
        \centering
        \includegraphics[width=0.90\textwidth]{graphics_lit/10-institution-themen.png}
        \caption{Aufteilung Institutionen nach Themen}
        \label{fig:10-institution-themen}
    \end{subfigure}
    %
    \caption{Analysen zu Institutionen}
    \label{fig:institution-analysen}
\end{figure}

\sh{Zugriff}
Die Literaturrecherche hat ergeben, dass der überwiegende Teil der vorgestellten Simulatoren offline einsetzbar ist (vgl. Abbildung~\ref{fig:11-zugriff}). Lediglich 22~\% der untersuchten Publikationen verfolgen einen Online-Ansatz\footnote{Hier wurden die Kategorien \enquote{online} und \enquote{on- und offline} zusammengefasst.}.

Die Abbildung~\ref{fig:13-zugriff-thema} zeigt die Verteilung der verschiedenen Zugriffsarten in Bezug auf die einzelnen Themenbereiche.

\begin{figure}[!htbp]
    \centering
    % --- linke Seite: Grafik ---
    \begin{subfigure}[b]{0.48\textwidth}
        \centering
        \includegraphics[width=0.90\textwidth]{graphics_lit/11-zugriff.png}
        \caption{Aufteilung Zugriff}
        \label{fig:11-zugriff}
    \end{subfigure}
    \hfill
    % 
    % --- rechte Seite: Grafik ---
    \begin{subfigure}[b]{0.48\textwidth}
        \centering
        \includegraphics[width=0.90\textwidth]{graphics_lit/13-zugriff-thema.png}
        \caption{Zugriffsart pro Thema}
        \label{fig:13-zugriff-thema}
    \end{subfigure}
    %
    \caption{Analysen zu zum Zugriff}
    \label{fig:zugriff-analysen}
\end{figure}

Die zeitliche Verteilung der verschiedenen Zugriffsarten ist in Abbildung~\ref{fig:12-zugriff-jahr} dargestellt, die entsprechenden Werte sind in Tabelle~\ref{tab:zugriff-zeit} aufgeführt.

\begin{figure}[!htbp]
    \centering
    % --- linke Seite: Grafik ---
    \begin{subfigure}[b]{0.48\textwidth}
        \centering
        \includegraphics[width=\textwidth]{graphics_lit/12-zugriff-jahr.png}
        \caption{Jährliche Aufteilung Zugriff (grafisch)}
        \label{fig:12-zugriff-jahr}
    \end{subfigure}
    \hfill
    % 
    % --- rechte Seite: Tabelle ---
    \begin{subfigure}[b]{0.48\textwidth}
        \centering
        \tiny
        \begin{tabularx}{\textwidth}{lXXXX}
            \hline
            \textbf{Zeitraum} & \textbf{online} & \textbf{offline} & \textbf{keine Info} & \textbf{on-/offline} \\
            \hline
            vor 2000      & 0  & 9  & 1 & 0 \\
            2000--2010    & 6  & 25 & 5 & 4 \\
            2010--2020    & 2  & 28 & 5 & 8 \\
            nach 2020     & 8  & 37 & 8 & 5 \\
            \hline
        \end{tabularx}
        \caption{Jährliche Aufteilung Zugriff (detailliert)}
        \label{tab:zugriff-zeit}
    \end{subfigure}
    %
    \caption{Darstellung der Zugriffsarten pro Zeitraum}
    \label{fig:zugriff-gesamt}
\end{figure}

Abbildung~\ref{fig:zugriff-gesamt} verdeutlicht, dass der Anteil der online nutzbaren Simulatoren in den Jahren nach 2010 höher liegt als in den Jahren davor. Dabei ist jedoch zu berücksichtigen, dass 64~\% der betrachteten Publikationen ebenfalls aus dem Zeitraum nach 2010 stammen.

Der in Kapitel~\ref{chap:3-2-development-sim} beschriebene Effekt des \textit{M-Learning} ab den 2000er-Jahren kann daher nicht eindeutig bestätigt werden. Simulatoren, die eine spezifische Hardware wie beispielsweise \ac{FPGA}s voraussetzen, werden in dieser Untersuchung ebenfalls der Kategorie \enquote{offline} zugeordnet, auch wenn der eigentliche Simulator potenziell unabhängig davon wäre.

\sh{Preis}
Aufgrund des steigenden finanziellen Drucks auf Studierende~\parencite[S.~1]{meier_bedeutung_2023} und auf deutsche Hochschulen~\cite{von_stuckrad_hochschulfinanzierung_nodate}, die wesentliche Zielgruppe der Simulatoren darstellen, erscheint der Preis aus Sicht von Lehrenden und Lernenden als zentrales Kriterium.


\begin{figure}[!htbp]
    \centering
    \includegraphics[width=0.9\textwidth]{graphics_lit/14-preis.png}
    \caption{Aufteilung Preis}
    \label{fig:14-preis2}
\end{figure}

\begin{figure}[!htbp]
    \centering
    % --- linke Seite: Grafik ---
    \begin{subfigure}[b]{0.48\textwidth}
        \centering
        \includegraphics[width=\textwidth]{graphics_lit/15-preis-jahr.png}
        \caption{Aufteilung Preis (grafisch)}
        \label{fig:15-preis-jahr}
    \end{subfigure}
    \hfill
    % --- rechte Seite: Tabelle ---
    \begin{subfigure}[b]{0.48\textwidth}
        \centering
        \tiny
        \begin{tabularx}{\textwidth}{lXXX}
            \hline
            \textbf{Zeitraum} & \textbf{kostenlos} & \textbf{kostenpflichtig} & \textbf{k.A.} \\
            \hline
            vor 2000      & 4  & 0 & 6  \\
            2000--2010    & 19 & 4 & 17 \\
            2010--2020    & 26 & 3 & 14 \\
            nach 2020     & 37 & 4 & 17 \\
            \hline
        \end{tabularx}
        \caption{Aufteilung Preis (detailliert)}
        \label{tab:15-preis-zeit}
    \end{subfigure}
    %
    \caption{Darstellung der Preisgestaltung der Simulatoren über verschiedene Zeiträume}
    \label{fig:15-preis-gesamt}
\end{figure}

Abbildung~\ref{fig:15-preis-jahr} zeigt die Verteilung zwischen kostenlosen und kostenpflichtigen Simulatoren. 57~\% werden als kostenlos und 7~\% als kostenpflichtig angegeben. Für den verbleibenden Anteil von 36~\% enthalten die untersuchten Publikationen keine Angaben zur Preisgestaltung. Eine nähere Betrachtung der kostenpflichtigen Simulatoren zeigt, dass diese nahezu gleichmäßig auf die Themenbereiche \enquote{Hardware und Logik} sowie \enquote{Prozessoren und Architekturen} entfallen. Insbesondere bei Subthemen wie \ac{FPGA} oder dem \textit{Einsatz echter Hardware} sind die initialen Kosten zu berücksichtigen.

Der zeitliche Verlauf in Abbildung~\ref{fig:15-preis-jahr} bzw. Tabelle~\ref{tab:15-preis-zeit} entspricht dem bereits zuvor beschriebenen allgemeinen Muster. Hieraus lassen sich daher keine eigenständigen Trends oder neuen Erkenntnisse ableiten.

\sh{Dokumentation}
In Bezug auf das Kriterium \textit{Dokumentation} gibt Abbildung~\ref{fig:16-dokumentation} genaue Einblicke über die Veteilung. In 87~\% der analysierten Publikationen verfügen die beschriebenen Simulatoren über eine Dokumentation, während 5~\% keine Dokumentation erwähnen und in 7~\% der Publikationen hierzu keine Angaben vorliegen.

\begin{figure}[!htbp]
    \centering
    \caption{Aufteilung Dokumentation}
    \label{fig:16-dokumentation}
    \includegraphics[width=0.90\textwidth]{graphics_lit/16-dokumentation.png}
\end{figure}

Die Bedeutung einer begleitenden Dokumentation für didaktische Simulatoren lässt sich auf mehrere lernpsychologische Theorien zurückführen, die in Kapitel~\ref{chap:3-1-psychology} ausführlich dargestellt werden:

\begin{itemize}
    \item \textit{\ac{CTML}}: Eine Dokumentation unterstützt das Kernprinzip \textit{Active Processing}, da Lernende neue Informationen mit vorhandenem Wissen verknüpfen und erweitern können.
    \item \textit{Exploratives Lernen}: Eine Dokumentation dient als Leitfaden im selbstgesteuerten Lernprozess.
    \item \textit{Erfahrungsbasiertes Lernen}: In der Phase der \textit{reflektierenden Beobachtung} ermöglicht eine Dokumentation, gemachte Erfahrungen aufzuarbeiten, einzuordnen und in neue Konzepte zu überführen. Sie wirkt dabei unterstützend wie eine Anleitung.
\end{itemize}

Abbildung~\ref{fig:16-dokumentation} verdeutlicht, dass die Mehrheit der untersuchten didaktischen Simulatoren über eine entsprechende Dokumentation verfügt, sodass die die Relevanz aus lernpsychologischer Sicht unterstützt wird.

\sh{Anzahl Zitationen}
Als abschließendes Kriterium wird die Anzahl der Zitationen untersucht. Tabelle~\ref{tab:zitationen} zeigt die am häufigsten zitierten Publikationen pro Thema, da diese für weiterführende oder aufbauende Untersuchungen in den jeweiligen Bereichen als maßgeblich gelten. Die Auswahl der dargestellten Publikationen erfolgte auf Grundlage des Durchschnitts aller Zitationen, der bei etwa 35 liegt. Publikationen mit mehr als 35 Zitationen werden nachfolgend dargestellt.

{
\tiny
\centering
\begin{longtable}{|c|p{6cm}|p{3cm}|c|}
    \caption{Häufig zitierte Publikationen pro Themengebiet\label{tab:zitationen}} \\
    \hline
    \textbf{Zitationen} & \textbf{Titel} & \textbf{Autor(en)} & \textbf{Jahr} \\
    \hline
    \endfirsthead

    \hline
    \textbf{Zitationen} & \textbf{Titel} & \textbf{Autor(en)} & \textbf{Jahr} \\
    \hline
    \endhead

    \hline
    \multicolumn{4}{l}{Fortsetzung auf der nächsten Seite} \\
    \hline
    \endfoot

    \hline
    \endlastfoot

    % --- Inhalte ---
    \multicolumn{4}{c}{\textbf{AI}} \\
    \hline
    211 & Teaching CS50 with AI: Leveraging Generative Artificial Intelligence in Computer Science Education & R. Liu, C. Zenke, C. Liu, A. Holmes, P. Thornton, D. J. Malan & 2024 \\
    \hline
    \multicolumn{4}{c}{\textbf{GPU}} \\
    \hline
    146 & MGPUSim: Enabling Multi-GPU Performance Modeling and Optimization & Y. Sun, T. Baruah, S. A. Mojumber, S. Dong, X. Gong, S. Treadway & 2019 \\
    397 & Accel-Sim: An Extensible Simulation Framework for Validated GPU Modeling & M. Khairy, Z. Shen, T. M. Aamodt, T. G. Rogers & 2020 \\
    \hline
    \multicolumn{4}{c}{\textbf{Grundlagen und Theorien}} \\
    \hline
    117 & Flexible Web-Based Educational System for Teaching Computer Architecture and Organization & J. Djordjevic, B. Nikolic, A. Milenkovic & 2005 \\
    \hline
    \multicolumn{4}{c}{\textbf{Hardware und Logik}} \\
    \hline
    46 & Harnessing FPGAs for Computer Architecture Education & M. Holland, J. Harris, S. Hauck & 2003 \\
    \hline
    \multicolumn{4}{c}{\textbf{Programmierung}} \\
    \hline
    41 & Using Simulators for Teaching Computer Organization and Architecture & P. W. C. Prasad, A. Alsadoon, A. Beg, A. Chan & 2016 \\
    51 & MarieSim: The MARIE Computer Simulator & L. Null, J. Lobur & 2003 \\
    \hline
    \multicolumn{4}{c}{\textbf{Prozessoren und Architekturen}} \\
    \hline
    162 & Control Flow Modeling in Statistical Simulation for Accurate and Efficient Processor Design Studies & L. Eeckhout, R. H. Bell, B. Stougie, K. De Bosschere, L. K. John & 2004 \\
    172 & Applying a Constructivist and Collaborative Methodological Approach in Engineering Education & L. Moreno, C. Gonzalez, I. Castilla, E. Gonzalez, J. Sigue & 2007 \\
    327 & Measuring Experimental Error in Microprocessor Simulation & R. Desikan, D. Burger, S. W. Keckler & 2001 \\
    \hline
    \multicolumn{4}{c}{\textbf{Speicher und Performance}} \\
    \hline
    53 & Cryogenic Computer Architecture Modeling with Memory-Side Case Studies & L. Gyu-Hyeon, M. Dongmoon, B. Ilkwon, K. Jangwoo & 2019 \\
    205 & A Simulation Based Study of TLB Performance & A. Borg, J. B. Chen, N. P. Jouppi & 1992 \\
    \hline
    \multicolumn{4}{c}{\textbf{Systeme und Anwendungen}} \\
    \hline
    139 & Virtual Reality in Computer Science Education: A Systematic Review & J. Priker, A. Dengel, M. Holly, S. Safikani & 2020 \\
    \hline
\end{longtable}
}


\section{Ergebnisse aus Tab.~\ref{tab:simulatoren}}


\chapter{Best Practices, Trends und Diskussion}\label{chap:5-discussion}

\section{Trends und Best Practices}

\TODO{Einleitender Satz}

\subsection{Trends und Best Practices in wissenschaftlichen Publikationen}

Aus Abbildung~\ref{fig:3-anzahl-themen} und Abbildung~\ref{fig:4-top3-themen} wird ersichtlich, dass die Themenbereiche \enquote{Prozessoren und Architekturen} sowie \enquote{Hardware und Logik} in den vergangenen 15 Jahren besonders häufig in wissenschaftlichen Publikationen zu didaktischen Simulatoren der Rechnerarchitektur aufgegriffen wurden.  

Im Themenbereich \enquote{Prozessoren und Architekturen} sind die Subthemen \textit{CPU}, \textit{MIPS}, \textit{Mikroprozessor}, \textit{Prozessor} und \textit{RISC} in annähernd gleicher Häufigkeit vertreten. Auch wenn diese Einordnung zunächst klassische Architekturkonzepte widerspiegelt, zeigt sich insbesondere beim Subthema \textit{RISC} eine aktuelle Relevanz in der Rechnerarchitektur.  

Wie in Kapitel~\ref{chap:3-3-development_ca} beschrieben, erfährt das \ac{RISC}-Paradigma durch die weite Verbreitung von ARM-basierten Systemen sowie durch die zunehmende Bedeutung von energieeffizienten Architekturen eine neue Aktualität. Somit spiegelt sich in den didaktischen Simulatoren ein Themenfeld wider, das auch in den jüngsten Entwicklungen der Rechnerarchitektur von zentraler Bedeutung ist.


Obwohl die Themenbereiche \enquote{AI} und \enquote{GPU} nur einen geringen Anteil der untersuchten Publikationen ausmachen (etwa 5~\%), ist erkennbar, dass entsprechende Simulatoren zu 88~\% in Veröffentlichungen nach 2020 thematisiert werden. Diese Beobachtung lässt sich in den Kontext zentraler Entwicklungen der Rechnerarchitektur einordnen: Große Technologieunternehmen investieren verstärkt in die Forschung zu \ac{AI}, während GPUs im Zuge paralleler Datenverarbeitung zunehmend an Bedeutung gewinnen. Auch in der Entwicklung didaktischer Simulatoren (vgl. Kapitel~\ref{chap:3-2-development-sim}) wird in jüngerer Zeit der Einsatz von \ac{AI} als didaktisches Instrument diskutiert.




\include{5-x-discussion}


\chapter{Fazit}

Ziel dieser Arbeit war es, den Stand didaktischer Simulatoren für die Lehre der Rechnerarchitektur systematisch zu erfassen, vergleichbar zu machen und daraus Best Practices sowie Entwicklungsperspektiven abzuleiten. Durch die Analyse von 151 Publikationen und 57 veröffentlichten Simulatoren konnten zentrale thematische und didaktische Muster identifiziert werden.

Die Ergebnisse verdeutlichen klare Schwerpunkte: Prozessoren und Architekturen, insbesondere \acs{RISC}, bilden nach wie vor den Kernbereich der didaktischen Simulatoren. Zugleich gewinnen GPU-, KI-bezogene und immersive Ansätze zunehmend an Bedeutung und spiegeln die aktuellen technologischen Entwicklungen der Rechnerarchitektur wider. Damit lassen sich sowohl eine starke Kontinuität klassischer Inhalte als auch erste Verschiebungen hin zu neuen Themenfeldern beobachten.

Didaktisch wirksam sind vor allem reduzierte Darstellungen im Sinne der \acl{CTML}, da sie komplexe Sachverhalte auf wesentliche Kernelemente verdichten. Gamification wird bislang nur vereinzelt integriert, obwohl die Literatur deutliche Potenziale zur Steigerung der Lernmotivation belegt. Häufig genannte Erfolgsfaktoren sind darüber hinaus eine ortsunabhängige Nutzung (online oder hybrid), die Kostenfreiheit, plattformübergreifende Verfügbarkeit sowie eine umfassende und verlässliche Dokumentation. Zusammengenommen führen diese Befunde zu Empfehlungen für die zukünftige Gestaltung, Bereitstellung und didaktische Nutzung von Simulatoren.

Aus der Diskussion ergeben sich zwei zentrale Schlussfolgerungen: Erstens sollten Simulatoren gezielt für kleine, klar strukturierte Lerneinheiten konzipiert werden, die den Lernprozess schrittweise begleiten und bei Bedarf durch spielerische Elemente ergänzt werden können. Zweitens besteht eine deutliche Forschungslücke in Bezug auf belastbare empirische Studien, die den Einsatz dieser Werkzeuge in realen Lehrkontexten untersuchen. Zukünftige Arbeiten sollten daher (1) genauere und weniger subjektive Maße zur Einschätzung der Relevanz nutzen, (2) die Nutzungsdauer und Interaktionsmuster als Indikatoren für Qualität und Motivation erfassen und (3) den schulischen Bildungsbereich stärker berücksichtigen, da Simulatoren bislang vorwiegend in der Hochschullehre verankert sind.

Darauf aufbauend ergibt sich eine Forschungsagenda, die von methodisch fundierten Feldstudien bis hin zur Entwicklung klarer Kategorien reicht, mit denen sich die Unterschiede zwischen \enquote{Gamified Learning} und \enquote{Game-Based Learning} systematisch erfassen lassen. Für die Praxis bedeutet dies, dass Lehrende bereits heute auf eine Reihe Gestaltungsprinzipien zurückgreifen können, während die Forschung in den kommenden Jahren verstärkt Evidenz für Wirksamkeit und Transferfähigkeit liefern sollte. Auf diese Weise trägt die Arbeit sowohl zur theoretischen Fundierung als auch zur praktischen Weiterentwicklung didaktischer Simulatoren im Bereich der Rechnerarchitektur bei.



\cleardoublepage
\clearpage

\listoffigures
\clearpage

\listoftables
\clearpage

\appendix
\chapter{Relevante Tabellen}

\input{paper/app-lit-overview}

\clearpage

\begin{landscape}
\fancyhf{} 
\renewcommand{\headrulewidth}{0pt} 
\fancyfoot[C]{\thepage} 
\tiny

\TODO{Simulatoren Codierung}

% https://tableconvert.com/csv-to-latex
\begin{longtable}{|c|p{1cm}|p{1cm}|p{1cm}|p{1cm}|p{1cm}|p{1cm}|p{1cm}|p{1cm}|p{1cm}|p{1cm}|p{1cm}|p{1cm}|p{1cm}|p{1cm}|p{1cm}|p{1cm}|p{1cm}|p{1cm}|}
    \caption{Ergebnisse Simulatorrecherche} \label{tab:simulatoren} \\
    \hline
    \textbf{ID} & \textbf{Name} & \textbf{Entwickler} & \textbf{Beschreibung} & \textbf{Zugriff} & \textbf{Betriebssystem} & \textbf{Programmiersprache} & \textbf{Simulatorart} & \textbf{Zielgruppe} & \textbf{Preis} & \textbf{Gamification} & \textbf{Themenbereich} & \textbf{Vorwissen} & \textbf{Beschäftigungsdauer} & \textbf{Dokumentation} & \textbf{Bekanntheitsgrad} & \textbf{Veröffentlichung} & \textbf{Wartungsstand} & \textbf{Quelle} \\
    \hline
    \endfirsthead

    % Kopf auf den Folgeseiten
    \hline
    \textbf{ID} & \textbf{Name} & \textbf{Entwickler} & \textbf{Beschreibung} & \textbf{Zugriff} & \textbf{Betriebssystem} & \textbf{Programmiersprache} & \textbf{Simulatorart} & \textbf{Zielgruppe} & \textbf{Preis} & \textbf{Gamification} & \textbf{Themenbereich} & \textbf{Vorwissen} & \textbf{Beschäftigungsdauer} & \textbf{Dokumentation} & \textbf{Bekanntheitsgrad} & \textbf{Veröffentlichung} & \textbf{Wartungsstand} & \textbf{Quelle} \\
    \hline
    \endhead

    % Fortsezung
    \hline
    \multicolumn{19}{l}{Fortsetzung nächste Seite} \\
    \hline
    \endfoot

    \hline
    \multicolumn{19}{l}{Ende der Tabelle} \\
    \hline
    \endlastfoot

    1 & Cache-Simulator & Holger Morgenstern & Anleitung zur Implementierung eines Cache-Simulators & Offline & Unabhängig & Java & didaktisch reduziert & Hochschule & Open Source & Nein & Cache & Wissen zu Caches werden während der Implementierung vermittelt & 1-2 Monate & vorhanden & niedrig & 2001 & 2001 & https://www.morgenstern.net/Cache/ \\ \hline
    2 & MMIX & Donald Knuth & 64-Bit-RISC Modellprozessor & Offline & Windows, MacOS, Linux & C & didaktisch reduziert & Hochschule, Forschung & Open Source & Nein & RISC & Assembler, Register, Speicher & 1-12 Stunden & vorhanden & hoch & 2011 & 2011 & https://mmix.cs.hm.edu/index.html \\ \hline
    3 & MikroSim & H. Peter Gumm, Martin Perner & Funktionsweise eines Mikroprozessors (CPU) & Offline & Windows & Visual Basic & didaktisch reduziert & Hochschule, kommerzielle Nutzung & Lizenz & Nein & Mikroprozessor & Aufbau CPU (ALU, Register, Steuerwerk), Speicherhierarchien, Mikrobefehle & 1-12 Stunden & vorhanden & hoch & 1992 & 2012 & http://www.mikrocodesimulator.de/ \\ \hline
    4 & Bonsai Computer & Klaus Merkert, Johannes Merkert & Funktionsweise eines Computer (ohne ALU) & Offline, Web & Unabhängig & JavaScript & didaktisch reduziert & Schule & Open Source & Nein & Grundlagen & Nein & 1-12 Wochen & vorhanden & hoch & 2004 & 2017 & https://bonsai.inf-schule.de/ \\ \hline
    5 & MurmelRechner & Felix Selter & Funktionsweise eines Computer (ohne ALU) & Web & Unabhängig & JavaScript & didaktisch reduziert & Schule & Open Source & Nein & Grundlagen & Nein & 30 - 60 Minuten & vorhanden & hoch & 2021 & 2022 & https://inf-schule.de/rechner/bonsai/murmelrechner \\ \hline
    6 & Johnny & Peter Dauscher & Funktionsweise eines Rechners nach der Von-Neumann-Architektur & Offline, Web & Unabhängig & Free Pascal/ JavaScript & didaktisch reduziert & Schule & Open Source & Nein & Rechnerarchitektur, von-Neumann-Architektur, & Nein & 30 - 60 Minuten & vorhanden & hoch & 2014 & 2021 & https://dev.inf-schule.de/content/12\_rechner/4\_johnny/johnny3/ \\ \hline
    7 & MOPS & Marco Haase & Funktionsweise eines Rechners nach der Von-Neumann-Architektur & Offline & Windows, MacOS, Linux & Python & didaktisch reduziert & Schule & Kostenlos & Nein & Rechnerarchitektur, von-Neumann-Architektur, & Nein & 30 - 60 Minuten & vorhanden & hoch & 2009 & 2024 & http://www.viktorianer.de/info/mops.html \\ \hline
    8 & LogiSim & Diverse (Github), ursprünglich Carl Burch & Ein Werkzeug zur Gestaltung und Simulation digitaler Schaltungen & Offline, Web & Unabhängig & Java/ JavaScript & didaktisch reduziert & Hochschule, Forschung & Open Source & Nein & Schaltungstechnik & Digitale Logik & 30 - 60 Minuten & vorhanden & hoch & 2001 & 2025 & https://cburch.com/logisim/de/index.html \\ \hline
    9 & CPUlator & Henry Wong & Simulator und Debugger für vollständige Computersysteme & Web & Unabhängig & C++, WebAssembly, JavaScript & didaktisch reduziert & Hochschule & Kostenlos & Nein & Assembler, CPU-Instruction Sets & Assembler, CPU, Speicher, Register & 30 - 60 Minuten & vorhanden & mittel & 2016 & 2024 & https://cpulator.01xz.net/ \\ \hline
    10 & RARS & TheThirdOnes (Github) & Ein Assembler + Runtime Simulator für RISC-V & Offline & Unabhängig & Java & didaktisch reduziert & Hochschule & Open Source & Nein & RISC-V Assembler & Grundkenntnisse & 30 - 60 Minuten & vorhanden & hoch & 2017 & 2023 & https://github.com/TheThirdOne/rars \\ \hline
    11 & MARS & Pete Sanderson, Ken Vollmar & IDE für MIPS Assembler Programmierung & Offline & Unabhängig & Java & didaktisch reduziert & Hochschule & Open Source & Nein & MIPS Assembler & Grundkenntnisse & 30 - 60 Minuten & vorhanden & hoch & 2014 & 2014 & https://dpetersanderson.github.io/ \\ \hline
    12 & gem5 & gem5 (Github) & Modularer Simulator für Rechnerarchitektur & Offline & Linux, MacOs, Docker & C++, Python & didaktisch reduziert & Hochschule, Forschung & Open Source & Nein & System- und Mikroarchitektur & Rechnerarchitektur, Betriebssysteme, Assembler & 1-12 Wochen & vorhanden & hoch & 2011 & 2025 & https://www.gem5.org/ \\ \hline
    13 & QEMU & Qemu (Gitlab) & Generischer und Open-Source-Maschinenemulator und Virtualisierung & Offline & Windows, MacOS, Linux & C & realitätsnah & Forschung & Open Source & Nein & Betriebssysteme, Systemarchitektur & Grundkenntnisse & 1-12 Wochen & vorhanden & hoch & 2003 & 2025 & https://www.qemu.org/ \\ \hline
    14 & QtSpim & James Larus & MIPS32-Simulator & Offline & Windows, MacOS, Linux & C++ & didaktisch reduziert & Hochschule & Open Source & Nein & Assembler, Computerarchitektur & Grundkenntnisse & 1-12 Stunden & vorhanden & hoch & 1990 & 2023 & https://spimsimulator.sourceforge.net/ \\ \hline
    15 & ARMSim\# & Horspool, Lyons, Serra & Simulator für die ARM7TDMI-Architektur & Offline & Windows, MacOS, Linux & C\# & didaktisch reduziert & Hochschule & Open Source & Nein & Assembler, Computerarchitektur & Grundkenntnisse & 1-12 Stunden & vorhanden & hoch & 2009 & 2015 & https://webhome.cs.uvic.ca/\~nigelh/ARMSim-V2.1/index.html \\ \hline
    16 & HDLBits & Henry Wong & Eine Sammlung Schaltungsdesign-Übungen zur Anwendung von HDL mit Verilog & Online & Unabhängig & JavaScript & didaktisch reduziert & Hochschule & Kostenlos & Nein & HDL & Grundkenntnisse in digitaler Logik und Verilog & 1-12 Stunden & vorhanden & hoch & 2016 & 2024 & https://hdlbits.01xz.net/wiki/Main\_Page \\ \hline
    17 & ASMBits & Henry Wong & Eine Sammlung Assemblerübungen mit sofotigem Feedback & Online & Unabhängig & JavaScript & didaktisch reduziert & Hochschule & Kostenlos & Nein & Assembler & Grundkenntnisse & 1-12 Stunden & vorhanden & hoch & 2016 & 2024 & https://asmbits.01xz.net/wiki/Main\_Page \\ \hline
    18 & Iverilog Simulator & Henry Wong & Verilog-Simulationen direkt im Browser auszuführen & Online & Unabhängig & JavaScript & didaktisch reduziert & Hochschule & Kostenlos & Nein & Verilog & Grundkenntnisse & 1-12 Stunden & vorhanden & hoch & 2016 & 2024 & https://hdlbits.01xz.net/wiki/Iverilog \\ \hline
    19 & Ripes & Ripes (Github) & RISC-V Simulator & Online & Unabhängig & C++ & didaktisch reduziert & Hochschule & Kostenlos & Nein & RISC-V & Grundkenntnisse & 1-12 Stunden & vorhanden & hoch & 2018 & 2025 & https://ripes.me/ \\ \hline
    20 & DineroIV & Jan Edler, Mark D. Hill & Cache Simulator & Offline & Linux & C & didaktisch reduziert & Hochschule & Open Source & Nein & Cache & Grundkenntnisse & 1-12 Stunden & vorhanden & hoch & 1998 & 2023 & https://pages.cs.wisc.edu/\~markhill/DineroIV/ \\ \hline
    21 & Pintos & Ben Pfaff & Betriebssystem-Framework für x86-Architektur & Offline & Docker, VM & C & didaktisch reduziert & Hochschule & Kostenlos & Nein & Betriebssystem & Grundkenntnisse & 1-12 Wochen & vorhanden & hoch & 2004 & 2025 & https://pkuflyingpig.gitbook.io/pintos \\ \hline
    22 & Bochs & Kevin Lawton (ursprünglich), Bochs & X86 Emulator & Offline & Windows, MacOS, Linux & C++ & realitätsnah & Hochschule, Forschung & Open Source & Nein & Betriebssystem & Grundkenntnisse & 1-12 Wochen & vorhanden & hoch & 2011 & 2025 & https://bochs.sourceforge.io/ \\ \hline
    23 & NachOS & Thomas Anderson, Wayne A. Christopher, Steven J. Procter & NachOS ist ein Betriebssystem-Framework & Offline & VM & C++ & didaktisch reduziert & Hochschule & Kostenlos & Nein & Betriebssystem & Grundkenntnisse in C++ & 1-12 Wochen & vorhanden & hoch & 1992 & 1996 & https://homes.cs.washington.edu/\~tom/nachos/ \\ \hline
    24 & Multi2Sim & Universität Valencia, Northeastern University & Multi2Sim ist ein modularer, cycle-akkurater Simulator für heterogene Systeme, der CPUs, GPUs und deren Interaktionen modelliert & Offline & Windows, MacOS, Linux & C++ & realitätsnah & Hochschule, Forschung & Kostenlos & Nein & GPU, Rechensystem & Fortgeschritten & 1-12 Stunden & vorhanden & hoch & 2011 & 2013 & http://www.multi2sim.org/ \\ \hline
    25 & Sniper & Ursprünglich von Trevor Carlson, Wim Heirman, Lieven Eeckhout u. a.); heute Github + Forschungsgruppen. & Ein paralleler, hochskalierbarer und relativ schneller Cycle-timing/interval-basierter Multi-Core-Simulator für x86 & Offline & Docker, Linux & C++ & realitätsnah & Hochschule, Forschung & Open Source & Nein & Multi-core Performance-Analyse, Speicherhierarchie-Evaluation & Fortgeschritten & 1-12 Wochen & vorhanden & mittel & 2011 & 2024 & https://snipersim.org/w/?utm\_source=chatgpt.com \\ \hline
    26 & Little Man Computer (diverse Anbieter) & Stuart Madnick & Funktionsweise eines Rechners nach der Von-Neumann-Architektur & Offline, Web & Unabhängig & Diverse & didaktisch reduziert & Schule, Hochschule & Kostenlos & Nein & Rechnerarchitektur, von-Neumann-Architektur, & Nein & 30 - 60 Minuten & vorhanden & hoch & 1965 & 2025 & Nur ein Bsp.: https://peterhigginson.co.uk/lmc/ \\ \hline
    27 & Tinkercad Circuits & Autodesk & Ein webbasiertes Tool zur Gestaltung, Simulation und Programmierung von elektronischen Schaltungen & Web & Unabhängig & JavaScript/. TypeScript & didaktisch reduziert & Schule & Kostenlos (mit Anmeldung) & Nein & Elektronik, Schaltungstechnik & Nein & 30 - 60 Minuten & vorhanden & hoch & 2011 & 2025 & https://www.tinkercad.com/circuits \\ \hline
    28 & CircuitVerse & CircuitVerse (Github) & Simulator für digitale Logikschaltungen & Web & Unabhängig & JavaScript & didaktisch reduziert & Schule, Hochschule & Open Source & Nein & Digitale Logikschaltungen & Grundkenntnisse in digitaler Logik & 1-12 Stunden & vorhanden & hoch & Keine Information & 2025 & https://circuitverse.org/ \\ \hline
    29 & Visual6502 & Barry Silverman, Greg James, Ed Spittles & Ein Transistor-genauer, taktexakter Simulator des MOS 6502 Mikroprozessors & Web & Unabhängig & JavaScript & realitätsnah & Forschung & Kostenlos & Nein & Mikroprozessor & Grundkenntnisse & 30 - 60 Minuten & vorhanden & niedrig & 2009 & 2010 & http://www.visual6502.org/ \\ \hline
    30 & Sim8085 & Debjit Biswas (Github) & Webbasierter Simulator für den Mikroprozessor Intel 8085 & Web & Unabhängig & JavaScript & didaktisch reduziert & Hochschule & Open Source & Nein & Mikroprozessor & Grundkenntnisse & 30 - 60 Minuten & vorhanden & mittel & 2018 & 2025 & https://www.sim8085.com/ \\ \hline
    31 & DLX SImulator & Universität Salzburg & Ein Simulator für die Lehr-DLX Architektur & Web & Unabhängig & Keine Informationen & didaktisch reduziert & Hochschule & Kostenlos & Nein & RISC & Grundkenntnisse & 30 - 60 Minuten & Nein & niedrig & Keine Information & Keine Information & https://lv.cosy.sbg.ac.at/digitale/dlxwsim/ \\ \hline
    32 & EduMIPS64 & EduMIPS64 (Github) & Simulator für die MIPS-64 Instruktionssatzarchitektur & Offline, Web & Unabhängig & Java & didaktisch reduziert & Hochschule & Open Source & Nein & ISA MIPS-64 & Grundkenntnisse & 30 - 60 Minuten & vorhanden & mittel & 2006 & 2025 & https://edumips.org/ \\ \hline
    33 & Venus & Keyhan Vakil & RISC-V Simulator & Web & Unabhängig & JavaScript & didaktisch reduziert & Hochschule & Open Source & Nein & RISC-V & Grundkenntnisse & 30 - 60 Minuten & vorhanden & mittel & 2017 & 2017 & https://github.com/kvakil/venus \\ \hline
    34 & Verilator & Ventilator (Github) & Verilog/SystemVerilog-Simulator, der HDL-Code in C++ oder SystemC übersetzt & Offline & Windows, Linux & C++, SystemVerilog & realitätsnah & Forschung & Open Source & Nein & Schaltungstechnik & Verilog/ SystemVeriolog & 1-12 Stunden & vorhanden & hoch & 1998 & 2025 & https://www.veripool.org/verilator/ \\ \hline
    35 & RISC-V Web Simulator & William J. Song & RISC-V Simulator & Web & Unabhängig & JavaScript & didaktisch reduziert & Hochschule & Kostenlos & Nein & RISC-V & Grundkenntnisse & 30 - 60 Minuten & beschränkt & niedrig & Keine Information & Keine Information & https://riscv.vercel.app/ \\ \hline
    36 & Spike & Riscv-software-isa (Github) & RISC-V Simulator & Offline & Unabhängig & C & realitätsnah & Hochschule, Forschung & Open Source & Nein & RISC-V & Grundkenntnisse & 30 - 60 Minuten & beschränkt & hoch & 2019 & 2025 & https://github.com/riscv-software-src/riscv-isa-sim \\ \hline
    37 & BRISC-V & ASCSlab & RISC-V Simulator & Web & Unabhängig & JavaScript & didaktisch reduziert & Hochschule & Kostenlos & Nein & RISC-V & Grundkenntnisse & 30 - 60 Minuten & vorhanden & mittel & 2018 & 2021 & https://ascslab.org/research/briscv/simulator/simulator.html \\ \hline
    38 & QtRVSim & cvut (Github) & RISC-V Simulator & Offline, Web & Unabhängig & C++ & didaktisch reduziert & Hochschule & Open Source & Nein & RISC-V & Grundkenntnisse & 30 - 60 Minuten & vorhanden & mittel & 2022 & 2024 & https://github.com/cvut/qtrvsim \\ \hline
    39 & CISC-simulator & praveen1496 & CISC-Simulator & Oflline & Unabhängig & Java & didaktisch reduziert & Hochschule & Open Source & Nein & CISC & Grundkenntnisse & 30 - 60 Minuten & vorhanden & niedrig & 2018 & 2019 & https://github.com/praveen1496/CISC-simulator \\ \hline
    40 & CISC 6461 & Nicholas Capurso, Aaron Smith, Joel Klein, Alex Remily, Evanna Reynoso & CISC-Simulator & Oflline & Unabhängig & Java & didaktisch reduziert & Hochschule & Open Source & Nein & CISC & Grundkenntnisse & 30 - 60 Minuten & vorhanden & niedrig & 2014 & 2015 & https://github.com/nickcapurso/CISC-Simulator-Group-Project-CSCI-6461 \\ \hline
    41 & Pipeline Simulator & Lehrstuhl EECS 370 Universität Michigan & Pipeline Verarbeitung von Instruktionen & Web & Unabhängig & JavaScript & didaktisch reduziert & Hochschule & Kostenlos & Nein & Pipelining & Grundkenntnisse & 30 - 60 Minuten & Nein & niedrig & Keine Information & Keine Information & https://vhosts.eecs.umich.edu/370simulators/pipeline/simulator.html \\ \hline
    42 & simpipe & leondavi (Github) & Pipeline Verarbeitung von Instruktionen & Offline & Unabhängig & Python & didaktisch reduziert & Hochschule & Open Source & Nein & Pipelining & Grundkenntnisse & 30 - 60 Minuten & Nein & niedrig & 2020 & 2021 & https://github.com/leondavi/simpipe \\ \hline
    43 & Superscalar Processor Simulator & SSutherlandDeeBristol (Github) & Superskalarer Prozessor mit Out-of-Order Ausführung & Offline & Unabhängig & Java & didaktisch reduziert & Hochschule & Open Source & Nein & Superskalarität & Grundkenntnisse & 30 - 60 Minuten & Nein & niedrig & 2019 & 2019 & https://github.com/SSutherlandDeeBristol/superscalar-processor-simulator/commits/master/?after=d69ab924e35355ac3583e646674409e48ae6f9f0+34 \\ \hline
    44 & RISC-V Simulator & Jiří Jaroš, Michal Majer, Jakub Horký, Jan Vávra & RISC-V Simulator & Web & Unabhängig & Keine Informationen & realitätsnah & Hochschule, Forschung & Kostenlos & Nein & RISC-V & Grundkenntnisse & 30 - 60 Minuten & vorhanden & niedrig & 2024 & 2024 & https://sc-nas.fit.vutbr.cz:11443/ \\ \hline
    45 & Superscalar-CPU-Simulator & Charana123 (Github) & Superskalarer Prozessor mit Out-of-Order Ausführung & Offline & Unabhängig & Python & didaktisch reduziert & Hochschule & Open Source & Nein & Superskalarität & Grundkenntnisse & 30 - 60 Minuten & Nein & niedrig & 2018 & 2019 & https://github.com/Charana123/Superscalar-CPU-Simulator/commits/master/ \\ \hline
    46 & SESC & IACOMA-Gruppe & Superskalarer Prozessor & Offline & Linux & C++ & realitätsnah & Hochschule, Forschung & Kostenlos & Nein & Superskalarität & Grundkenntnisse & 30 - 60 Minuten & vorhanden & hoch & 2005 & 2005 & https://sesc.sourceforge.net/ \\ \hline
    47 & Logigator & Logigator GmbH & Tool zum Entwerfen, Simulieren und Verwalten digitaler Logikschaltungen & Web & Unabhängig & JavaScript & didaktisch reduziert & Schule & Lizenz & Nein & Digitale Logikschaltungen & Nein & 30 - 60 Minuten & vorhanden & hoch & Keine Information & Keine Information & https://logigator.com/de \\ \hline
    48 & Digital Logic Sim & Sebastian Lague & Tool zum Entwerfen, Simulieren und Verwalten digitaler Logikschaltungen & Offline & Windows, MacOS, Linux & C\# & didaktisch reduziert & Schule & Kostenlos & Nein & Digitale Logikschaltungen & Nein & 30 - 60 Minuten & vorhanden & mittel & 2020 & 2025 & https://digital-logic-sim.de.softonic.com/\#google\_vignette \\ \hline
    49 & Digital & Helmut Neemann & Tool zum Entwerfen, Simulieren und Verwalten digitaler Logikschaltungen & Offline & Windows, MacOS, Linux & Java & didaktisch reduziert & Hochschule & Open Source & Nein & Digitale Logikschaltungen & Grundkenntnisse & 30 - 60 Minuten & vorhanden & mittel & 2020 & 2025 & https://github.com/hneemann/Digital \\ \hline
    50 & EDA Playground & Duolos & HDL-Code-Simulator zu schreiben, zu simulieren und zu teilen & Web & Unabhängig & Keine Informationen & didaktisch reduziert & Hochschule & Kostenlos (mit Anmeldung) & Nein & Digitale Logikschaltungen & Grundkenntnisse & 30 - 60 Minuten & vorhanden & mittel & 2013 & 2025 & https://www.edaplayground.com/ \\ \hline
    51 & IEEE-754 Floating Point Converter & Tobias H. Schmidt & Eine Dezimalzahl in ihre IEEE-754 32-Bit Floating Point Darstellung umzuwandeln (und umgekehrt) & Web & Unabhängig & JavaScript & realitätsnah & Hochschule & Kostenlos & Nein & Floating Point & Nein & 30 - 60 Minuten & vorhanden & niedrig & Keine Information & Keine Information & https://www.h-schmidt.net/FloatConverter/IEEE754.html \\ \hline
    52 & BinaryConvert & Keine Informationen & Ein Online-Konverter für Zahlensysteme & Web & Unabhängig & JavaScript & realitätsnah & Hochschule & Kostenlos & Nein & Zahlensysteme & Nein & 30 - 60 Minuten & vorhanden & niedrig & Keine Information & Keine Information & https://www.binaryconvert.com/ \\ \hline
    53 & GPGPU-Sim & Tor Aamodt & Simulator für Grafikprozessoren & Offline & Linux & C++ & realitätsnah & Hochschule, Forschung & Open Source & Nein & GPU & Grundkenntnisse & 1-12 Wochen & vorhanden & hoch & 2007 & 2025 & https://github.com/gpgpu-sim/gpgpu-sim\_distribution \\ \hline
    54 & ESESC & K. Ardestani, R. Renau & Multicore Simulator & Offline & Linux & C++ & realitätsnah & Hochschule, Forschung & Open Source & Nein & Multicore Prozessorarchitektur & Fortgeschritten & 1-12 Wochen & vorhanden & hoch & 2013 & 2021 & http://masc.soe.ucsc.edu/esesc/ \\ \hline
    55 & CACTI & The Cacti Group & Monitoring- und Graphing-Werkzeug & Offline & Windows, MacOS, Linux & PHP, C, … & realitätsnah & Hochschule, Forschung & Open Source & Nein & Netzwerk- und System-Monitoring & Grundkenntnisse & 1-12 Stunden & vorhanden & hoch & 2017 & 2025 & https://www.cacti.net \\ \hline
    56 & HotSpot & University of Virginia & Es simuliert Temperaturverhalten von integrierten Schaltungen (2D \& 3D ICs), unterstützt Microfluidic Cooling, Floorplans, transienter und stationärer Temperaturfluss; & Oflline & Unabhängig & C & realitätsnah & Hochschule, Forschung & Open Source & Nein & Thermische Modellierung & Grundkenntnisse & 1-12 Stunden & vorhanden & hoch & 2021 & 2022 & https://github.com/uvahotspot/HotSpot \\ \hline
    57 & SimScale & SimScale GmH & Cloud-basierte CAE-(Computer Aided Engineering) Plattform & Web & Unabhängig & Keine Informationen & realitätsnah & Hochschule, Forschung & Lizenz & Nein & CFD (Fluid Dynamics), FEA (Strukturmechanik), Wärmeübertragung (Thermodynamics) & Grundkenntnisse & 1-12 Stunden & vorhanden & hoch & 2013 & 2025 & https://www.simscale.com/ \\ \hline
\end{longtable}
\end{landscape}

\printbibliography
\MakeBibliography[nosplit]

\end{document}
