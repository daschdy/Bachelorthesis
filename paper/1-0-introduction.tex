\chapter{Einleitung}

\section{Motivation}

Die nachfolgende Bachelorarbeit behandelt das Thema "Best Practices und Trends bei didaktischen Simulatoren für Rechnerarchitektur: Eine systematische Literaturrecherche". Sie wurde am Lehrstuhl für Rechnerarchitektur der Universität Erlangen-Nürnberg im Sommersemester 2025 geschrieben und bildet die abschließende Prüfungsleistung des Bachelorstudiums Informatik.

\TODO{Noch weiter ausarbeiten?}

Studien zeigen, dass die Erfolgsquote von Studienanfänger:innen in naturwissenschaftlichen Studiengängen deutlich geringer ist als in anderen Fachrichtungen \parencite[S.~370]{burdinski_lehrvideos_2024}. Laut dem aktuellen MINT-Nachwuchsbarometer 2024 lag die Abbruchquote im Jahr 2022 bei 50,5\,\%, Tendenz steigend. Die Hintergründe für diese Studienabbrüche betitelt das MINT-Nachwuchsbarometer als vielfältig, die wesentlichen Ursachen liegen jedoch in einer Diskrepanz zwischen den Erwartungen und Interessen der Studierenden und den tatsächlichen Studieninhalten sowie in unzureichenden fachlichen Voraussetzungen \parencite[S.~21]{joachim_herz_stiftung_mint_2024}.

Gerade im Bereich der Rechnerarchitektur stellen komplexe und abstrakte Inhalte eine besondere Hürde dar \parencite[S.~1]{grober_championship_2022}. Didaktische Simulatoren können hier ansetzen, indem sie schwer fassbare Konzepte anschaulich machen, den Lernenden unmittelbares Feedback ermöglichen und so den Zugang zu den Inhalten erleichtern \parencite[S.~11]{zeichner_using_2020}. Empirische Studien zeigen, dass Simulatoren das Verständnis der internen Funktionsweise von Rechnern verbessern \parencite[S.~215]{prasad_using_2015}, von Studierenden überwiegend positiv bewertet werden \parencite[S.~8]{besim_understanding_2012} und durch spielerische Ansätze zusätzliche Motivation erzeugen können \parencite[S.~453]{schlag_gamifizierung_2021}. Auch aktuelle Evaluierungen visueller CPU-Simulatoren belegen den didaktischen Nutzen solcher Werkzeuge \parencites[S.~11]{maxnuck_soares_use_2016}[S.~75]{cortinovis_further_2024}. Dadurch tragen didaktische Simulatoren potenziell dazu bei, die Motivation und den Lernerfolg zu steigern und langfristig Studienabbrüche zu reduzieren \parencite[S.~272]{kornelsen_expedition_2005}.

\section{Zielsetzung}

Ziel dieser Arbeit ist es, didaktische Simulatoren mit Relevanz für die Lehre in der Rechnerarchitektur systematisch zu erfassen und vergleichbar zu machen. Hierzu werden sowohl verfügbare als auch wissenschaftlich erläuterte Simulatoren anhand definierter Kriterien untersucht und kategorisiert. Auf dieser Grundlage entsteht eine strukturierte Übersicht, die sowohl die Vielfalt der Werkzeuge als auch ihre spezifischen Einsatzgebiete sichtbar macht.

Darüber hinaus werden aktuelle Trends und Entwicklungen identifiziert, um einen zeitgemäßen Überblick über den Stand der Forschung und Lehre in diesem Bereich zu geben. Dies umfasst die Analyse der historischen Entwicklung didaktischer Simulatoren sowie eine Clusterung nach Simulatortypen. Aus diesen Analysen werden schließlich zentrale Entwicklungen und Best Practices abgeleitet.

\section{Aufbau der Arbeit}

\TODO{Auf das Kapitel 3 genauer eingehen, sodass man versteht, wieso die Kapitel entsprechend gewählt wurden}

Die Arbeit gliedert sich in sechs Kapitel, die inhaltlich aufeinander aufbauen. Nach der Einleitung, in der Motivation und Zielsetzung dargelegt werden, folgt im zweiten Kapitel die Methodik. Dort wird beschrieben, wie relevante Simulatoren identifiziert wurden, welche Kriterien zur Analyse herangezogen werden und in welchem Umfang die Literaturrecherche abgegrenzt ist. Im dritten Kapitel werden die theoretischen Grundlagen gelegt: Es werden die Relevanz und Herausforderungen didaktischer Simulatoren in der Lehre diskutiert, zentrale Konzepte wie etwa Gamification vorgestellt und verschiedene Definitionen sowie Typen von Simulatoren erläutert.

Das vierte Kapitel widmet sich den Ergebnissen der Literaturrecherche. Hier werden die historische Entwicklung didaktischer Simulatoren nachgezeichnet und die erfassten Werkzeuge sowohl thematisch nach Inhalten der Rechnerarchitektur als auch nach Simulatortypen kategorisiert. Darauf aufbauend erfolgt in Kapitel fünf eine Diskussion der Ergebnisse. Neben einer Einordnung aktueller Trends und Herausforderungen werden hier auch Best Practices sowie Empfehlungen für zukünftige Entwicklungen abgeleitet. Den Abschluss bildet Kapitel sechs, das die wichtigsten Erkenntnisse zusammenfasst, die zentrale Zielsetzung der Arbeit reflektiert und einen Ausblick auf weiterführende Forschungsansätze gibt.
