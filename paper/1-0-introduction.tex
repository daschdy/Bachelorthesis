\chapter{Einleitung}

\section{Motivation}

Die nachfolgende Bachelorarbeit behandelt das Thema "Best Practices und Trends bei didaktischen Simulatoren für Rechnerarchitektur: Eine systematische Literaturrecherche". Sie wurde am Lehrstuhl für Rechnerarchitektur der Universität Erlangen-Nürnberg im Sommersemester 2025 geschrieben und bildet die abschließende Prüfungsleistung des Bachelorstudiums Informatik.

Studien zeigen, dass die Erfolgsquote von Studienanfänger:innen in naturwissenschaftlichen Studiengängen deutlich geringer ist als in anderen Fachrichtungen.\parencite[S. 370]{burdinski_lehrvideos_2024} Laut dem aktuellen MINT-Nachwuchsbarometer 2024 lag die Abbruchquote im Jahr 2022 bei 50,5\,\%, Tendenz steigend. Die Hintergründe für Studienabbrüche sind vielfältig, die wesentlichen Ursachen liegen jedoch in einer Diskrepanz zwischen den Erwartungen und Interessen der Studierenden und den tatsächlichen Studieninhalten sowie in unzureichenden fachlichen Voraussetzungen.\parencite[S. 21]{joachim_herz_stiftung_mint_2024}

Gerade im Bereich der Rechnerarchitektur stellen komplexe und abstrakte Inhalte eine besondere Hürde dar. Didaktische Simulatoren können hier ansetzen, indem sie schwer fassbare Konzepte anschaulich machen, den Lernenden unmittelbares Feedback ermöglichen und so den Zugang zu den Inhalten erleichtern. Empirische Studien zeigen, dass Simulatoren das Verständnis der internen Funktionsweise von Rechnern verbessern,\parencite[S. 215]{prasad_using_2015} von Studierenden überwiegend positiv bewertet werden,\parencite[S. 8]{besim_understanding_2012} und durch spielerische Ansätze zusätzliche Motivation erzeugen können.\TODO{daschmann_improving_2016} Auch aktuelle Evaluierungen visueller CPU-Simulatoren belegen den didaktischen Nutzen solcher Werkzeuge.\TODO{cpvsim_2024} Dadurch tragen didaktische Simulatoren potenziell dazu bei, die Motivation und den Lernerfolg zu steigern und langfristig Studienabbrüche zu reduzieren.

\section{Zielsetzung}

Innerhalb dieser Arbeit werden Simulatoren vorgestellt und untersucht, die Relevanz in den Themengebieten der Rechnerarchitektur haben und für didaktische Zwecke genutzt werden.

\section{Aufbau der Arbeit}
