\chapter{Methodik}

\section{Recherche und Auswahl der Simulatoren}

Das Ziel der Recherche bestand darin, aktuelle Trends und Best Practices didaktischer Simulatoren in der Rechnerarchitektur zu identifizieren. Hierfür wurde ein systematisches Vorgehen gewählt, um sowohl wissenschaftliche Publikationen als auch einschlägige Fachliteratur und konkrete Werkzeuge zu erfassen.

Die Suche nach relevanter Literatur erfolgte in den technisch ausgerichteten Datenbanken \textit{IEEE Xplore} und \textit{ACM Digital Library}, ergänzt durch die Universitätskataloge der \textit{Friedrich-Alexander-Universität Erlangen-Nürnberg} sowie der \textit{Otto-Friedrich-Universität Bamberg}. Zusätzlich wurde \textit{Google Scholar} herangezogen, um die Suche durch weitere Quellen wie technische Dokumentationen, Konferenzbeiträge oder Abschlussarbeiten zu erweitern. Diese sogenannte „graue Literatur“, die nicht durch klassische Verlage veröffentlicht und in der Regel nicht peer-reviewed ist, wurde insbesondere zur Identifikation aktueller Trends berücksichtigt. Eingeschlossen wurden nur Quellen, die frei zugänglich oder im Rahmen des studentischen Abonnements verfügbar waren.

Als Suchstrings wurden Kombinationen aus den Begriffen \textit{"computer architecture"}, \textit{"education"}, \textit{"simulator"}, \textit{"teaching"}, \textit{"Rechnerarchitektur"} und \textit{"Lehre"} verwendet.

Die Suchergebnisse wurden nach Relevanz sowie nach der didaktischen Ausrichtung der beschriebenen Simulatoren gefiltert. Um die Vielzahl potenzieller Simulatoren einzugrenzen, wurden folgende Kriterien definiert:

\begin{itemize}
  \item \textit{Didaktische Ausrichtung}: Der Simulator sollte primär für Lehr- und Lernzwecke in Hochschulen oder anderen Bildungseinrichtungen entwickelt oder eingesetzt worden sein. Zusätzlich wurden ausgewählte Simulatoren mit Relevanz für die Forschung berücksichtigt.
  \item \textit{Abdeckung der Lerninhalte}: Es wurde geprüft, welche Teilgebiete der Rechnerarchitektur (z. B. Pipeline, Cache, Mikroarchitektur) abgedeckt werden.
\end{itemize}

Insgesamt wurden im Rahmen der Recherche rund 150 Simulatoren identifiziert, die thematisch der Rechnerarchitektur zugeordnet werden können. Diese Grundgesamtheit umfasst sowohl didaktisch orientierte als auch forschungsnahe Simulatoren, die in der Literatur, in Lehrmaterialien oder als Open-Source-Werkzeuge dokumentiert sind.

Neben der systematischen Literaturrecherche wurde zusätzlich eine Aufstellung bereits veröffentlichter und frei verfügbarer Simulatoren erstellt. Hierbei wurden insbesondere Simulatoren berücksichtigt, die im Rahmen von Lehrveranstaltungen an Hochschulen oder als Open-Source-Projekte verbreitet sind, jedoch nicht zwingend in wissenschaftlichen Publikationen beschrieben wurden.

Diese parallele Erfassung diente dazu, die Grundgesamtheit zu erweitern und auch praxisrelevante Werkzeuge in die Analyse einzubeziehen, die in der wissenschaftlichen Literatur unter Umständen keine oder nur geringe Beachtung gefunden haben.

Im Kapitel 4~\ref{results} werden die Ergebnisse der Recherche untersucht und entsprechend gefiltert.

\section{Kriterienkatalog}

\TODO{Beschreiben, wie die Clustereinteilung vergeben wurde. Bspw. FPGA = 1}

\TODO{Bei mehreren Simulatoren den Querschnitt}

\TODO{Unterscheidung zwischen Simulatoren und Literaturrecherche}

Tabelle~\ref{tab:kriterien} fasst die für die Analyse herangezogenen Kriterien zusammen.

Ein zentrales Kriterium ist der \textit{Zugriff}. Je nach organisatorischem Kontext ist entscheidend, ob ein Simulator online oder offline verfügbar ist. Während eine Online-Lösung eine Internetverbindung erfordert und dadurch eingeschränkt nutzbar sein kann, setzt ein Offline-Simulator häufig eine Installation voraus, die wiederum von Betriebssystem oder Hardwareanforderungen abhängt.

Die \textit{Programmiersprache} ist insbesondere für Studierende der Informatik von Relevanz. Sie kann die Möglichkeit eröffnen, Erweiterungen oder Plugins zu entwickeln und den Simulator an spezifische Bedürfnisse anzupassen.

Unter dem Kriterium \textit{Simulatorart} wird der Abstraktionsgrad eingeordnet, also ob der Simulator didaktisch reduziert oder realitätsnah ist. Ergänzend werden auch die Visualisierung und der Grad der Interaktivität berücksichtigt.

Das Kriterium \textit{Zielgruppe} erlaubt eine Zuordnung zu unterschiedlichen Bildungskontexten (z. B. Schule, Hochschule) und gibt Auskunft darüber, ob und in welchem Umfang Vorkenntnisse erforderlich sind.

Für Lehrende spielt zudem der \textit{Preis} eine zentrale Rolle. Es wird unterschieden, ob es sich um freie oder lizenzpflichtige Software handelt, da der Kostenfaktor maßgeblich über den praktischen Einsatz entscheidet.

Ein weiterer Aspekt ist die \textit{Gamification}. Studien zeigen, dass spielerische Elemente die Lernmotivation steigern können. Unterschieden werden hier insbesondere levelbasierte und storytelling-orientierte Ansätze.\parencite[S. 106f]{sailer_gamification_2020}\parencite[S. 13]{baah_enhancing_2024}

Da die Rechnerarchitektur eine Vielzahl von \textit{Themenbereichen} abdeckt, wird erfasst, auf welchen inhaltlichen Schwerpunkt sich ein Simulator bezieht (z. B. Pipelining oder Caching).

Unter \textit{Beschäftigungsdauer} wird die zu erwartende Nutzungszeit betrachtet. Dabei ist relevant, ob die Dauer der Auseinandersetzung mit dem Simulator die notwendige Einführungs- und Erklärungszeit übersteigt.
\TODO{Bekanntheitsgrad = Menge der Literatur?}

Die Qualität der \textit{Dokumentation} ist ein weiteres zentrales Kriterium. Eine klare und umfassende Dokumentation erleichtert Installation, Anwendung und Problemlösung und wirkt sich direkt auf die Nutzbarkeit des Simulators aus.

Da die Arbeit auch Trends und Entwicklungen berücksichtigt, werden zudem der \textit{Bekanntheitsgrad}, der \textit{Wartungsstand} (letztes Update) sowie das Jahr der \textit{Veröffentlichung} aufgenommen, um den Stellenwert eines Simulators einzuordnen und veraltete Software zu identifizieren.

\begin{table}

\centering
\caption{Kriterienkatalog}
\label{tab:kriterien}
\small

\begin{tabular}{|l|l|l|}
\hline

\textbf{Nr.} & \textbf{Kriterien} & \textbf{Kategorie} \\
\hline

\multirow{4}{*}{1} & \multirow{4}{*}{Zugriff} & Offline \\
                   &                          & Betriebssystem \\
                   &                          & Online \\
                   &                          & Hardwareanforderungen \\
\hline

\multirow{3}{*}{2} & \multirow{3}{*}{Programmiersprache} & Entwicklungssprache \\
                   &                                     & Embedding von Programmiersprachen \\
                   &                                     & Integration in Lernplattformen \\
\hline

\multirow{3}{*}{3} & \multirow{3}{*}{Simulatorart} & Abstraktionsgrad \\
                   &                               & Visualisierung \\
                   &                               & Interaktivität \\
\hline

\multirow{2}{*}{4} & \multirow{2}{*}{Zielgruppe} & Institution \\
                   &                             & Vorkenntnisse \\
\hline

\multirow{2}{*}{5} & \multirow{2}{*}{Preis} & Freeware \\
                   &                        & Open Source \\
\hline

\multirow{2}{*}{6} & \multirow{2}{*}{Gamification} & Level \\
                   &                               & Storytelling \\
\hline

\multirow{2}{*}{7} & \multirow{2}{*}{Themenbereich} & Pipelining \\ 
                   &                                & Caching \\
\hline

\multirow{3}{*}{8} & \multirow{3}{*}{Beschäftigungsdauer} & kurzfrisitg \\
                   &                                      & mittelfristig \\
                   &                                      & langfrisitig \\ 
\hline

\multirow{2}{*}{9} & \multirow{2}{*}{Dokumentation} & Verfügbarkeit \\
                   &                                & Zugriff \\
\hline

10 & Bekanntheitsgrad & Skala von 0 - 10 \\
\hline

11 & Wartungsstand & letztes Update \\
\hline

12 & Veröffentlichung & Aktualität \\
\hline

\end{tabular}
\end{table}

\section{Limitierungen}
