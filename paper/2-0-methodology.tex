\chapter{Methodik}

\section{Literaturauswahl und -recherche}

Das Ziel der Recherche besteht darin, aktuelle Trends und Best Practices didaktischer Simulatoren in der Rechnerarchitektur zu identifizieren. Hierfür wird ein systematisches Vorgehen gewählt, um sowohl wissenschaftliche Publikationen als auch einschlägige Fachliteratur und konkrete Werkzeuge zu erfassen.

Die Suche nach relevanter Literatur erfolgt in den technisch ausgerichteten Datenbanken \textit{IEEE Xplore} und \textit{ACM Digital Library}, ergänzt durch die Universitätskataloge der \textit{Friedrich-Alexander-Universität Erlangen-Nürnberg} sowie der \textit{Otto-Friedrich-Universität Bamberg}. Zusätzlich wird \textit{Google Scholar} herangezogen, um die Suche durch weitere Quellen wie technische Dokumentationen, Konferenzbeiträge oder Abschlussarbeiten zu erweitern. Diese sogenannte \enquote{graue Literatur}, die nicht durch klassische Verlage veröffentlicht und in der Regel nicht peer-reviewed ist, spiegeln die aktuellen Trends wieder. Eingeschlossen werden nur Quellen, die frei zugänglich oder im Rahmen des studentischen Abonnements verfügbar sind.

Als Suchstrings werden Kombinationen aus den Begriffen \textit{\enquote{computer architecture}}, \textit{\enquote{education}}, \textit{\enquote{simulator}}, \textit{\enquote{teaching}}, \textit{\enquote{Rechnerarchitektur}} und \textit{\enquote{Lehre}} verwendet. Diese Suchergebnisse werden nach Relevanz sowie nach der didaktischen Ausrichtung der beschriebenen Simulatoren gefiltert. 

Um die Vielzahl potenzieller Simulatoren einzugrenzen, werden folgende Kriterien definiert:

\begin{itemize}
  \item \textit{Didaktische Ausrichtung}: Der Simulator soll primär für Lehr- und Lernzwecke in Hochschulen oder anderen Bildungseinrichtungen entwickelt oder eingesetzt werden. Zusätzlich werden ausgewählte Simulatoren mit Relevanz für die Forschung berücksichtigt.
  \item \textit{Abdeckung der Lerninhalte}: Es wird geprüft, welche Teilgebiete der Rechnerarchitektur (z.B. Pipeline, Cache, Mikroarchitektur) abgedeckt werden.
\end{itemize}

Insgesamt werden im Rahmen der Recherche rund 200 Simulatoren identifiziert, die thematisch der Rechnerarchitektur zugeordnet werden können. Diese Grundgesamtheit umfasst sowohl didaktisch orientierte als auch forschungsnahe Simulatoren, die in der Literatur, in Lehrmaterialien oder als Open-Source-Werkzeuge dokumentiert sind.

Neben der systematischen Literaturrecherche (siehe dazu Tabelle~\ref{tab:literatur}) wird zusätzlich eine Aufstellung bereits veröffentlichter und frei verfügbarer Simulatoren (siehe dazu Tabelle~\ref{tab:simulatoren}) erstellt. Im weiteren Verlauf der Arbeit werden die beiden Tabellen durch Literatur- und Simulatorrecherche unterschieden. Hierbei werden insbesondere Simulatoren berücksichtigt, die im Rahmen von Lehrveranstaltungen an Hochschulen oder als Open-Source-Projekte verbreitet sind, jedoch nicht zwingend in wissenschaftlichen Publikationen beschrieben werden.

Diese parallele Erfassung dient dazu, die Grundgesamtheit zu erweitern und auch praxisrelevante Werkzeuge in die Analyse einzubeziehen, die in der wissenschaftlichen Literatur unter Umständen keine oder nur geringe Beachtung finden.

\section{Kriterienkatalog}\label{chap:kriterienkatalog}

Um die vorhandene Forschung sowie veröffentlichte Simulatoren systematisch zu analysieren, werden neben den bibliographischen Angaben (z.B. \textit{Autor}, \textit{Titel}, \textit{Jahr}) zusätzliche Kriterien erfasst, die im Folgenden vorgestellt werden.

\sh{Kriterien der Literaturrecherche (Tabelle~\ref{tab:literatur})}
Die Kriterien \textit{Typ} und \textit{Zeitschrift} sind gemeinsam zu betrachten, da sie eine genauere Einordnung der Veröffentlichungen ermöglichen. Anhand dieser Angaben lässt sich beispielsweise erkennen, ob die Grundgesamtheit der erfassten Arbeiten überwiegend aus Büchern, Journalartikeln oder Konferenzbeiträgen besteht und in welchen Fachzeitschriften didaktische Simulatoren häufiger diskutiert werden. Somit tragen diese Kriterien zu einer besseren Kontextualisierung der recherchierten Literatur bei.

Anschließend wird das \textit{Thema} der jeweiligen Veröffentlichung dargestellt. Da einzelne Simulatoren mehrere Aspekte abdecken können, wurde in solchen Fällen auch die Kategorie \enquote{Grundkenntnisse} eingeführt, unter der grundlegende Konzepte der Rechnerarchitektur zusammengefasst sind.

Um die Analyse zu vereinheitlichen und die Vergleichbarkeit zu erhöhen, wurden die vielfältigen Themenbereiche der Rechnerarchitektur in inhaltliche Cluster (siehe Tabelle~\ref{tab:cluster}) überführt. Diese thematische Gruppierung erleichtert es, Schwerpunkte und Trends in den Publikationen zu identifizieren.

\begin{table}[h]
  \centering
  \caption{Themenbereiche und zugehörige Cluster}
  \label{tab:cluster}
  \small
  \begin{tabularx}{\textwidth}{lX}
    \toprule
    \textbf{Cluster} & \textbf{Themen} \\
    \midrule
    Grundlagen und Theorien & Allgemein, Grundkenntnisse, Von-Neumann-Architektur, Rechnerarchitektur, AIMC, ADL, Floating Point, Zahlensysteme, Multicore Prozessorarchitektur \\
    \addlinespace
    Prozessoren und Architekturen & Prozessor, CPU, Mikroprozessor, Mikroprogrammierung, MIPS, RISC, CISC, Superskalarität, Pipelining, NoC \\
    \addlinespace
    Speicher und Performance & Speicherhierarchie, Cache, TLB, Temperatur \\
    \addlinespace
    Hardware und Logik & Digitale Logik, HDL, VHDL, Verilog, FPGA, Einsatz echter Hardware \\
    \addlinespace
    Programmierung & Assembler, Compiler \\
    \addlinespace
    Systeme und Anwendungen & Cloud Computing, IoT, VR, Betriebssysteme \\
    \addlinespace
    Monitoring & Monitoring \\
    \addlinespace
    GPU & GPU \\
    \addlinespace
    AI & AI \\
    \bottomrule
  \end{tabularx}
\end{table}

Ein weiterer untersuchter Aspekt ist die \textit{Gamification}. Studien zeigen, dass spielerische Elemente die Lernmotivation signifikant steigern können. Besonders verbreitet sind dabei levelbasierte sowie storytelling-orientierte Ansätze \parencites[S.~106f]{sailer_gamification_2020}[S.~13]{baah_enhancing_2024}.

Zur Auswertung wird das Vorhandensein von Gamification-Elementen binär codiert: Ein Wert von \enquote{0} kennzeichnet Simulatoren ohne spielerische Komponenten, während \enquote{1} angibt, dass entsprechende Elemente integriert sind.

Unter dem Kriterium \textit{Abstraktion} wird der Grad der didaktischen Vereinfachung verstanden. Simulatoren mit einer stark reduzierten Darstellung werden mit \enquote{1} codiert, realitätsnahe Abbildungen hingegen mit \enquote{0}. 

Eng damit verknüpft ist das Kriterium \textit{Institution}, das den Kontext erfasst, in dem ein Simulator entwickelt oder eingesetzt wird. Auf diese Weise lässt sich unterscheiden, ob die Lerninhalte und der Anforderungsgrad primär auf den schulischen Bereich, die hochschulische Lehre oder andere Bildungskontexte ausgerichtet sind.

Eine Übersicht der hierfür verwendeten Codierungen ist in Tabelle~\ref{tab:legende_codierung} dargestellt.

\begin{table}[h]
  \centering
  \caption{Codierungen der Kriterien}
  \label{tab:legende_codierung}
  \tiny
  \begin{tabularx}{\textwidth}{p{2cm}p{1.5cm}p{1.7cm}p{1.7cm}p{1.5cm}p{1.5cm}}
    \toprule
    \textbf{Kriterium} & \textbf{0} & \textbf{1} & \textbf{2} & \textbf{3} & \textbf{4} \\
    \midrule
    Abstraktion       & realitätsnah & reduziert & -- & -- & -- \\
    Institution       & Schule & Hochschule & Forschung, Beruf & Weiterbildung & -- \\
    Gamification      & keine Elemente & Elemente vorhanden & -- & -- & -- \\
    Zugriff           & online & offline & k.A. & on- und offline & -- \\
    Preis             & kostenlos & kostenpflichtig & k.A. & -- & -- \\
    Dokumentation     & vorhanden & nicht dokumentiert & k.A. & -- & -- \\
    OS*               & Linux & Windows & macOS & VM & unabhängig \\
    Vorwissen*        & ohne & Grundlagen & fortgeschritten & -- & -- \\
    Zeit*             & kurz & mittel & lang & -- & -- \\
    Bekanntheitsgrad* & niedrig & mittel & hoch & -- & -- \\
    \bottomrule
  \end{tabularx}
  \vspace{0.5em}
  {\tiny Anmerkung: k.A. = keine Angabe; VM = Virtual Machine; * relevant für veröffentliche Simulatoren}
\end{table}

Ein zentrales Kriterium ist der \textit{Zugriff}. Je nach organisatorischem Kontext spielt es eine entscheidende Rolle, ob ein Simulator online oder offline verfügbar ist. Online-Lösungen erfordern eine dauerhafte Internetverbindung und können dadurch in ihrer Nutzbarkeit eingeschränkt sein, während Offline-Simulatoren zwar unabhängig vom Netz funktionieren, jedoch meist eine Installation sowie bestimmte Hard- oder Betriebssystemvoraussetzungen benötigen \parencites[S.~11f]{ferguson_importance_2024}[S.~1380]{sharma_study_2022}.

Insbesondere im Rahmen der Analyse der veröffentlichten Simulatoren (vgl. Tabelle~\ref{tab:simulatoren}) spielen die Hardwareanforderungen eine zentrale Rolle.

Für Lehrende stellt der \textit{Preis} ein zentrales Kriterium dar. Zur Codierung wird unterschieden, ob es sich um frei verfügbare Software (\enquote{0}) oder um kostenpflichtige Software (\enquote{1}) handelt. Der Kostenfaktor ist dabei ein maßgebliches Kriterium für den praktischen Einsatz in der Lehre.

Die Qualität der \textit{Dokumentation} ist ein weiteres Kriterium. Eine klare und umfassende Dokumentation erleichtert Installation, Anwendung und Problemlösung und beeinflusst damit unmittelbar die Nutzbarkeit eines Simulators. Zur Codierung wird festgelegt: \enquote{0} kennzeichnet das Vorhandensein einer Dokumentation, \enquote{1} das Fehlen entsprechender Unterlagen. Fehlende Angaben werden mit \enquote{2} vermerkt.

Die Anzahl der \textit{Zitationen} dient als Indikator für die Bekanntheit und Rezeption einer Veröffentlichung. Die Werte werden für jede Publikation über \href{https://scholar.google.de/}{Google Scholar} erhoben. Als Beispiel dient Abbildung~\ref{fig:anzahl_zitationen}. Hier wird die Anzahl der Zitationen (ID~\#6 der Tabelle~\ref{tab:literatur}) dargestellt.

\begin{figure}[htbp]
    \centering
    \includegraphics[width=0.90\textwidth]{img/Anzahl_Zitationen.png}
    \caption{Erhebung \enquote{Anzahl Zitationen}}
    \label{fig:anzahl_zitationen}
\end{figure}

\sh{Kriterien der Simulatorrecherche (Tabelle~\ref{tab:simulatoren})}
Die zuvor beschriebenen Kriterien lassen sich auch auf die Tabelle der veröffentlichten Simulatoren anwenden. Für diese Analyse werden jedoch zusätzliche Merkmale definiert, die im Folgenden erläutert werden.

Ein erster Aspekt ist das \textit{Vorwissen}. Es berücksichtigt den Kenntnisstand der Lernenden und zeigt auf, wie viel fachliches Wissen notwendig ist, um den Simulator sinnvoll einsetzen zu können. Zur Codierung wird unterschieden zwischen \enquote{0 = kein Vorwissen}, \enquote{1 = Grundkenntnisse} und \enquote{2 = fortgeschrittenes Wissen}.

Darüber hinaus spielt die \textit{Bekanntheit} eines Simulators eine Rolle. Sie dient als Indikator für Verbreitung und Rezeption und wird gestuft codiert: \enquote{0 = niedrige}, \enquote{1 = mittlere}, \enquote{2 = hohe Bekanntheit}. Grundlage dieser Einstufung sind Faktoren wie Erwähnungen in der Literatur, Zitationshäufigkeit oder die Sichtbarkeit in der Lehrpraxis.

Ein weiteres Kriterium ist die \textit{Beschäftigungsdauer} bzw. \textit{Zeit}, die angibt, über welchen Zeitraum ein Simulator typischerweise genutzt wird und wie lange es dauert, den Lernenden das Wissen zu vermitteln. Dabei wird unterschieden zwischen \enquote{0 = kurzfristige}, \enquote{1 = mittelfristige} und \enquote{2 = langfristige} Nutzung. So kann ermittelt werden, ob ein Simulator vor allem für einzelne Übungen, für ein Semester oder für längerfristige Lehrszenarien konzipiert ist.

Besondere Bedeutung kommt schließlich den Kriterien \textit{Veröffentlichung} und \textit{Wartungsstand} zu, die eng miteinander verknüpft sind. Während die \textit{Veröffentlichung} das Jahr der Erstpublikation eines Simulators angibt und damit eine zeitliche Einordnung in den Kontext der didaktischen und technischen Entwicklung ermöglicht, verweist der \textit{Wartungsstand} auf das Jahr der letzten Aktualisierung. In Kombination erlauben beide Kriterien die Einschätzung, ob ein Simulator kontinuierlich gepflegt wird oder ob er seit längerer Zeit nicht mehr aktualisiert und somit nur eingeschränkt einsetzbar ist.

\section{Limitierungen}

Trotz des systematischen Vorgehens weist die vorliegende Arbeit einige Limitierungen auf, die bei der Interpretation der Ergebnisse berücksichtigt werden sollten.

Zunächst ist die \textit{Literaturauswahl} zu nennen. Relevante Arbeiten könnten aufgrund eingeschränkter Datenbanken oder fehlender Zugangsrechte unberücksichtigt geblieben sein. Zusätzlich werden nur deutsch- und englischsprachige Veröffentlichungen untersucht. Damit besteht die Möglichkeit, dass das tatsächliche Spektrum an Veröffentlichungen nicht vollständig abgebildet wird.

Darüber hinaus betreffen Einschränkungen die \textit{Bewertungskriterien}. Die verwendeten Codierungen stellen eine notwendige Reduktion komplexer Eigenschaften dar. Zudem beruhen einige Einschätzungen, beispielsweise zum Bekanntheitsgrad einzelner Simulatoren, auf subjektiver Bewertung.

Ein weiterer Punkt betrifft die \textit{zeitliche Dynamik}. Die Analyse stellt eine Momentaufnahme dar, da sich Wartungsstände stetig verändern können. Somit besteht die Möglichkeit, dass die hier dargestellten Ergebnisse im Zeitverlauf an Aktualität verlieren.

Schließlich ist die \textit{Generalisierbarkeit} der Ergebnisse eingeschränkt. Die Arbeit konzentriert sich ausschließlich auf didaktische Simulatoren in der Rechnerarchitektur. Eine Übertragung der Ergebnisse auf andere Fachbereiche oder didaktische Werkzeuge ist daher nur bedingt möglich.
