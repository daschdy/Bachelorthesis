\chapter{Fazit}

Ziel dieser Arbeit war es, den Stand didaktischer Simulatoren für die Lehre der Rechnerarchitektur systematisch zu erfassen, vergleichbar zu machen und daraus Best Practices sowie Entwicklungsperspektiven abzuleiten. Durch die Analyse von 151 Publikationen und 57 veröffentlichten Simulatoren konnten zentrale thematische und didaktische Muster identifiziert werden.

Die Ergebnisse verdeutlichen klare Schwerpunkte: Prozessoren und Architekturen, insbesondere \acs{RISC}, bilden nach wie vor den Kernbereich der didaktischen Simulatoren. Zugleich gewinnen GPU-, KI-bezogene und immersive Ansätze zunehmend an Bedeutung und spiegeln die aktuellen technologischen Entwicklungen der Rechnerarchitektur wider. Damit lassen sich sowohl eine starke Kontinuität klassischer Inhalte als auch erste Verschiebungen hin zu neuen Themenfeldern beobachten.

Didaktisch wirksam sind vor allem reduzierte Darstellungen im Sinne der \acl{CTML}, da sie komplexe Sachverhalte auf wesentliche Kernelemente verdichten. Gamification wird bislang nur vereinzelt integriert, obwohl die Literatur deutliche Potenziale zur Steigerung der Lernmotivation belegt. Häufig genannte Erfolgsfaktoren sind darüber hinaus eine ortsunabhängige Nutzung (online oder hybrid), die Kostenfreiheit, plattformübergreifende Verfügbarkeit sowie eine umfassende und verlässliche Dokumentation. Zusammengenommen führen diese Befunde zu Empfehlungen für die zukünftige Gestaltung, Bereitstellung und didaktische Nutzung von Simulatoren.

Aus der Diskussion ergeben sich zwei zentrale Schlussfolgerungen: Erstens sollten Simulatoren gezielt für kleine, klar strukturierte Lerneinheiten konzipiert werden, die den Lernprozess schrittweise begleiten und bei Bedarf durch spielerische Elemente ergänzt werden können. Zweitens besteht eine deutliche Forschungslücke in Bezug auf belastbare empirische Studien, die den Einsatz dieser Werkzeuge in realen Lehrkontexten untersuchen. Zukünftige Arbeiten sollten daher (1) genauere und weniger subjektive Maße zur Einschätzung der Relevanz nutzen, (2) die Nutzungsdauer und Interaktionsmuster als Indikatoren für Qualität und Motivation erfassen und (3) den schulischen Bildungsbereich stärker berücksichtigen, da Simulatoren bislang vorwiegend in der Hochschullehre verankert sind.

Darauf aufbauend ergibt sich eine Forschungsagenda, die von methodisch fundierten Feldstudien bis hin zur Entwicklung klarer Kategorien reicht, mit denen sich die Unterschiede zwischen \enquote{Gamified Learning} und \enquote{Game-Based Learning} systematisch erfassen lassen. Für die Praxis bedeutet dies, dass Lehrende bereits heute auf eine Reihe Gestaltungsprinzipien zurückgreifen können, während die Forschung in den kommenden Jahren verstärkt Evidenz für Wirksamkeit und Transferfähigkeit liefern sollte. Auf diese Weise trägt die Arbeit sowohl zur theoretischen Fundierung als auch zur praktischen Weiterentwicklung didaktischer Simulatoren im Bereich der Rechnerarchitektur bei.

