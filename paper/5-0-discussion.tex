\chapter{Diskussion und Best Practices}

% E-Learning
E-Learning ist nicht das gelbe vom Ei\label{elearning}. Niegemann arbeitet in seinem Lehrbuch auf dem Jahr 2009 heraus, dass die Euphorie für E-Learning etwas seit 2002 abzuflachen schien, da viele Erwartungen nicht erfüllt würden.\parencite[S.~14]{niegemann_kompendium_2008}

\begin{itemize}
	\item Kosteneinsparungen könnten häufig nicht erzielt werden, sodass didaktisch wertvolle Lernsoftware mit erhöhten finanziellen Aufwendugen verbunden sei.
	\item Ob Texte und Bilder nun in einer Lernsoftware oder in einem Vorlesungsskript dargestellt werde, weise keinen großen Unterschied auf, da es lediglich Texte und Bilder blieben.
	\item Eine erhöhte Abbrruchrate beim E-Learning sei zu verzeichnen, da die Lernenden das selbstständige Lernen nicht gewohnt seien und keine hinreichende tutorielle Unterstützung erhielten.
	\item Der Zeitaufwand sei beim problembasierten bzw. entdeckenden Lernen in multimedialen Lernumgebungen hoch, sodass auch bei nachgewiesener Effektivität dieser Medien nicht alle Lerninhalte vermittelt werden könnten; die Effektivität des problembasierten Lernens sei zudem keineswegs belegt.
	\item In virtuellen Arbeitsgruppen im Web zeigten sich genau die gleichen Probleme, die von Präsenzarbeitsgruppen bekannt seien.
\end{itemize}

Trotz der oben genannten Schwierigkeiten verdeutlicht Niegemann, dass sich E-Learning als Lehr- und Lernform etabliert habe. Quelle der Probleme sei hier die mangelnde didaktische Konzeption dieser Programme.\parencite[S.~14]{niegemann_kompendium_2008}

% Mobile Learning
Gleichzeitig stellt Mobile Learning neue Anforderungen an Didaktik, Usability und Gestaltung von Lerninhalten, da Bildschirmgröße, Nutzungskontext und Interaktionsmöglichkeiten stärker variieren als bei Desktop-Systemen.

\section{Trends und Herausforderungen}

\section{Empfehlungen für die Zukunft}
