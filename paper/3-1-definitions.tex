\section{Definitionen und Begrifflichkeiten}

\subsection{Didaktische Simulatoren}
Um den Begriff (didaktischer) \textit{Simulator} zu definieren, muss zunächst das zugrundeliegende Konzept des \textit{Modells} betrachtet werden. White und Ingalls~\parencite[S.~12]{white_introduction_2009} beschreiben ein Modell als eine vereinfachte Abstraktion der Realität, die durch die Auswahl des geeigneten Umfangs und Detaillierungsgrades die relevanten Eigenschaften eines Untersuchungsgegenstands abbildet. Modelle kommen insbesondere dann zum Einsatz, wenn das reale System zu komplex, zu unpraktisch oder zu kostenintensiv wäre, um es direkt zu untersuchen \parencite[S.~5]{banks_what_2008}.

Darauf aufbauend stellt die \textit{Simulation} ein Teilgebiet der Modellbildung dar. Unter Simulation versteht man die Durchführung von Experimenten mit einem Modell, das die wesentlichen Eigenschaften des zugrundeliegenden Systems nachahmt, um dessen Verhalten unter verschiedenen Bedingungen untersuchen zu können \parencites[S.~12]{white_introduction_2009}[S.~6]{banks_what_2008}.

Ein \textit{Simulator} ist schließlich das Werkzeug oder System -- meist in Form von Software --, das diese Simulationen ermöglicht. Er implementiert das Modell und bietet eine Benutzungsumgebung, in der Interaktionen und Experimente mit dem Modell durchgeführt werden können \parencite[S.~304f]{duran_what_2020}.

Die Definition des Begriffs \textit{Simulator} ist eindeutig abzugrenzen von der Bezeichnung \textit{Emulator}. Ein Emulator hingegen strebt eine möglichst detailgetreue Nachbildung eines Zielsystems an, sodass Software oder Peripheriegeräte, die für das Originalsystem entwickelt wurden, unverändert darauf ausgeführt werden können \parencite[S.~1683]{mcgregor_relationship_2002}.

Besonders im Kontext der Lehre kommen sogenannte \textit{didaktische Simulatoren} zum Einsatz. Diese unterscheiden sich von präzisen Forschungs- oder Industriesimulatoren dadurch, dass sie in erster Linie auf Verständlichkeit, Visualisierung und Interaktivität ausgerichtet sind. Ziel ist nicht die vollständige, detailgetreue Nachbildung eines Systems, sondern die Förderung von Lernprozessen durch eine für die Lernenden zugängliche Abstraktion komplexer Sachverhalte \parencites[S.~256]{muller_entwicklung_2020}[S.~1]{nystrom_teaching_2024}.

\subsection{Konzepte digitalen Lernens}

Da im Rahmen der Recherche verschiedene Konzepte des digitalen Lernens identifiziert wurden, werden diese nachfolgend kurz vorgestellt und im Hinblick auf ihre Relevanz für den Einsatz didaktischer Simulatoren erläutert.

\sh{Organisationsformen des digitalen Lernens}
Digitale Medien und Technologien haben in den letzten Jahrzehnten eine Vielzahl an Lernformen hervorgebracht, die sich in Reichweite, Methodik und Grad der Individualisierung unterscheiden (siehe dazu Kapitel~\ref{chap:3-2-development-sim} und Kapitel~\ref{chap:3-3-development_ca}).

Der Begriff \textit{E-Learning} dient als Oberbegriff für alle Formen des Lernens, die digitale Medien sowie Informations- und Kommunikationstechnologien zur Unterstützung oder Durchführung von Lehr- und Lernprozessen einsetzen. Dazu zählen sowohl webbasierte Kurse und Lernplattformen als auch multimediale Materialien wie Videos, interaktive Übungen oder Simulationen \parencite[S.~6]{kerres_mediendidaktik_2018}. Zentrale Merkmale des E-Learnings sind die Orts- und Zeitunabhängigkeit, die Möglichkeit zur Interaktivität sowie der multimediale Charakter \parencite[S.~186f]{sanderson_e-learning_2002}. 

Andere Formen wie \textit{M-Learning} oder \textit{Blended Learning} stellen spezifische Ausprägungen dieses übergeordneten Begriffs dar \parencites[S.~74]{magenheim_blended_2003}[S.~3]{balaji_perspective_2016}. \textit{Mobile Learning} (kurz: M-Learning) wird in der Literatur häufig als eine Erweiterung bzw. neue Ausprägung des E-Learning verstanden, die durch den Einsatz mobiler Endgeräte wie Smartphones, Tablets oder Notebooks über drahtlose Netzwerke ermöglicht wird \parencites[S.~3f]{balaji_perspective_2016}[S.~197]{basak_kumar_e-learning_2018}. M-Learning bezeichnet damit den Einsatz mobiler Technologien zur Unterstützung von Lernprozessen und kann als Schnittstelle zwischen Online-Lernen und mobiler Computertechnologie betrachtet werden \parencite[S.~265]{traxler_defining_2005}.

Während \textit{M-Learning} in erster Linie die technische Mobilität des Lernens betont \parencites[S.~3f]{balaji_perspective_2016}[S.~197]{basak_kumar_e-learning_2018}, stellt \textit{Blended Learning} (im deutschsprachigen Raum auch als \textit{hybrides Lernen} oder \textit{vermischter Unterricht} bezeichnet) eine Weiterentwicklung dar, indem Methoden der Präsenzlehre mit Ansätzen des E-Learning verknüpft werden. Ziel ist es, Lernprozesse zu fördern, in denen multimediale Materialien effektiv in individuelle sowie kooperative Lernphasen integriert werden \parencites[S.~74]{magenheim_blended_2003}[S.~29]{pfeiffer_simulationsumgebungen_2008}. In der Weiterentwicklung zu \textit{Blended Learning~2.0} werden klassische Präsenzformate noch stärker mit digitalen und hybriden Komponenten kombiniert. Kennzeichnend ist dabei der verstärkte Einsatz von Web~2.0-Technologien und sozialen Medien, wodurch flexible und personalisierbare Lernsettings entstehen, die sowohl selbstgesteuertes als auch kooperatives Lernen unterstützen \parencites{seufert_schulleitertagung_2014}{news_aktuell_gmbh_e-learning_2025}.

Im Gegensatz zum Blended Learning, das Präsenz- und Onlineformate kombiniert, stehen \textit{\acp{MOOC}}, die internetbasierte Lehrveranstaltungen für eine sehr große Zahl von Teilnehmenden anbieten. Charakteristisch sind multimediale Inhalte, Online-Übungen, Diskussionsforen sowie Peer- und Selbstbewertungen. Trotz ihrer Offenheit folgen MOOCs einem strukturierten Lehrplan mit klar definierten Lernzielen \parencites[S.~5]{yuan_moocs_2013}[S.~204]{liyanagunawardena_moocs_2013}.

Eng verknüpft mit diesem Konzept sind \textit{\acp{OER}}, die frei zugängliche und offen lizenzierte Lehr- und Lernmaterialien bereitstellen. Sie können kostenfrei genutzt, angepasst und weiterverbreitet werden und eröffnen damit vielfältige Möglichkeiten zur gemeinsamen Gestaltung von Lernangeboten. Maßgeblich sind hierbei die sogenannten \enquote{5R-Rechte} (\textit{retain, reuse, revise, remix, redistribute}), die den Grad der Offenheit bestimmen \parencite[S.~134f]{wiley_defining_2018}.


\sh{Lerntechnologien \& -umgebungen}
Neben didaktischen Konzepten spielt auch die technologische Entwicklung eine zentrale Rolle bei der Gestaltung digitaler Lernumgebungen. Neue Technologien eröffnen dabei nicht nur innovative Interaktionsmöglichkeiten, sondern auch datenbasierte Ansätze zur Analyse von Lernprozessen sowie virtuelle Räume, in denen praxisnahes Lernen ermöglicht wird \parencite[S.1]{celik_technology_2024}.

\textit{Immersive Technologien} fassen Ansätze wie \ac{AR}, \ac{VR} und \ac{MR} zusammen, die es Lernenden ermöglichen, in digitale Umgebungen einzutauchen und dort interaktiv zu agieren. Häufig werden diese Technologien auch unter dem Oberbegriff \ac{XR} diskutiert \parencites[S.~82]{alnagrat_review_2022}[S.~256]{chen_information_2024}. In der Lehre finden immersive Technologien zunehmend Anwendung, da sie Motivation und Interaktivität fördern und hochgradig effektive Lernumgebungen schaffen können \parencite[S.~1]{izouaouen_education_2025}. Im Gegensatz zu dieser interaktiven Lernform, stehen die datengetriebenen \textit{Learning Analytics}.

Hierbei bezeichnen \textit{Learning Analytics} die Erfassung, Sammlung, Analyse und Berichterstattung von Daten über Lernende und deren Kontexte mit dem Ziel, Lernprozesse sowie die Lernumgebungen, in denen sie stattfinden, besser zu verstehen und zu optimieren. Besondere Potenziale ergeben sich aus der Aufdeckung bislang verborgener Informationen in den Daten sowie aus deren gezielter Nutzung, etwa für didaktische Interventionen oder zur Vorhersage von Lernverläufen \parencite[S.~294]{xiao_applying_2019}.

Ergänzend zu diesen Ansätzen bieten \textit{Virtual Labs} konkrete Umgebungen, in denen Lernende Wissen anwenden und experimentell vertiefen können – oftmals unter Einsatz sowohl immersiver Technologien als auch von Learning-Analytics Methoden. \textit{Virtuelle Labs} sind computerbasierte, interaktive Umgebungen, die es ermöglichen, Aufgaben auszuführen, die normalerweise in einem physischen Labor stattfinden würden. Über entsprechende Benutzeroberflächen können Simulationen, Animationen und teilweise sogar die Fernsteuerung realer Laborhardware erfolgen. Zahlreiche Studien haben den Einsatz virtueller Labore als Lehr- und Lerninstrument untersucht und ihre Wirksamkeit in nahezu allen Fällen bestätigt \parencite[S.~117]{achuthan_value_2011}.

In den letzten Jahren haben sich virtuelle Labore und Remote-Experimente durch Fortschritte in Webtechnologien und Anwendungen weiterentwickelt. Ziel ist es, die Erfahrungen eines klassischen Präsenzlabors möglichst realitätsnah abzubilden und dabei einen vergleichbaren Grad an Zugriff, Funktionalität und Flexibilität zu gewährleisten. Der Einsatz von virtuellen Welten und Mixed-Reality-Technologien eröffnet zudem neue Möglichkeiten für kollaboratives Arbeiten in immersiven 3D-Umgebungen, in denen Lernende komplexe Simulationen und Datensätze interaktiv erkunden und visualisieren können \parencite[S.~1]{savin-baden_understanding_2012}.

% feingradige Unterscheidung hier nicht notwendig, da nur auf Gamification eingegangen wird in der Literaturrecherche
\iffalse
\sh{Didaktische Konzepte zur Motivation}
Zur Förderung von Motivation und Engagement im Lernprozess haben sich in der Literatur verschiedene Konzepte etabliert, die spielerische Elemente auf unterschiedliche Weise nutzen \parencite[S.~485]{triantafyllou_gamification_2025}. Die zentralen Ansätze sind Gamification, Gamified Learning und Game-Based Learning. Sie unterscheiden sich darin, ob Spielelemente lediglich in bestehende Lernumgebungen integriert werden oder ob das Spiel selbst den didaktischen Kern bildet.

\textit{Gamification} bezeichnet den Ansatz, Designelemente aus (Video-)Spielen in nicht-spielerische Kontexte zu übertragen \parencites[S.~2]{deterding_gamification_2011}[S.~9]{kapp_gamification_2012}. Durch den Einsatz solcher Spielelemente sollen Lernprozesse attraktiver gestaltet und Motivation sowie kognitives Engagement gefördert werden. Ein höheres Maß an Involviertheit kann dabei aus hochschuldidaktischer Sicht zu einem aktiveren und nachhaltigeren Lernen führen \parencites[S.~97ff]{chi_active-constructive-interactive_2009}[S.~1821]{chi_translating_2018}.

Der Begriff Gamification geht auf den Software-Entwickler Nick Pelling zurück, der Anfang der 2000er Jahre eine spieleähnliche Benutzeroberfläche für Bank- und Verkaufsautomaten entwarf \parencites{pelling_short_2011}[S.~2f]{deterding_gamification_2011}. In den folgenden Jahren haben sich neben Gamification auch verwandte Konzepte wie \textit{Gamefulness}, \textit{Gameful Design} oder \textit{Playful Interaction Design} etabliert, die sich teilweise überschneiden, jedoch unterschiedliche Akzentuierungen aufweisen \parencite[S.~2f]{deterding_gamification_2011}. Deterding et al. \parencite[S.~2f]{deterding_gamification_2011} haben eine weit verbreitete Definition geprägt, die das Spiel als konstitutive Einheit betont und es von der allgemeinen Spielfreude (\textit{Playfulness}) abgrenzt [S.~452f]{schlag_gamifizierung_2021}.

Zu den typischen Gestaltungselementen der Gamification zählen unter anderem Punktesysteme, Ranglisten, Abzeichen, Belohnungen und Fortschrittsanzeigen. Sie sollen Lernende motivieren, indem sie Leistung sichtbar machen und Anreize zur weiteren Beschäftigung mit den Inhalten schaffen. Da sich die Spieleindustrie sowie die Anwendungsszenarien im Bildungsbereich kontinuierlich verändern, ist die Auswahl möglicher Spielelemente jedoch nicht abschließend festgelegt, sondern unterliegt einem dynamischen Wandel \parencite[S.~2f]{hamari_does_2014}.

Während Gamification also vor allem die Übertragung einzelner Spielelemente in nicht-spielerische Kontexte beschreibt, geht \textit{Gamified Learning} einen Schritt weiter und rückt das didaktische Gesamtdesign in den Vordergrund. Unter Gamified Learning wird der gezielte Einsatz von Spielelementen in Lehr- und Lernszenarien verstanden, wobei der didaktische Rahmen im Vordergrund steht. Die Theorie des gamifizierten Lernens beschreibt dieses Zusammenspiel anhand mehrerer Dimensionen: Der Instruktion selbst, den Einstellungen und Verhaltensweisen der Lernenden, den eingesetzten Spielelementen sowie dem daraus resultierenden Lernerfolg \parencites[S.~6f]{landers_developing_2014}[S.~453]{schlag_gamifizierung_2021}.

Gamification kann dabei auf zwei Wegen wirken: Einerseits kann das Verhalten der Lernenden die Wirksamkeit der Instruktion verstärken oder abschwächen (moderierender Effekt). Andererseits können die durch Spielelemente angestoßenen Einstellungen und Verhaltensweisen selbst zum Lernerfolg beitragen (mediierender Effekt) \parencites[S.~6f]{landers_developing_2014}[S.~453]{schlag_gamifizierung_2021}.

Damit Gamified Learning erfolgreich ist, müssen Spielelemente also so gestaltet werden, dass sie lernförderliche Verhaltensweisen hervorrufen und in ein didaktisch sinnvolles Instruktionsdesign eingebettet sind \parencites[S.~6f]{landers_developing_2014}[S.~453]{schlag_gamifizierung_2021}.

Im Gegensatz zu Gamification und Gamified Learning, bei denen Spielelemente in bestehende Lernumgebungen integriert werden, steht beim \textit{Game-Based Learning} das Spiel selbst im Zentrum des didaktischen Prozesses. Denn Game-Based Learning bezeichnet den Einsatz von digitalen Spielen mit didaktischer Zielsetzung im Lehr- und Lernkontext. Im deutschsprachigen Raum wird hierfür häufig auch der Begriff \textit{Serious Games} verwendet, womit digitale, multimediale Spiele gemeint sind, die nicht primär der Unterhaltung dienen, sondern einen \enquote{ernsthaften} Hintergrund verfolgen \parencite[S.~14]{niegemann_kompendium_2008}. 
\fi

\sh{Didaktische Konzepte zur Motivation}
Zur Förderung von Motivation und Engagement im Lernprozess haben sich in der Literatur verschiedene Konzepte etabliert, die spielerische Elemente aufgreifen \parencite[S.~485]{triantafyllou_gamification_2025}. Dazu zählen \textit{Gamification}, \textit{Gamified Learning} und \textit{Game-Based Learning}. Sie unterscheiden sich darin, ob einzelne Spielelemente in bestehende Lernumgebungen integriert oder ganze Spiele als didaktischer Kern eingesetzt werden \parencites[S.~2]{deterding_gamification_2011}[S.~14]{niegemann_kompendium_2008}. Typische Gestaltungselemente sind etwa Punktesysteme, Ranglisten oder Abzeichen, die Motivation und Engagement der Lernenden fördern sollen \parencite[S.~2f]{hamari_does_2014}. 

Im Rahmen der Literaturrecherche wurde zwischen den Begrifflichkeiten nicht differenziert, sondern lediglich geprüft, ob ein Simulator spielerische Elemente aufweist.

\sh{Individuelle \& adaptive Lernformen}
Im Kontext digitaler Lehre gewinnen Konzepte zunehmend an Bedeutung, die Lernprozesse stärker an die individuellen Voraussetzungen, Bedürfnisse und Ziele der Lernenden anpassen. Dazu zählen insbesondere \textit{Microlearning}, \textit{Adaptive Learning}, \textit{Personalized Learning} sowie \textit{Collaborative Learning}. 

\textit{Microlearning} segmentiert Inhalte in kurze, fokussierte Einheiten, die flexibel und kontextabhängig genutzt werden können \parencite[S.~74]{chong_mvr-cls_2022}. Aufbauend darauf ermöglichen \textit{Adaptive Learning}-Systeme eine dynamische Anpassung von Inhalten und Lernpfaden an den Wissensstand der Lernenden \parencite[S.~448]{zhao_research_2019}. \textit{Personalized Learning} verfolgt einen umfassenderen Ansatz, der neben technologischen auch didaktische und organisatorische Anpassungen umfasst, um Motivation, Selbststeuerung und nachhaltigen Lernerfolg zu fördern \parencites[S.~6ff]{pane_informing_2017}[S.~2f]{gunawardena_personalized_2024}[S.~236ff]{walkington_appraising_2020}. Ergänzend hebt \textit{Collaborative Learning} die Bedeutung sozialer Interaktion hervor, bei der Wissen durch Austausch und Kooperation gemeinschaftlich konstruiert wird \parencite[S.~486]{laal_benefits_2021}.

% Themen werden in der Arbeit nicht diskutiert, das sehr kurz gehalten
\iffalse
Im Kontext digitaler Lehre gewinnen Konzepte zunehmend an Bedeutung, die Lernprozesse stärker an die individuellen Voraussetzungen, Bedürfnisse und Ziele der Lernenden anpassen. Zu diesen zählen Ansätze wie Microlearning, Adaptive Learning und Personalized Learning, die eine inhaltliche oder methodische Individualisierung anstreben, sowie Collaborative Learning, das den sozialen Austausch als wichtigen Bestandteil des Lernprozesses hervorhebt.

\textit{Microlearning} bezeichnet eine Form des digitalen Lernens, bei der Inhalte in kleine, leicht verdauliche Einheiten segmentiert werden. Ziel ist es, Lernenden einen schnellen und fokussierten Zugriff auf Wissen zu ermöglichen, der sich flexibel in ihren Alltag integrieren lässt. Typische Merkmale sind die Kürze der Lerneinheiten, ihre thematische Fokussierung sowie die Möglichkeit, sie in unterschiedlichen Kontexten, etwa mobil oder \enquote{on demand}, zu nutzen \parencite[S.~74]{chong_mvr-cls_2022}.

Ein sogenannter \textit{Microlearning Service} beschreibt, wie solche Lernressourcen systematisch bereitgestellt und an die individuellen Interessen der Lernenden angepasst werden können. Dieser Prozess umfasst drei wesentliche Schritte: Die Segmentierung von Inhalten in kleine Lerneinheiten (1), die Annotation dieser Einheiten mit Metadaten (2) und die Empfehlung passender Inhalte an die jeweiligen Nutzerinnen und Nutzer (3) \parencite[S.~152--154]{lin_survey_2019}.

Microlearning zeichnet sich vor allem durch die Reduktion und flexible Aufbereitung von Lerninhalten aus, während adaptive Lernsysteme zusätzlich Inhalte dynamisch an den jeweiligen Wissensstand der Lernenden anpassen.

\textit{Adaptive Learning} bezeichnet Lernsysteme, die Inhalte, Aufgaben und Lernpfade automatisch an die individuellen Bedürfnisse der Lernenden anpassen. Grundlage ist in der Regel eine Diagnose des aktuellen Wissensstandes sowie relevanter Lernmerkmale, auf deren Basis das System Schwierigkeitsgrad, Reihenfolge und Auswahl der Lerninhalte dynamisch steuert. Ergänzt wird dies durch kontinuierliches Feedback und eine laufende Anpassung des Lernprozesses, sodass Lernende dort abgeholt werden, wo sie stehen, und in ihrem individuellen Tempo gefördert werden können. Ziel adaptiver Lernsysteme ist es, Lernumgebungen effizienter, personalisierter und motivierender zu gestalten \parencite[S.~448]{zhao_research_2019}.

Während Adaptive Learning vor allem die technologische Umsetzung individueller Anpassungen in den Blick nimmt, verfolgt Personalized Learning einen umfassenderen Ansatz, der auch didaktische und organisatorische Aspekte einschließt.

\textit{Personalized Learning} bezeichnet Lehr- und Lernformen, die sich stark an den individuellen Bedürfnissen, Interessen und Zielen der Lernenden orientieren. Im Vordergrund steht dabei die Anpassung von Lerninhalten, Methoden und Lernwegen an persönliche Voraussetzungen, Stärken und Schwächen. Während adaptive Systeme eine technologische Umsetzung dieser Idee ermöglichen, umfasst personalisiertes Lernen darüber hinaus auch didaktische und organisatorische Maßnahmen, etwa die Möglichkeit zur Wahl von Lernpfaden, zur Anpassung des Lerntempos oder zur Bearbeitung individueller Projekte. Ziel ist es, Lernumgebungen zu schaffen, die Motivation, Selbststeuerung und nachhaltigen Lernerfolg fördern \parencites[S.~6ff]{pane_informing_2017}[S.~2f]{gunawardena_personalized_2024}[S.~236--239]{walkington_appraising_2020}.

Neben diesen auf Individualisierung ausgerichteten Konzepten betont \textit{Collaborative Learning} den Wert gemeinsamer Interaktion und Wissenskonstruktion.

\textit{Collaborative Learning} bezeichnet eine Lehr- und Lernmethode, bei der Lernende gemeinsam an Aufgaben oder Problemstellungen arbeiten und ihr Wissen aktiv miteinander austauschen. Im Vordergrund steht nicht nur der individuelle Lernerfolg, sondern auch die gemeinsame Verantwortung innerhalb der Gruppe sowie die gegenseitige Unterstützung. Durch Diskussion, Interaktion und Kooperation wird Wissen nicht nur reproduziert, sondern gemeinsam konstruiert, wodurch sowohl kognitive als auch soziale Kompetenzen wie Kommunikation, Teamfähigkeit und Verantwortungsbewusstsein gefördert werden \parencite[S.~486]{laal_benefits_2021}.
\fi

\subsection{Lernpsychologische Grundlagen}\label{chap:3-1-psychology}

Nachfolgend werden relevante lernpsychologische Theorien vorgestellt. Eine komprimierte Übersicht findet sich in Tabelle~\ref{tab:lernpsychologie}.

\sh{Behaviorismus}
Der Behaviorismus versteht Lernen als eine Veränderung im beobachtbaren Verhalten, die durch Erfahrungen mit der Umwelt hervorgerufen wird. Zentrale Mechanismen sind die klassische und die operante Konditionierung \parencite[S.~15]{pfeiffer_simulationsumgebungen_2008}.

Die \textit{klassische Konditionierung}, die auf Pawlow zurückgeht, beschreibt Lernprozesse, bei denen ein ursprünglich neutraler Stimulus durch wiederholte Kopplung mit einem unkonditionierten Stimulus die Fähigkeit erlangt, dieselbe Reaktion auszulösen wie der unkonditionierte Stimulus selbst. Ein bekanntes Beispiel ist Pawlows Experiment mit Hunden, bei dem ein Glockenton (neutraler Stimulus) mit der Darbietung von Futter (unkonditionierter Stimulus) kombiniert wurde, sodass schließlich allein der Ton eine Speichelreaktion hervorrief \parencite[S.~7ff]{furstenau_lehr-lern-theorien_2019}.

Die \textit{operante Konditionierung}, entwickelt von Skinner, knüpft an Thorndikes \textit{Law of Effect} an, das besagt, dass Verhaltensweisen, die zu befriedigenden Konsequenzen führen, mit höherer Wahrscheinlichkeit erneut gezeigt werden. Im Zentrum steht hier die Beziehung zwischen Verhalten und Konsequenzen: Folgt auf ein Verhalten eine Verstärkung, steigt die Wahrscheinlichkeit, dass es unter ähnlichen Bedingungen wiederholt wird. Skinner unterscheidet dabei zwischen respondenten Verhaltensweisen, die durch einen Stimulus ausgelöst werden, und operanten Verhaltensweisen, die aktiv auf die Umwelt einwirken. Während klassische Konditionierung das Erlernen respondenten Verhaltens erklärt, bezieht sich die operante Konditionierung auf die Steuerung und Veränderung operanten Verhaltens \parencite[S.~15ff]{furstenau_lehr-lern-theorien_2019}.

\sh{Kognitivismus}
Der Kognitivismus versteht Lernen wiederum als einen Prozess der aktiven Informationsaufnahme, -verarbeitung und -speicherung. Unter Kognition werden sämtliche \enquote{Prozesse des Wissens} verstanden, darunter Denken, Erinnerung, Sprache, Kreativität und Wahrnehmung. Lernende gelten in dieser Perspektive nicht als passiv durch äußere Reize gesteuert, sondern als aktive Individuen, die Informationen selbstständig verarbeiten und in bestehende Wissensstrukturen integrieren \parencite[S.~1]{furstenau_lehr-lern-theorien_2019}.

\begin{figure}[htbp]
    \centering
    \includegraphics[width=0.90\textwidth]{img/Computermodell.png}
    \caption{Computermodell von Bower \& Hildgard}~\parencite[S.~234]{bower_theorie_1984}
    \label{fig:computermodell}
\end{figure}

Eine wichtige Grundlage bildet das Computermodell von Bower \& Hilgard, das die menschliche Informationsverarbeitung analog zu einem Informationssystem beschreibt (siehe Abbildung~\ref{fig:computermodell}). Hierbei werden eingehende Reize zunächst sensorisch aufgenommen, im Arbeitsgedächtnis verarbeitet und schließlich im Langzeitgedächtnis gespeichert. Die Kapazität des Arbeitsgedächtnisses ist dabei sowohl zeitlich als auch räumlich begrenzt: Informationen verfallen ohne Wiederholung nach wenigen Sekunden und können nur in einer beschränkten Anzahl gleichzeitig präsent gehalten werden. Daraus ergibt sich die Notwendigkeit, Inhalte zu strukturieren und durch Prozesse wie \enquote{Chunking} mit Vorwissen zu verknüpfen \parencite[S.~15]{pfeiffer_simulationsumgebungen_2008}.

Eine der einflussreichsten Theorien des multimedialen Lernens innerhalb des kognitivistischen Paradigmas ist die \textit{\ac{CTML}} von Mayer \parencite[S.~102ff]{mayer_multimedia_2001}. Sie basiert auf drei zentralen Annahmen:
\begin{enumerate}
	\item Informationen werden über getrennte Kanäle verarbeitet – einen visuellen und einen auditiven (Dual Channel),
	\item die Verarbeitungskapazität ist in beiden Kanälen begrenzt (Limited Capacity),
	\item Lernen erfordert aktive kognitive Verarbeitung, bei der Lernende neue Informationen mit vorhandenem Wissen verknüpfen (Active Processing).
\end{enumerate}

Ergänzend stützt sich die Theorie auf Paivios \textit{Dual-Coding-Theory} \parencite[S.~102f]{paivia_dual_2006}, nach der verbale und nonverbale Informationen in getrennten, aber miteinander verknüpften Repräsentationssystemen verarbeitet werden \parencite[S.~66f]{furstenau_lehr-lern-theorien_2019}.

\sh{Konstruktivismus}
\textit{Konstruktivistische Lerntheorien} verstehen Lernen als aktive Konstruktion von Wissen durch die Lernenden. Im Unterschied zu kognitivistischen Ansätzen wird dabei die Eigenaktivität und Selbststeuerung des Lernprozesses besonders betont. Lernen ist zudem stets in soziale und kulturelle Kontexte eingebettet und wird als situierte Kognition aufgefasst, die in authentischen Interaktionen und realitätsnahen Lernumgebungen stattfindet \parencite[S.~1f]{furstenau_lehr-lern-theorien_2019}.

Auch wenn Mayer die \ac{CTML} klar im kognitivistischen Paradigma verortet, lassen sich in ihr konstruktivistische Elemente wiederfinden – etwa die Betonung der aktiven Auseinandersetzung der Lernenden mit den dargebotenen Inhalten. Digitale Lernangebote wie E-Learning-Kurse oder Lernplattformen orientieren sich häufig an diesen Prinzipien, indem sie Inhalte in überschaubare Einheiten gliedern, visuelle und textuelle Informationen aufeinander abstimmen und strukturierende Hilfen bereitstellen. Damit trägt der Kognitivismus entscheidend dazu bei, digitales Lernen wirksam und lernförderlich zu gestalten \parencites[S.~105f]{mayer_multimedia_2001}{mayer_mayers_nodate}.

\sh{Erfahrungsbasierte Lerntheorie}
Die erfahrungsbasierte Lerntheorie - \textit{\ac{ELT}} - nach Kolb versteht Lernen als einen kognitiven Prozess, der durch die ständige Anpassung an und Auseinandersetzung mit der Umwelt geprägt ist. Wissen wird dabei nicht bloß durch Instruktion übernommen, sondern aktiv aus Erfahrung erzeugt \parencite[S.~30]{bergsteiner_kolbs_2010}.

Grundlage ist ein zyklisches Modell (siehe Abbildung~\ref{fig:etl_cycle}), das vier Phasen umfasst: Konkrete Erfahrung (\textit{Concrete Experience}), reflektierende Beobachtung (\textit{Reflective Observation}), abstrakte Konzeptualisierung (\textit{Abstract Conceptualization}) und aktives Experimentieren (\textit{Active Experimentation}). Effektives Lernen setzt voraus, dass Lernende diese Phasen wiederholt durchlaufen, indem sie Erfahrungen machen, reflektieren, in abstrakte Konzepte überführen und in neuen Handlungssituationen anwenden. Damit versteht die ELT Lernen als kreativen Prozess der Wissenskonstruktion, in dem Lernende je nach Situation flexibel zwischen verschiedenen Lernfähigkeiten wählen \parencite[S.~2f]{mccarthy_experiential_2010}.

\begin{figure}[htbp]
    \centering
    \includegraphics[width=0.90\textwidth]{img/ELT_cycle.png}
    \caption{The Experimental Learning Cycle}~\cite{mcleod_kolbs_2025}
    \label{fig:etl_cycle}
\end{figure}

Gerade in digitalen Lernumgebungen wie Simulatoren oder virtuellen Laboren zeigt sich die Relevanz dieses Ansatzes, da hier konkrete Erfahrungen in einem geschützten Rahmen gesammelt und in abstrahierte Konzepte überführt werden können \parencites[S.~3182]{reyes_enhancing_2024}[S.~7]{bazie_effect_2024}.

\sh{Exploratives Lernen}
\textit{Exploratives Lernen} bezeichnet einen Lernansatz, bei dem Lernende durch eigenständiges Erforschen, Ausprobieren und Entdecken Wissen erwerben \parencite[S.~15]{grabinger_rich_2016}. Diese Theorie hat ihren Ursprung im Konstruktivismus \parencite[S.~271]{kornelsen_expedition_2005}. Im Gegensatz zu instruktionsgesteuerten Formen liegt der Fokus hier auf dem aktiven Handeln und dem selbstständigen Generieren von Problemlösungen. Der Lernprozess ist dabei offen, dynamisch und oft von Versuch und Irrtum geprägt, wodurch Lernende ihre eigenen Hypothesen entwickeln und überprüfen können \parencite[S.~141]{lucke_strukturierte_2005}.

Für digitale Lernumgebungen, insbesondere für didaktische Simulatoren und Virtual Labs, ist exploratives Lernen von besonderer Relevanz. Diese Werkzeuge schaffen sichere Rahmenbedingungen, in denen Lernende komplexe Sachverhalte eigenständig erproben, Fehler gefahrlos machen und daraus unmittelbar Rückmeldungen erhalten können. Auf diese Weise fördern Simulatoren nicht nur die aktive Auseinandersetzung mit Lerninhalten, sondern auch Problemlösefähigkeiten, Motivation und nachhaltiges Verständnis \parencite[S.~131]{engelhardt-nowitzki_einsatz_2006}.

\sh{Konnektivismus}
Der \textit{Konnektivismus} wurde von Siemens~\cite{siemens_connectivism_2005} und Downes~\cite{downes_introduction_2008} als Theorie des digitalen Lernens entwickelt, die klassische Ansätze wie Behaviorismus, Kognitivismus und Konstruktivismus ergänzt. Er versteht Wissen nicht mehr ausschließlich als individuelle Ressource, sondern als Netzwerk aus Menschen, digitalen Informationen und Technologien. Lernen bedeutet in diesem Verständnis, relevante Wissensknoten zu identifizieren, Verbindungen herzustellen und Netzwerke kontinuierlich zu pflegen \parencite[S.~14]{abbas_theory_2015}.

In Bezug auf digitale Bildung ist der Konnektivismus besonders für technologiegestützte Lernformen wie MOOCs, soziale Plattformen oder offene Bildungsressourcen bedeutsam \parencite[S.~14]{abbas_theory_2015}.

\sh{Multimediales Lernen}
Was unter \textit{multimedialem Lernen} verstanden wird, hängt stark von der zugrunde liegenden theoretischen Perspektive ab. Je nach Schwerpunkt kann das Verständnis eher behavioristisch, kognitivistisch oder konstruktivistisch geprägt sein \parencite[S.~64]{furstenau_lehr-lern-theorien_2019}. 

\textit{Lernen als Reaktionsverstärkung}: Aus behavioristischer Sicht dient multimediales Lernen vor allem dazu, bestimmte Reiz-Reaktions-Muster durch Wiederholung zu verfestigen. Typische Beispiele sind Vokabeltrainer oder Programme zum Erlernen des Zehnfingersystems. Die Rolle der Lernumgebung besteht darin, passende Stimuli bereitzustellen, richtige Antworten zu belohnen und falsche entsprechend zu sanktionieren \parencite[S.~64]{furstenau_lehr-lern-theorien_2019}. 

\textit{Lernen als Informationsaufnahme und -verarbeitung}: In einem eher kognitivistischen Verständnis geht es darum, neues Wissen zu sammeln und mit bereits vorhandenem Vorwissen zu verknüpfen. Multimedia-Umgebungen fungieren hier als Quellen, die Informationen in strukturierter Form bereitstellen, etwa durch digitale Texte, Videos oder Webseiten \parencite[S.~64]{furstenau_lehr-lern-theorien_2019}. 

\textit{Lernen als aktive Wissenskonstruktion}: Konstruktivistisch geprägte Ansätze betonen dagegen die Eigenaktivität der Lernenden. Wissen wird nicht passiv aufgenommen, sondern auf Grundlage des Vorwissens aktiv interpretiert und weiterentwickelt. Multimediales Lernen bedeutet in diesem Sinne, dass durch die Kombination von Texten, Bildern oder Animationen neue mentale Repräsentationen aufgebaut oder bestehende modifiziert werden \parencite[S.~64]{furstenau_lehr-lern-theorien_2019}.

In der Praxis zeigt sich jedoch, dass multimediales Lernen häufig nicht eindeutig einer dieser theoretischen Strömungen zugeordnet werden kann. Stattdessen vereinen viele Lernumgebungen Elemente verschiedener Ansätze, etwa indem sie behavioristische Übungsphasen mit kognitivistisch fundierter Informationsvermittlung und konstruktivistisch orientierten Erkundungsmöglichkeiten kombinieren. Dieser sogenannte \textit{Blended Theoretical Approach} verdeutlicht, dass moderne digitale Lernumgebungen typischerweise eine Synthese mehrerer Lerntheorien darstellen, um unterschiedlichen Anforderungen und Lernbedürfnissen gerecht zu werden \parencites[S.~186]{picciano_theories_2021}[S.~37]{liu_theoretical_2024}.

{
\small
\centering
\begin{longtable}{p{3cm}p{8.25cm}}
    \caption{Zusammenfassung der lernpsychologischen Theorien\label{tab:lernpsychologie}}\\
    \toprule
    \textbf{Theorie / Ansatz} & \textbf{Zentrale Erkenntnisse} \\
    \midrule
    \endfirsthead

    \multicolumn{2}{c}%
    {{\bfseries Fortsetzung von Tabelle~\thetable}} \\
    \toprule

    \textbf{Theorie / Ansatz} & \textbf{Zentrale Erkenntnisse} \\
    \midrule
    \endhead

    \midrule \multicolumn{2}{l}{{Fortsetzung auf der nächsten Seite}} \\
    \endfoot

    \bottomrule
    \endlastfoot

    Behaviorismus & Lernen = Veränderung beobachtbaren Verhaltens durch Reiz-Reaktions-Muster. Klassische Konditionierung (Pawlow) und operante Konditionierung (Skinner) als Hauptmechanismen; Fokus auf äußeres Verhalten. \\

    Kognitivismus & Lernen = aktive Informationsaufnahme, -verarbeitung und -speicherung. Computermodell (Bower \& Hilgard) mit sensorischem Register, Arbeits- und Langzeitgedächtnis. Begrenzte Kapazität erfordert Strukturierung (Chunking). Zentrale Theorie: \ac{CTML} (Mayer) mit Dual-Channel, Limited Capacity, Active Processing. Ergänzend: Dual-Coding-Theory (Paivio). \\

    Konstruktivismus & Lernen = aktive Konstruktion von Wissen. Betonung von Eigenaktivität, Selbststeuerung und situierter Kognition in sozialen und kulturellen Kontexten. Relevanz für digitale Lernangebote durch Gliederung, Abstimmung visueller und textueller Informationen sowie Strukturierungshilfen. \\

    Erfahrungsbasierte Lerntheorie & Lernen als kreativer Prozess durch Erfahrung (Kolb). Zyklus: konkrete Erfahrung – reflektierende Beobachtung – abstrakte Konzeptualisierung – aktives Experimentieren. Besonders relevant für digitale Lernumgebungen wie Simulatoren, die geschützte Erfahrungsräume bieten. \\

    Exploratives Lernen & Wissenserwerb durch Erforschen, Ausprobieren, Entdecken. Fokus auf Eigenaktivität, Problemlösen, Hypothesenbildung. Besonders geeignet für Simulatoren und Virtual Labs, die sicheres Ausprobieren und unmittelbares Feedback ermöglichen. \\

    Konnektivismus & Wissen = Netzwerk aus Menschen, Informationen und Technologien (Siemens, Downes). Lernen = Identifikation relevanter Wissensknoten und Aufbau von Verbindungen. Besonders bedeutsam für MOOCs, soziale Plattformen und offene Bildungsressourcen. \\

    Multimediales Lernen & Verständnis abhängig von theoretischem Zugang: (1) behavioristisch: Reiz-Reaktions-Verstärkung (z.B. Vokabeltrainer). (2) kognitivistisch: Informationsaufnahme und Verknüpfung mit Vorwissen. (3) konstruktivistisch: aktive Wissenskonstruktion durch Kombination von Medien. In der Praxis oft Mischformen als \textit{Blended Theoretical Approach}. \\
\end{longtable}
}