\section{Definition und Begrifflichkeiten}

\subsection{Didaktische Simulatoren}
Um den Begriff (didaktischer) \textit{Simulator} zu definieren, muss zunächst das zugrundeliegende Konzept des \textit{Modells} betrachtet werden. White und Ingalls beschreiben ein Modell als eine vereinfachte Abstraktion der Realität, die durch die Auswahl des geeigneten Umfangs und Detaillierungsgrades die relevanten Eigenschaften eines Untersuchungsgegenstands abbildet. Modelle kommen insbesondere dann zum Einsatz, wenn das reale System zu komplex, zu unpraktisch oder zu kostenintensiv wäre, um es direkt zu untersuchen \parencites[S.~12]{white_introduction_2009}[S.~5]{banks_what_2008}.

Darauf aufbauend stellt die \textit{Simulation} ein Teilgebiet der Modellbildung dar. Unter Simulation versteht man die Durchführung von Experimenten mit einem Modell, das die wesentlichen Eigenschaften des zugrundeliegenden Systems nachahmt, um dessen Verhalten unter verschiedenen Bedingungen untersuchen zu können \parencites[S.~12]{white_introduction_2009}[S.~6]{banks_what_2008}.

Ein \textit{Simulator} schließlich ist das Werkzeug oder System -- meist in Form von Software --, das diese Simulationen ermöglicht. Er implementiert das Modell und bietet eine Benutzungsumgebung, in der Interaktionen und Experimente mit dem Modell durchgeführt werden können \parencite[S.~304f]{duran_what_2020}.

Die Definition des Begriffs \textit{Simulator} ist eindeutig abzugrenzen von der Bezeichnung \textit{Emulator}. Ein Emulator hingegen strebt eine möglichst detailgetreue Nachbildung eines Zielsystems an, sodass Software oder Peripheriegeräte, die für das Originalsystem entwickelt wurden, unverändert darauf ausgeführt werden können \parencite[S.~1683]{mcgregor_relationship_2002}.

Besonders im Kontext der Lehre kommen sogenannte \textit{didaktische Simulatoren} zum Einsatz. Diese unterscheiden sich von hochgradig präzisen Forschungs- oder Industriesimulatoren dadurch, dass sie in erster Linie auf Verständlichkeit, Visualisierung und Interaktivität ausgerichtet sind. Ziel ist nicht die vollständige, detailgetreue Nachbildung eines Systems, sondern die Förderung von Lernprozessen durch eine für die Lernenden zugängliche Abstraktion komplexer Sachverhalte \parencites[S.~256]{muller_entwicklung_2020}[S.~1]{nystrom_teaching_2024}.

\subsection{Konzepte digitalen Lernens}

Da im Rahmen der Literaturrecherche verschiedene Konzepte des digitalen Lernens identifiziert wurden, werden diese nachfolgend kurz vorgestellt und im Hinblick auf ihre Relevanz für den Einsatz didaktischer Simulatoren erläutert.

\subsubsection{Organisationsformen des digitalen Lernens}

\sh{E-Learning}
Der Begriff \textit{E-Learning} dient als Oberbegriff für alle Formen des Lernens, die digitale Medien sowie Informations- und Kommunikationstechnologien zur Unterstützung oder Durchführung von Lehr- und Lernprozessen einsetzen. Dazu zählen sowohl webbasierte Kurse und Lernplattformen als auch multimediale Materialien wie Videos, interaktive Übungen oder Simulationen \parencite[S.~6]{kerres_mediendidaktik_2018}. Zentrale Merkmale des E-Learning sind die Orts- und Zeitunabhängigkeit, die Möglichkeit zur Interaktivität sowie der multimediale Charakter \parencite[S.~186f]{sanderson_e-learning_2002}. 

Andere Formen wie \textit{M-Learning} oder \textit{Blended Learning} stellen spezifische Ausprägungen dieses übergeordneten Begriffs dar \parencites[S.~74]{magenheim_blended_2003}[S.~3]{balaji_perspective_2016}.

\sh{M-Learning}
\textit{Mobile Learning} (kurz: M-Learning) wird in der Literatur häufig als eine Erweiterung bzw. neue Ausprägung des E-Learning verstanden, die durch den Einsatz mobiler Endgeräte wie Smartphones, Tablets, Notebooks oder PDAs über drahtlose Netzwerke ermöglicht wird \parencites[S.~3f]{balaji_perspective_2016}[S.~197]{basak_kumar_e-learning_2018}. 

M-Learning bezeichnet damit den Einsatz mobiler Technologien zur Unterstützung von Lernprozessen und kann als Schnittstelle zwischen Online-Lernen und mobiler Computertechnologie betrachtet werden \parencite[S.~265]{traxler_defining_2005}. 

\sh{Blended Learning}
Blended Learning ist eine Lehr- und Lernform, die Methoden der Präsenzlehre mit Konzepten des E-Learning verbindet. Ziel ist die Förderung von Lernprozessen, in denen multimediale Materialien effektiv in individuelle und kooperative Lernphasen integriert werden können \parencite[S.~74]{magenheim_blended_2003}. Im deutschsprachigen Raum wird Blended Learning auch als \textit{hybrides Lernen} oder als \textit{vermischter Unterricht} bezeichnet \parencite[S.~29]{pfeffer_simulationsumgebungen_2008}.

In der Weiterentwicklung zu \textit{Blended Learning~2.0} werden klassische Präsenzformate noch stärker mit digitalen und hybriden Komponenten verknüpft. Charakteristisch ist hier der verstärkte Einsatz von Web~2.0-Technologien und sozialen Medien, wodurch flexible, personalisierbare Lernsettings entstehen, die sowohl selbstgesteuertes als auch kooperatives Lernen unterstützen \parencites{seufert_schulleitertagung_2014}{news_aktuell_gmbh_e-learning_2025}.

\sh{Massive Open Online Course (MO       extit{Massive Open Online Courses} (MOOCs) sind internetbasierte Lehrveranstaltungen, die in der Regel für eine sehr große Zahl von Teilnehmenden konzipiert sind und ohne formale Zugangsbeschränkungen offen angeboten werden. Charakteristisch ist die Kombination aus multimedialen Inhalten, Online-Übungen, Diskussionsforen sowie Peer- oder Selbstbewertungen. Trotz der Offenheit verfolgen MOOCs einen strukturierten Lehrplan mit klar definierten Lernzielen \parencites[S.~5]{yuan_moocs_2013}[S.~204]{liyanagunawardena_moocs_2013}.

\sh{Open Educational Resources (OER)}
\textit{Open Educational Resources} (OER) bezeichnen Lehr- und Lernmaterialien, die sowohl frei zugänglich als auch offen lizenziert sind. Sie können ohne Kosten genutzt, angepasst und weiterverbreitet werden und eröffnen damit vielfältige Möglichkeiten zur gemeinsamen Gestaltung von Lernangeboten. Maßgeblich sind hierbei die sogenannten \enquote{5R-Rechte} (\textit{retain, reuse, revise, remix, redistribute}), die den Grad der Offenheit bestimmen \parencite[S.~134f]{wiley_defining_2018}.

\subsubsection{Lerntechnologien und Umgebungen}

\sh{Immersive Technologien}
\textit{Immersive Technologien} fassen Ansätze wie \ac{AR}, \ac{VR} und \ac{MR} zusammen, die es Lernenden ermöglichen, in digitale Umgebungen einzutauchen und dort interaktiv zu agieren. Häufig werden diese Technologien auch unter dem Oberbegriff \ac{XR} diskutiert \parencites[S.~82]{alnagrat_review_2022}[S.~256]{chen_information_2024}. In der Lehre finden immersive Technologien zunehmend Anwendung, da sie Motivation und Interaktivität fördern und hochgradig effektive Lernumgebungen schaffen können \parencite[S.~1]{izouaouen_education_2025}.

\sh{Learning Analytics}
\textit{Learning Analytics} bezeichnet die Erfassung, Sammlung, Analyse und Berichterstattung von Daten über Lernende und deren Kontexte mit dem Ziel, Lernprozesse sowie die Lernumgebungen, in denen sie stattfinden, besser zu verstehen und zu optimieren. Besondere Potenziale ergeben sich aus der Aufdeckung bislang verborgener Informationen in den Daten sowie aus deren gezielter Nutzung, etwa für didaktische Interventionen oder zur Vorhersage von Lernverläufen \parencite[S.~294]{xiao_applying_2019}.

\sh{Virtual Labs}
\textit{Virtuelle Labs} sind computerbasierte, interaktive Umgebungen, die es ermöglichen, Aufgaben auszuführen, die normalerweise in einem physischen Labor stattfinden würden. Über entsprechende Benutzeroberflächen können Simulationen, Animationen und teilweise sogar die Fernsteuerung realer Laborhardware erfolgen. Zahlreiche Studien haben den Einsatz virtueller Labore als Lehr- und Lerninstrument untersucht und ihre Wirksamkeit in nahezu allen Fällen bestätigt \parencite[S.~117]{achuthan_value_2011}.

In den letzten Jahren haben sich virtuelle Labore und Remote-Experimente durch Fortschritte in Webtechnologien und Anwendungen weiterentwickelt. Ziel ist es, die Erfahrungen eines klassischen Präsenzlabors möglichst realitätsnah abzubilden und dabei einen vergleichbaren Grad an Zugriff, Funktionalität und Flexibilität zu gewährleisten. Der Einsatz von virtuellen Welten und Mixed-Reality-Technologien eröffnet zudem neue Möglichkeiten für kollaboratives Arbeiten in immersiven 3D-Umgebungen, in denen Lernende komplexe Simulationen und Datensätze interaktiv erkunden und visualisieren können \parencite[S.~1]{savin-baden_understanding_2012}.

\subsubsection{Didaktische Konzepte zur Motivation}

\sh{Gamification}
\textit{Gamification} bezeichnet den Ansatz, Designelemente aus (Video-)Spielen in nicht-spielerische Kontexte zu übertragen \parencites[S.~2]{deterding_gamification_2011}[S.~9]{kapp_gamification_2012}. Durch den Einsatz solcher Spielelemente sollen Lernprozesse attraktiver gestaltet und Motivation sowie kognitives Engagement gefördert werden. Ein höheres Maß an Involviertheit kann dabei aus hochschuldidaktischer Sicht zu einem aktiveren und nachhaltigeren Lernen führen \parencites[S.~97ff]{active-constructive-interactive_2009}[S.~1821]{chi_translating_2018}.

Der Begriff Gamification geht auf den Software-Entwickler Nick Pelling zurück, der Anfang der 2000er Jahre eine spieleähnliche Benutzeroberfläche für Bank- und Verkaufsautomaten entwarf \parencites{pelling_origins_2011}[S.~2f]{deterding_gamification_2011}. In den folgenden Jahren haben sich neben Gamification auch verwandte Konzepte wie \textit{gamefulness}, \textit{gameful design} oder \textit{playful interaction design} etabliert, die sich teilweise überschneiden, jedoch unterschiedliche Akzentuierungen aufweisen \parencite[S.~2f]{deterding_gamification_2011}. Deterding et al. (2011) haben eine weit verbreitete Definition geprägt, die das Spiel als konstitutive Einheit betont und es von der allgemeinen Spielfreude (\textit{playfulness}) abgrenzt \parencites[S.~2f]{deterding_gamification_2011}[S.~452f]{schlag_gamifizierung_2021}.

Zu den typischen Gestaltungselementen der Gamification zählen unter anderem Punktesysteme, Ranglisten, Abzeichen, Belohnungen und Fortschrittsanzeigen. Sie sollen Lernende motivieren, indem sie Leistung sichtbar machen und Anreize zur weiteren Beschäftigung mit den Inhalten schaffen. Da sich die Spieleindustrie sowie die Anwendungsszenarien im Bildungsbereich kontinuierlich verändern, ist die Auswahl möglicher Spielelemente jedoch nicht abschließend festgelegt, sondern unterliegt einem dynamischen Wandel \parencite[S.~2f]{mari_does_2014}.

\sh{Gamified Learning}
Unter \textit{Gamified Learning} wird der gezielte Einsatz von
Spielelementen in Lehr- und Lernszenarien verstanden, wobei der
didaktische Rahmen im Vordergrund steht. Die Theorie des
gamifizierten Lernens beschreibt dieses Zusammenspiel anhand
mehrerer Dimensionen: der Instruktion selbst, den Einstellungen
und Verhaltensweisen der Lernenden, den eingesetzten Spielelementen
sowie dem daraus resultierenden Lernerfolg \parencites[S.~6f]{landers_developing_2014}[S.~453]{schlag_gamifizierung_2021}.

Gamification kann dabei auf zwei Wegen wirken: Einerseits kann das
Verhalten der Lernenden die Wirksamkeit der Instruktion verstärken
oder abschwächen (moderierender Effekt). Andererseits können die
durch Spielelemente angestoßenen Einstellungen und Verhaltensweisen
selbst zum Lernerfolg beitragen (mediierender Effekt) \parencites[S.~6f]{landers_developing_2014}[S.~453]{schlag_gamifizierung_2021}.

Damit Gamified Learning erfolgreich ist, müssen Spielelemente also
so gestaltet werden, dass sie lernförderliche Verhaltensweisen
hervorrufen und in ein didaktisch sinnvolles Instruktionsdesign
eingebettet sind \parencites[S.~6f]{landers_developing_2014}[S.~453]{schlag_gamifizierung_2021}.

\sh{Game-Based Learning}
\textit{Game-Based Learning} bezeichnet den Einsatz von digitalen
Spielen mit didaktischer Zielsetzung im Lehr- und Lernkontext. Im
deutschsprachigen Raum wird hierfür häufig auch der Begriff
\textit{Serious Games} verwendet, womit digitale, multimediale Spiele
gemeint sind, die nicht primär der Unterhaltung dienen, sondern einen
\enquote{ernsthaften} Hintergrund verfolgen \parencite[S.~14]{niegemann_kompendium_2008}.

\subsubsection{Individuelle \& adaptive Lernformen}

\sh{Microlearning}

\sh{Adaptive Learning}

\sh{Personalized Learning}

\sh{Collaborative Learning}

\subsection{Lernpsychologische Grundlagen}

\sh{Behaviorismus}
"Law of Effect" (deutsch: \textit{Gesetz der Wirkung oder Effektgesetz}) des amerikanischen Psychologen Edward L. Thorndike. Sie zählt zu den Grundlagen wissenschaftlicher Lerntheorien, insbesondere im Bereich des Behaviorismus und der operanten Konditionierung.

\sh{Kognitivismus}

\sh{Konstruktivismus}

