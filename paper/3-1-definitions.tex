\section{Definition und Begrifflichkeiten}

\subsection{Didaktische Simulatoren}
Um den Begriff (didaktischer) \textit{Simulator} zu definieren, muss zunächst das zugrundeliegende Konzept des \textit{Modells} betrachtet werden. White und Ingalls beschreiben ein Modell als eine vereinfachte Abstraktion der Realität, die durch die Auswahl des geeigneten Umfangs und Detaillierungsgrades die relevanten Eigenschaften eines Untersuchungsgegenstands abbildet. Modelle kommen insbesondere dann zum Einsatz, wenn das reale System zu komplex, zu unpraktisch oder zu kostenintensiv wäre, um es direkt zu untersuchen.\parencite[S.~12]{white_introduction_2009}\parencite[S.~5]{banks_what_2008}

Darauf aufbauend stellt die \textit{Simulation} ein Teilgebiet der Modellbildung dar. Unter Simulation versteht man die Durchführung von Experimenten mit einem Modell, das die wesentlichen Eigenschaften des zugrundeliegenden Systems nachahmt, um dessen Verhalten unter verschiedenen Bedingungen untersuchen zu können.\parencite[S.~12]{white_introduction_2009}\parencite[S.~6]{banks_what_2008}

Ein \textit{Simulator} schließlich ist das Werkzeug oder System -- meist in Form von Software --, das diese Simulationen ermöglicht. Er implementiert das Modell und bietet eine Benutzungsumgebung, in der Interaktionen und Experimente mit dem Modell durchgeführt werden können.\parencite[S.~304f]{duran_what_2020}

Die Definition des Begriffs \textit{Simulator} ist eindeutig abzugrenzen von der Bezeichnung \textit{Emulator}. Ein Emulator hingegen strebt eine möglichst detailgetreue Nachbildung eines Zielsystems an, sodass Software oder Peripheriegeräte, die für das Originalsystem entwickelt wurden, unverändert darauf ausgeführt werden können \parencite[S.~1683]{mcgregor_relationship_2002}.

Besonders im Kontext der Lehre kommen sogenannte \textit{didaktische Simulatoren} zum Einsatz. Diese unterscheiden sich von hochgradig präzisen Forschungs- oder Industriesimulatoren dadurch, dass sie in erster Linie auf Verständlichkeit, Visualisierung und Interaktivität ausgerichtet sind. Ziel ist nicht die vollständige, detailgetreue Nachbildung eines Systems, sondern die Förderung von Lernprozessen durch eine für Studierende zugängliche Abstraktion komplexer Sachverhalte.\parencite[S.~256]{muller_entwicklung_2020}\parencite[S.~1]{nystrom_teaching_2024}

\subsection{Konzepte Digitalen Lernens}

\sh{E-Learning}

\sh{M-Learning}

\sh{Blended Learning}

\sh{Blended Learning 2.0}
Blended Learning~2.0 an Bedeutung: Klassische Präsenzformate wurden stärker mit digitalen, hybriden Komponenten verbunden, wodurch flexible, personalisierbare Lernsettings entstanden.\parencite{bonk2020}

\sh{Immersive Technologien}

\sh{Learning Analytics 2.0}

\sh{Gamification}

\sh{Gamified Learning}

"Law of Effect" (deutsch: \textit{Gesetz der Wirkung oder Effektgesetz}) des amerikanischen Psychologen Edward L. Thorndike. Sie zählt zu den Grundlagen wissenschaftlicher Lerntheorien, insbesondere im Bereich des Behaviorismus und der operanten Konditionierung.
