\section{Best Practices und Trends}

Ausgehend von den Erkenntnissen aus Kapitel~\ref{chap:results_lit} und Kapitel~\ref{chap:results_sim} widmet sich dieser Abschnitt der Analyse von Best Practices und Trends. Ziel ist es, Kriterien herauszustellen, die ein didaktischer Simulator im Jahr 2025 nach aktuellen lernpsychologischen Theorien und Forschungsständen der Rechnerarchitektur erfüllen sollte. 

Da in den genannten Kapiteln sowohl wissenschaftliche Publikationen zu didaktischen Simulatoren als auch veröffentlichte Anwendungen selbst untersucht werden, werden die Ergebnisse hier zusammengeführt, um theoretische und praxisnahe Erkenntnisse zu verbinden.

Die Analysen zeigen, dass die Fachrichtung \enquote{Prozessoren und Architekturen}, insbesondere das Thema \enquote{\acs{RISC}}, am häufigsten behandelt wird. Entsprechende Hintergründe werden in Kapitel~\ref{chap:results_lit} erläutert. Auch die Themen \enquote{AI} und \enquote{GPU} verzeichnen in den letzten Jahren eine zunehmende Relevanz. Dicht darauf folgt das Themenfeld \enquote{Speicher und Performance}. Simulatoren, die immersive Technologien thematisieren oder einsetzen, sind zwar selten, wurden jedoch insbesondere in den vergangenen fünf Jahren verstärkt aufgegriffen.

Der Faktor \enquote{Gamification} wird in der Theorie häufig diskutiert, wobei Studien positive Effekte auf Lernende nahelegen. In den analysierten Publikationen und Simulatoren wird er jedoch nur in geringem Umfang berücksichtigt.

Für die schulische und hochschulische Bildung zeigt sich, dass Simulatoren überwiegend didaktisch reduziert gestaltet sind, um den Kernprinzipien der \ac{CTML} zu entsprechen.

Bezüglich der Zugriffsformen (online/offline) ergibt sich kein eindeutiges Ergebnis. Auffällig ist jedoch, dass das Online-Angebot in den vergangenen Jahren deutlich zugenommen hat. Die Wahl der Zugriffsform ist letztlich themenabhängig. Wird ein Simulator offline angeboten, sollte er für alle gängigen Betriebssysteme verfügbar sein. Hinsichtlich der Kosten verdeutlichen die Analysen und die Bedürfnisse von Lehrenden und Lernenden, dass eine kostenfreie Bereitstellung zu bevorzugen ist.

Beim Design eines didaktischen Simulators sind sowohl die Komplexität als auch die Bearbeitungsdauer relevant. Aus lernpsychologischer Sicht führen kleine, klar strukturierte Lerneinheiten zu höheren Lernerfolgen. Simulatoren sollten daher Inhalte in überschaubaren Abschnitten präsentieren und kurze Bearbeitungszeiten ermöglichen.

Darüber hinaus sollten Simulatoren mit begleitender Dokumentation ausgestattet sein, um Lernende und Lehrende auf unterschiedlichen Ebenen zu unterstützen.

Eine Zusammenfassung der Ergebnisse bietet Tabelle~\ref{tab:best_practices}.

\begin{table}[ht]
    \centering
    \caption{Best Practices und Empfehlungen für didaktische Simulatoren der Rechnerarchitektur}
    \label{tab:best_practices}
    \small
    \begin{tabularx}{\textwidth}{p{4cm}X}
        \toprule
        \textbf{Kriterium} & \textbf{Empfehlung/ Best Practice} \\
        \midrule

        Thema & Fokus auf \enquote{Prozessoren und Architekturen}, insbesondere \enquote{RISC}, da dieses Themenfeld am häufigsten adressiert wird. Trends in den Bereichen \enquote{AI}, \enquote{GPU} sowie \enquote{immersive Technologien} sollten berücksichtigt werden. \\

        Gamification & Gamification-Elemente sollten verstärkt implementiert werden, da lernpsychologische Studien positive Effekte zeigen, obwohl diese bisher selten berücksichtigt werden. \\

        Abstraktion & Simulatoren sollten nach den Prinzipien der \ac{CTML} didaktisch reduziert gestaltet sein, um Überlastung zu vermeiden und Lernziele klar zu fokussieren. \\

        Zugriff & Online-Angebote nehmen an Bedeutung zu; die Wahl zwischen Offline-, Online- oder Hybrid-Varianten ist jedoch von der Thematik abhängig. \\

        OS & Offline-Simulatoren sollten für alle gängigen Betriebssysteme verfügbar gemacht werden. \\

        Preis & Kostenfreie Simulatoren sind zu bevorzugen, da sie die Bedürfnisse von Lernenden und Lehrenden bestmöglich unterstützen. \\

        Zeit & Inhalte sollten in kleinen Einheiten vermittelt werden; die Bearbeitungszeit ist entsprechend gering zu halten, um den Lernerfolg zu maximieren. \\

        Dokumentation & Simulatoren sollten mit klaren Anleitungen und begleitender Dokumentation bereitgestellt werden, um Lernende und Lehrende auf unterschiedlichen Ebenen zu unterstützen. \\
        \bottomrule
    \end{tabularx}
\end{table}


