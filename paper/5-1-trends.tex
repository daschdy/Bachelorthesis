\section{Trends und Best Practices}

\TODO{Einleitender Satz}

\subsection{Trends und Best Practices in wissenschaftlichen Publikationen}

Aus Abbildung~\ref{fig:3-anzahl-themen} und Abbildung~\ref{fig:4-top3-themen} wird ersichtlich, dass die Themenbereiche \enquote{Prozessoren und Architekturen} sowie \enquote{Hardware und Logik} in den vergangenen 15 Jahren besonders häufig in wissenschaftlichen Publikationen zu didaktischen Simulatoren der Rechnerarchitektur aufgegriffen wurden.  

Im Themenbereich \enquote{Prozessoren und Architekturen} sind die Subthemen \textit{CPU}, \textit{MIPS}, \textit{Mikroprozessor}, \textit{Prozessor} und \textit{RISC} in annähernd gleicher Häufigkeit vertreten. Auch wenn diese Einordnung zunächst klassische Architekturkonzepte widerspiegelt, zeigt sich insbesondere beim Subthema \textit{RISC} eine aktuelle Relevanz in der Rechnerarchitektur.  

Wie in Kapitel~\ref{chap:3-3-development_ca} beschrieben, erfährt das \ac{RISC}-Paradigma durch die weite Verbreitung von ARM-basierten Systemen sowie durch die zunehmende Bedeutung von energieeffizienten Architekturen eine neue Aktualität. Somit spiegelt sich in den didaktischen Simulatoren ein Themenfeld wider, das auch in den jüngsten Entwicklungen der Rechnerarchitektur von zentraler Bedeutung ist.


Obwohl die Themenbereiche \enquote{AI} und \enquote{GPU} nur einen geringen Anteil der untersuchten Publikationen ausmachen (etwa 5~\%), ist erkennbar, dass entsprechende Simulatoren zu 88~\% in Veröffentlichungen nach 2020 thematisiert werden. Diese Beobachtung lässt sich in den Kontext zentraler Entwicklungen der Rechnerarchitektur einordnen: Große Technologieunternehmen investieren verstärkt in die Forschung zu \ac{AI}, während GPUs im Zuge paralleler Datenverarbeitung zunehmend an Bedeutung gewinnen. Auch in der Entwicklung didaktischer Simulatoren (vgl. Kapitel~\ref{chap:3-2-development-sim}) wird in jüngerer Zeit der Einsatz von \ac{AI} als didaktisches Instrument diskutiert.


