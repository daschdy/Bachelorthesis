\section{Diskussion der Ergebnisse}

Die Analyse der erhobenen Simulatoren zeigt, dass die Mehrheit didaktisch reduziert ist, überwiegend für die Hochschullehre entwickelt wurde und dass Gamification-Ansätze bisher kaum implementiert sind.

Gleichzeitig treten \acs{RISC}-, GPU-, \acs{KI}- und immersive Technologien zunehmend als Themenfelder in der Rechnerarchitektur hervor. Diese Ergebnisse lassen sich in Bezug auf aktuelle Forschungsarbeiten aus den Jahren 2024 und 2025 wie folgt einordnen.

Xu et al.~\cite{xu_present_2025} argumentieren in ihrer bibliometrischen Analyse, dass virtuelle Simulatoren nicht lediglich als ergänzende Lehrmittel, sondern als transformative Elemente des Bildungswesens zu verstehen sind. Sie verorten Simulatoren als strategische Ressource, die über die reine Visualisierung von Inhalten hinausgeht und neue didaktische Praktiken sowie strukturelle Veränderungen im Lehr- und Lernprozess  ermöglicht. Diese Einschätzung korrespondiert mit den in der vorliegenden Arbeit identifizierten Dynamiken in aufkommenden Themenfeldern wie Künstliche Intelligenz und immersiven Technologien (\acs{VR}/ \acs{AR}). Die zuvor genannten Autor:innen heben primär den gesellschaftlichen und bildungspolitischen Stellenwert hervor, der der in der vorliegenden Arbeit bewusst nachrangig behandelt wird. Ein Aspekt, der von den Autor:innen nicht adressiert wird, in der zukünftigen Arbeit jedoch relevant erscheint, betrifft die bildungspolitischen Kostenfragen und die Bedeutung von \ac{OER} für die Hochschullehre. 

Hingegen zeigen Kefalis et al.~\cite{kefalis_digital_2025} in ihrem systematischen Review, dass Simulatoren in unterschiedlichen Bildungsstufen und Lehrformaten flexibel einsetzbar sind. Demgegenüber sind die in dieser Arbeit erfassten Simulatoren stark auf die Hochschulbildung fokussiert und weniger auf die Institutionen Schule und Weiterbildung. In der allgemeinen \acs{STEM} Forschung ist der Einsatz von Simulatoren breiter angelegt. Daher kann weitergehende Forschung die Schulbildung intensiver einschließen und analysieren.

Nyström~\cite{nystrom_teaching_2024} wiederum hebt hervor, dass simulatorgestütztes Lernen nicht nur Wissen vermittelt, sondern auch neue pädagogische Praktiken und Rollen erfordert. Diese Perspektive ergänzt die Ergebnisse insofern, als dass Simulatoren nicht nur technische Hilfsmittel sind, sondern auch die Arbeitsweise von Lehrenden verändern. In der vorliegenden Analyse werden diese Aspekte bislang nicht adressiert -- ein Hinweis darauf, dass künftige Forschung stärker auch organisationale und didaktische Rahmenbedingungen berücksichtigen sollte.

Costabile et al.~\cite{costabile_using_2025} belegen empirisch, dass Simulatoren aktives Lernen fördern und insbesondere leistungsschwächere Studierende stark profitieren. Damit liefern sie einen wichtigen Beleg für die didaktische Wirksamkeit von Simulatoren, den die vorliegende systematische Bestandsaufnahme inhaltlich stützt, aber selbst nicht empirisch überprüft. Für die Lehre der Rechnerarchitektur ergibt sich daraus die Implikation, dass insbesondere heterogene Lerngruppen von didaktisch reduzierten Simulatoren profitieren könnten. Künftige Studien könnten dies durch begleitende Evaluationen in Lehrveranstaltungen empirisch prüfen.

Abschließend berichten Cortinovis et al.~\cite{cortinovis_further_2024} von Ergebnissen mit CPUVSIM, die positive Effekte, aber auch Herausforderungen sichtbar machen. Ihr Plan, qualitative mit quantitativen Methoden zu kombinieren, unterstreicht die Notwendigkeit robuster Wirksamkeitsstudien. Die vorliegende Arbeit zeigt zwar Trends und Best Practices, weist aber ebenfalls auf die Lücke systematischer Evaluationsstudien hin.

Eine Reflexion der im Kapitel~\ref{chap:kriterienkatalog} gewählten Kriterien führt zu folgenden Erkenntnissen: Das Kriterium \enquote{Bekanntheit} der veröffentlichten Simulatoren erweist sich aufgrund seiner Subjektivität als nur eingeschränkt aussagekräftig. Für künftige Arbeiten wäre es daher sinnvoll, eine zusätzliche objektive Quelle – analog zu Google Scholar – heranzuziehen, um die Relevanz von Simulatoren belastbarer zu erfassen. 

Zudem erscheint das Kriterium \enquote{Gamification} erweiterungsbedürftig. Sollte dieser Aspekt künftig stärker an Bedeutung gewinnen, wäre eine Differenzierung in Kategorien wie \enquote{Gamified Learning} und \enquote{Game-Based Learning} sinnvoll. In diesem Zusammenhang rückt auch die Dimension \enquote{Zeit} stärker in den Vordergrund: Bei spielerisch geprägten Lernsimulatoren gewinnt insbesondere die Nutzungsdauer als Indikator für Motivation und Lernintensität an Relevanz.
