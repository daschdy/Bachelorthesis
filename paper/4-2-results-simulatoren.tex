\section{Ergebnisse aus Tabelle~\ref{tab:simulatoren}}

Dieses Kapitel präsentiert die Ergebnisse der Literaturrecherche in Bezug auf bereits veröffentlichte Simulatoren (vgl. Tabelle~\ref{tab:simulatoren}). Auch hier werden die in Kapitel~\ref{chap:kriterienkatalog} definierten Kriterien berücksichtigt. Da sich die zugrunde liegende Auswahl von Simulatoren von der in Kapitel~\ref{chap:results_lit} beschriebenen Grundgesamtheit innerhalb der Kriterien und im Umfang unterscheidet, werden die folgenden Analysen in angepasster Form dargestellt.

\vspace{1em}

Abbildung~\ref{fig:1-anzahl-jahr} verdeutlicht, dass der Großteil der untersuchten Simulatoren zu Beginn der 2000er-Jahre veröffentlicht wurde. Zwischen 2000 und 2020 lassen sich insgesamt 37 Veröffentlichungen identifizieren, was einem Anteil von 74~\% entspricht. Darüber hinaus konnte bei sieben Simulatoren kein Veröffentlichungsdatum ermittelt werden.

Hinsichtlich des Wartungsstands der bereits veröffentlichten Simulatoren zeigt Abbildung~\ref{fig:2-veroeffentlichungen}, dass dieser im Zeitverlauf zunimmt. Etwa 71~\% der Simulatoren wurden in den vergangenen fünf Jahren aktualisiert und können somit als \enquote{aktuell} eingestuft werden.

\begin{figure}[!htbp]
    \centering
    % --- linke Seite: Grafik ---
    \begin{subfigure}[b]{0.48\textwidth}
        \centering
        \includegraphics[width=0.90\textwidth]{graphics_sim/1-anzahl-jahr.png}
        \caption{Jährliche Übersicht der Veröffentlichungen}
        \label{fig:1-anzahl-jahr}
    \end{subfigure}
    \hfill
    % 
    % --- rechte Seite: Grafik ---
    \begin{subfigure}[b]{0.48\textwidth}
        \centering
        \includegraphics[width=0.90\textwidth]{graphics_sim/2-veroeffentlichung.png}
        \caption{Jährliche Übersicht des Wartungsstands}
        \label{fig:2-veroeffentlichungen}
    \end{subfigure}
    %
    \caption{Analysen zum Jahr der Veröffentlichung und zum Wartungsstand}
    \label{fig:veroeffentlichung_wartungsstand}
\end{figure}

Die thematische Verteilung der Simulatoren ist Abbildung~\ref{fig:7-thema} zu entnehmen. Etwa 40~\% der Simulatoren befassen sich mit dem Themenbereich \enquote{Prozessoren und Architekturen}, gefolgt von 18~\% hardwarebezogenen und 14~\% grundlagenvermittelnden Simulatoren. Die übrigen Anwendungen verteilen sich auf die Themenbereiche \enquote{Speicher und Performance}, \enquote{Programmierung}, \enquote{Systeme und Anwendungen}, \enquote{GPU} sowie \enquote{Monitoring}.

Die Zuordnung der Veröffentlichungsjahre zu den Zeiträumen \enquote{vor 2000}, \enquote{2000--2010}, \enquote{2010--2020} und \enquote{nach 2020} ist in Abbildung~\ref{fig:8-thema-jahr} dargestellt. Hier wird deutlich, dass die Mehrheit der Simulatoren zum Themenbereich \enquote{Prozessoren und Architekturen} im Zeitraum 2010 bis 2020 veröffentlicht wurde.

\begin{figure}[!htbp]
    \centering
    % --- linke Seite: Grafik ---
    \begin{subfigure}[b]{0.48\textwidth}
        \centering
        \includegraphics[width=0.90\textwidth]{graphics_sim/7-thema.png}
        \caption{Übersicht der Themenverteilung}
        \label{fig:7-thema}
    \end{subfigure}
    \hfill
    % 
    % --- rechte Seite: Grafik ---
    \begin{subfigure}[b]{0.48\textwidth}
        \centering
        \includegraphics[width=0.90\textwidth]{graphics_sim/8-thema-jahr.png}
        \caption{Jährliche Übersicht der Themenverteilung}
        \label{fig:8-thema-jahr}
    \end{subfigure}
    %
    \caption{Analysen zur Themenverteilung}
    \label{fig:themen-gesamt}
\end{figure}

Für den Einsatz didaktischer Simulatoren stellt die Zugriffsart ein wesentliches Kriterium dar. Daher wurde diese in Abbildung~\ref{fig:3-zugriff} dargestellt. Den größten Anteil bilden offline nutzbare Simulatoren (53~\%). Etwa 11~\% der Simulatoren sind sowohl online als auch offline verfügbar, während 37~\% ausschließlich online zugänglich sind.

Abbildung~\ref{fig:4-zugriff-jahr} verdeutlicht, wie sich die Zugriffsarten im Zeitverlauf entwickelt haben.

\begin{figure}[!htbp]
    \centering
    % --- linke Seite: Grafik ---
    \begin{subfigure}[b]{0.48\textwidth}
        \centering
        \includegraphics[width=0.90\textwidth]{graphics_sim/3-zugriff.png}
        \caption{Übersicht der Zugriffsarten}
        \label{fig:3-zugriff}
    \end{subfigure}
    \hfill
    % 
    % --- rechte Seite: Grafik ---
    \begin{subfigure}[b]{0.48\textwidth}
        \centering
        \includegraphics[width=0.90\textwidth]{graphics_sim/4-zugriff-jahr.png}
        \caption{Jährliche Übersicht der Zugriffsarten}
        \label{fig:4-zugriff-jahr}
    \end{subfigure}
    %
    \caption{Analysen zur Zugriffsart}
    \label{fig:zugriff-gesamt}
\end{figure}

Bezogen auf das Kriterium \ac{OS} wird keine Abbildung dargestellt, da einzelne Simulatoren mehreren Betriebssystemen zugeordnet werden können. Stattdessen zeigt Tabelle~\ref{tab:os} die Verteilung der Simulatoren auf die jeweiligen Betriebssysteme. Auffällig ist, dass der größte Anteil als \enquote{unabhängig} klassifiziert ist. Diese Simulatoren sind online verfügbar und damit nicht an ein spezifisches Betriebssystem gebunden. In der Kategorie \enquote{unabhängig} wurden rund 19~\% der Simulatoren im Zeitraum 2010 bis 2020 veröffentlicht, weitere 14~\% in den Jahren 2020 bis 2025.

\begin{table}[h]
	\centering
	\caption{Verteilung der unterstützten Betriebssysteme}
	\label{tab:os}
	\begin{tabular}{l r}
		\toprule
		\textbf{Betriebssystem(e)} & \textbf{Anzahl} \\
		\midrule
		Linux                     & 4  \\
		Linux, Windows            & 1  \\
		Linux, Windows, macOS     & 11 \\
		Windows                   & 1  \\
		Windows, VM               & 1  \\
		VM                        & 2  \\
		unabhängig                & 37 \\
		\bottomrule
	\end{tabular}
\end{table}

Im Rahmen der Simulatorrecherche wird auch das Abstraktionsniveau erfasst (vgl. Abbildung~\ref{fig:5-abstraktion}). Der überwiegende Teil (ca. 72~\%) der Simulatoren ist dabei als \enquote{didaktisch reduziert} zu klassifizieren. Diese Simulatoren verteilen sich gemäß Abbildung~\ref{fig:6-abstraktion-thema} im Wesentlichen auf die Themenbereiche \enquote{Prozessoren und Architekturen} sowie \enquote{Hardware und Logik}.

\begin{figure}[!htbp]
    \centering
    % --- linke Seite: Grafik ---
    \begin{subfigure}[b]{0.48\textwidth}
        \centering
        \includegraphics[width=0.90\textwidth]{graphics_sim/5-abstraktion.png}
        \caption{Übersicht der Abstraktionslevel}
        \label{fig:5-abstraktion}
    \end{subfigure}
    \hfill
    % 
    % --- rechte Seite: Grafik ---
    \begin{subfigure}[b]{0.48\textwidth}
        \centering
        \includegraphics[width=0.90\textwidth]{graphics_sim/6-abstraktion-thema.png}
        \caption{Übersicht der Abstraktionslevel pro Thema}
        \label{fig:6-abstraktion-thema}
    \end{subfigure}
    %
    \caption{Analysen zum Abstraktionslevel}
    \label{fig:abstraktion-gesamt}
\end{figure}

Neben dem Abstraktionsniveau wird auch die Zielgruppe der jeweiligen Simulatoren betrachtet. Dabei erfolgt eine Unterscheidung zwischen den Kategorien \enquote{Schule}, \enquote{Hochschule} sowie \enquote{Forschung, Beruf}. Da eine Anwendung potenziell für mehrere Zielgruppen relevant sein kann und somit mehreren Kategorien zugeordnet wird, wird auf eine grafische Darstellung verzichtet. Tabelle~\ref{tab:institutionen} bietet stattdessen eine Übersicht über die Verteilung der Simulatoren auf die jeweiligen Institutionstypen.

\begin{table}[h]
	\centering
	\caption{Verteilung der Institutionstypen der Simulatoren}
	\label{tab:institutionen}
	\begin{tabular}{l r}
		\toprule
		\textbf{Institution(en)} & \textbf{Anzahl} \\
		\midrule
		Schule                        & 7  \\
		Schule, Hochschule            & 2  \\
		Hochschule                    & 31 \\
		Hochschule, Forschung, Beruf  & 14 \\
		Forschung, Beruf              & 3  \\
		\bottomrule
	\end{tabular}
\end{table}

Hinsichtlich des Kriteriums \textit{Gamification} ist festzuhalten, dass die in Tabelle~\ref{tab:simulatoren} aufgeführten Simulatoren keine spielerischen Elemente enthalten und dieses Kriterium daher nicht weiter berücksichtigt werden kann.  

Der überwiegende Teil der Simulatoren ist kostenfrei verfügbar; lediglich 5~\% sind kostenpflichtig. Eine Übersicht der Verteilung zeigt Abbildung~\ref{fig:9-preis}.

\begin{figure}[!htbp]
    \centering
    \includegraphics[width=0.90\textwidth]{graphics_sim/9-preis.png}
    \caption{Preisverteilung der veröffentlichtten Simulatoren}
    \label{fig:9-preis}
\end{figure}

Da in dieser Arbeit überwiegend didaktische Simulatoren vorgestellt werden, ist das Kriterium \enquote{Vorwissen} von besonderem Interesse. Etwa 77~\% der Simulatoren setzen Grundkenntnisse in den jeweiligen Themengebieten voraus (vgl. Abbildung~\ref{fig:10-vorwissen}). Rund 5~\% erfordern fortgeschrittene Kenntnisse, während lediglich 18~\% ohne Vorkenntnisse genutzt werden können. Die Verteilung nach Themengebieten in Verbindung mit dem erforderlichen Vorwissen ist in Abbildung~\ref{fig:11-vorwissen-thema} dargestellt.

\begin{figure}[!htbp]
    \centering
    % --- linke Seite: Grafik ---
    \begin{subfigure}[b]{0.48\textwidth}
        \centering
        \includegraphics[width=0.90\textwidth]{graphics_sim/10-vorwissen.png}
        \caption{Verteilung Vorwissen}
        \label{fig:10-vorwissen}
    \end{subfigure}
    \hfill
    % 
    % --- rechte Seite: Grafik ---
    \begin{subfigure}[b]{0.48\textwidth}
        \centering
        \includegraphics[width=0.90\textwidth]{graphics_sim/11-vorwissen-thema.png}
        \caption{Verteilung Vorwissen auf Themen}
        \label{fig:11-vorwissen-thema}
    \end{subfigure}
    %
    \caption{Analysen Vorwissen}
    \label{fig:vorwissen-gesamt}
\end{figure}

Die Abbildungen~\ref{fig:12-zeit} und \ref{fig:13-vorwissen-thema} geben Aufschluss über die Nutzungsdauer der untersuchten Simulatoren. Der überwiegende Teil weist eine kurze Nutzungszeit auf, während lediglich 2~\% eine längere Nutzungsdauer vorsehen. In den Themenbereichen \enquote{Prozessoren und Architekturen} sowie \enquote{Hardware und Logik} finden sich ausschließlich Simulatoren mit kurzer Nutzungszeit.

\begin{figure}[!htbp]
    \centering
    % --- linke Seite: Grafik ---
    \begin{subfigure}[b]{0.48\textwidth}
        \centering
        \includegraphics[width=0.90\textwidth]{graphics_sim/12-zeit.png}
        \caption{Verteilung Zeit}
        \label{fig:12-zeit}
    \end{subfigure}
    \hfill
    % 
    % --- rechte Seite: Grafik ---
    \begin{subfigure}[b]{0.48\textwidth}
        \centering
        \includegraphics[width=0.90\textwidth]{graphics_sim/13-zeit-thema.png}
        \caption{Verteilung Zeit auf Themen}
        \label{fig:13-vorwissen-thema}
    \end{subfigure}
    %
    \caption{Analysen zur Nutzungsdauer}
    \label{fig:nutzungsdauer-gesamt}
\end{figure}

86~\% der Simulatoren haben eine Dokumentation (siehe dazu Abbildung~\ref{fig:14-dokumentation}).

\begin{figure}[!htbp]
    \centering
    \includegraphics[width=0.90\textwidth]{graphics_sim/14-dokumentation.png}
    \caption{Verteilung der Dokumentation}
    \label{fig:14-dokumentation}
\end{figure}

Als abschließendes Kriterium wird die Bekanntheit der Simulatoren betrachtet. Insgesamt werden 78~\% der Simulatoren als mittel oder hoch bekannt eingestuft (vgl. Abbildung~\ref{fig:15-bekanntheit}). Eine Übersicht darüber, welche Themenbereiche vergleichsweise geringere Bekanntheit aufweisen, bietet Abbildung~\ref{fig:16-bekanntheit-thema}.

\begin{figure}[!htbp]
    \centering
    % --- linke Seite: Grafik ---
    \begin{subfigure}[b]{0.48\textwidth}
        \centering
        \includegraphics[width=0.90\textwidth]{graphics_sim/15-bekanntheit.png}
        \caption{Verteilung Bekanntheit}
        \label{fig:15-bekanntheit}
    \end{subfigure}
    \hfill
    % 
    % --- rechte Seite: Grafik ---
    \begin{subfigure}[b]{0.48\textwidth}
        \centering
        \includegraphics[width=0.90\textwidth]{graphics_sim/16-bekanntheit-thema.png}
        \caption{Verteilung Bekanntheit auf Themen}
        \label{fig:16-bekanntheit-thema}
    \end{subfigure}
    %
    \caption{Analysen zur Bekanntheit}
    \label{fig:bekanntheit-gesamt}
\end{figure}
